\documentclass[9pt,a4paper, twocolumn]{article}


\usepackage[utf8]{inputenc}
\usepackage[T1]{fontenc}
\usepackage{amsmath}
\usepackage{amsfonts}
\usepackage{amssymb}
\usepackage{url}
\usepackage{makeidx}
\usepackage{graphicx}
\usepackage{graphicx, adjustbox}
\usepackage{lmodern}
\usepackage{fourier}
\usepackage{float}
\usepackage{caption}
\usepackage{wrapfig}
\usepackage{mhchem}
\usepackage{multicol}
\usepackage{soul}


\usepackage[top = 1cm, bottom = 1cm, left = 0.7cm, right = 0.7cm]{geometry}



%Colors
\usepackage[dvipsnames]{xcolor}


\definecolor{black}{RGB}{0, 0, 0}
\definecolor{richblack}{RGB}{7, 14, 13}
\definecolor{charcoal}{RGB}{45, 67, 77}
\definecolor{delectricblue}{RGB}{93, 117, 131}
\definecolor{cultured}{RGB}{245, 245, 245}
\definecolor{lightgray}{RGB}{211, 216, 218}
\definecolor{silversand}{RGB}{190, 194, 198}
\definecolor{spanishgray}{RGB}{148, 150, 157}
\definecolor{darkliver}{RGB}{64, 63, 76}

\colorlet{lightdelectricblue}{delectricblue!30}
\colorlet{lightdarkliver}{darkliver!30}


%ColorDefines
\newcommand{\trueblack}[1]{\textcolor{black}{#1}}
\newcommand{\rich}[1]{\textcolor{richblack}{#1}}
\newcommand{\lightblack}[1]{\textcolor{charcoal}{#1}}
\newcommand{\lightrich}[1]{\textcolor{delectricblue}{#1}}


%Boxes
\usepackage{tcolorbox}
\newtcolorbox{calloutbox}{center,%
    colframe =red!0,%
    colback=cultured,
    title={Callout},
    coltitle=richblack,
    attach title to upper={\ ---\ },
    sharpish corners,
    enlarge by=0.5pt}

\newtcolorbox[use counter=equation]{eq}{center,
	colframe =red!0,
	colback=cultured,
	title={\thetcbcounter},
	coltitle=richblack,
	detach title,
	after upper={\par\hfill\tcbtitle},
	sharpish corners,
    enlarge by=0.5pt }
    
\newtcolorbox{qt}{center,
	colframe=delectricblue,
	colback=white!0,
	title={\large "},
	coltitle=delectricblue,
	attach title to upper,
	after upper ={\large "},
	sharp corners,
	enlarge by=0.5pt,
	boxrule=0pt,
	leftrule=2pt}
	
\newtcolorbox{exc}{center,%
    colframe =red!0,%
    colback=darkliver!15,
    title={Excercise},
    coltitle=richblack,
    attach title to upper={\ ---\ },
    sharpish corners,
    enlarge by=0.5pt}
    
\newcounter{theo}
\newtcolorbox[use counter=theo]{theorem}
	{center,%
    colframe =red!0,%
    colback=cultured,
    title={Theorem \thetcbcounter},
    coltitle=richblack,
    attach title to upper={\ ---\ },
    sharpish corners,
    enlarge by=0.5pt}

\newcounter{defcounting}
\newtcolorbox[use counter=defcounting]{define}
{center,%
	colframe=darkliver!50,%
	colback=white!0,
	title={\textcolor{black}{\textbf{\textbf{Definition}} \  \thetcbcounter  \ --}},
	coltitle=darkliver!50,
	attach title to upper,
	after upper ={ },
	sharp corners,
	enlarge by=0.5pt,
	boxrule=0pt,
	leftrule=2pt,
    rightrule = 0pt}

\newcounter{lemmacount}
\newtcolorbox[use counter=lemmacount]{lemma}
{center,%
    colframe=charcoal!50,%
    colback=white!0,
    title={\textcolor{black}{\textbf{\textit{Lemma}} \  \thetcbcounter  \ --}},
    coltitle=darkliver!50,
    attach title to upper,
    after upper ={ },
    sharp corners,
    enlarge by=0.5pt,
    boxrule=2pt}
    

\newcounter{examplecounter}
\newtcolorbox[use counter=examplecounter]{example}
	{center,%
    colframe =red!0,%
    colback=cultured,
    title={Example},
    coltitle=richblack,
    attach title to upper={\ ---\ },
    sharpish corners,
    enlarge by=0.5pt}

    

        
    
% Highlighters
\newcommand{\hldl}[1]{%
	\sethlcolor{lightdarkliver}%
	\hl{#1}
}
\newcommand{\hldb}[1]{%
    \sethlcolor{lightdelectricblue}%
    \hl{#1}%
}


% Images
\newcounter{figurecounter}
\setcounter{figurecounter}{1}

\newcommand{\img}[3]{
    \begin{figure}[h!]
        \centering
        \captionsetup{justification=centering,margin=0cm,labelformat=empty}
        \includegraphics[width=#2\linewidth]{./img/#1}
        \label{figure}
        \caption{\small\textbf{fig-\thefigurecounter} -- \textcolor{darkliver}{#3}}
    \end{figure}
    \addtocounter{figurecounter}{1}}

\newcommand{\imgr}[3]{
    \begin{wrapfigure}{r}{#2\textwidth}
        \centering
        \captionsetup{justification=centering,margin=0cm,labelformat=empty}
        \includegraphics[width=\linewidth]{./img/#1}
        \label{figure}
        \caption{\small \textbf{fig: \thefigurecounter} -- \textcolor{darkliver}{#3}}
    \end{wrapfigure}
    \addtocounter{figurecounter}{1}}

\newcommand{\imgl}[3]{
    \begin{wrapfigure}{l}{#2\textwidth}
        \centering
        \captionsetup{justification=centering,margin=0cm,labelformat=empty}
        \includegraphics[width=\linewidth]{./img/#1}
        \label{figure}
        \caption{\small \textbf{fig: \thefigurecounter} -- \textcolor{darkliver}{#3}}
    \end{wrapfigure}
    \addtocounter{figurecounter}{1}}

% New commands
\newenvironment{callout}
	{\begin{calloutbox}\color{charcoal}\textbf\textit}
	{\end{calloutbox}}

% for this file
\newcommand{\newpoint}[1]{\ \\ \indent$\mathsection$ \textbf{#1}}
\newcommand{\curveL}{\mathcal{L}}
\newcommand{\curveA}{\mathcal{A}}
\newcommand{\curveP}{\mathcal{P}}
\newcommand{\thm}{\text{Thm}}
\newcommand{\proof}{\\ \ \\ $\blacktriangleright$ \textit{proof: }}
\newcommand{\distinct}{ \\ \hrule}


\title{Lectures On Mathematical Logic\\ \large Lecture One - Languages and Structures}
\date{\today}
\author{Amir H. Ebrahimnezhad \\ \small \textit{University of Tehran Department of Physics.}}

\parskip=12pt % adds vertical space between paragraphs


\begin{document}
    \maketitle
    \section*{Preface}
    The Lecture notes before you are followed from courses and books that have been read by me. I would try to work and update this every once in a while. This edition is updated until \today. For any question or corrections please contact me \textit{Thisismeamir@outlook.com}.
    \tableofcontents
    \newpage
    \section{Formal Languages}
        In logic, a formal language consists of words whose letters are taken from an alphabet and are well-formed according to a specific set of rules. In this context we construct a restricted formal language, and our goal is to be able to form a certain statements about certain kinds of mathematical structures.
        \begin{define}
            A \textbf{First-Order Language} $\curveL$ is an infinite collection of distinct symbols, no one of which is properly contained in another, separated into the following categories:
            \begin{enumerate}
                \item Parantheses: $(,)$.
                \item Connectives: $\lor, \neg$
                \item Quantifier: $\forall$
                \item Variables: One for each natural number $\rightarrow v_n$
                \item Equality symbol: $=$
                \item Constant symbols: Some set of zero or more symbols.
                \item Function symbols: For each positive integer $n$, some set of zero or more $n$-ary functions symbols.
                \item Relation symbols: For each positive integer $n$, some set of zero or more $n$-ary relation symbols.
            \end{enumerate}
        \end{define}
        \begin{callout}
            To say a function or a relation is $n$-ary, means that it is intended to represent a function or relation of $n$ variables.
        \end{callout}
        \begin{callout}
            To specify a language, all we have to do is to determine which, if any, constant, function, and relation symbols we wish to use.
        \end{callout}
        \newpoint{Formulas and Terms:} Suppose a language $\curveL = \left\{v, + , <\right\}$. Any combination of these elements are said to be a word in that language, but not all the combinations are pleasing. For instance:
        $$
        v + v < (v + v) + v
        $$
        and 
        $$
        v ++<v
        $$
        are both words in the laguage we made but one would argue that the latter has no meaning and thus is of no interest for us. Therefore we would start defining some meaningful words for our language:
        \begin{define}
            If $\curveL$ is a language, \textbf{a term of $\curveL$} is a nonempty finite string $t$ of symbols from $\curveL$ such that either:
            \begin{enumerate}
                \item $t$ is a variable, or
                \item $t$ is a constant symbol, or
                \item $t:\equiv ft_1t_2t_3\dots t_n$ where $f$ is an $n$-ary function symbol of $\curveL$ and each of the $t_i$ is a term of $\curveL$
            \end{enumerate}
            \
        \end{define}
        \begin{callout}
            The symbol $:\equiv$ is not a part of strings of the language $\curveL$. Rather it is a meta-linguistic symbol that means that the strings of $\curveL$-symbols on each side of it are identical.
        \end{callout}
        The terms of $\curveL$ play the role of the nuons of the language. To make meaningful mathematical statements about some mathematical structure, we will want to be able to make assertions about the objects of the structure. These assertions will be the formulas of $\curveL$:
        \begin{define}
            If $\curveL$ is a first-order language, \textbf{a formula of $\curveL$} is a noneempty finite string $\phi$ of symbols from $\curveL$ such that either:
            \begin{enumerate}
                \item $\phi:\equiv = t_1t_2$ where $t_i$ is a term in $\curveL$, or
                \item $\phi:\equiv Rt_1t_2\dots t_n$ where $R$ is an $n$-ary relation and $t_i$s are terms in $\curveL$, or
                \item $\phi :\equiv (\neg\alpha)$ where $\alpha$ is a formula of $\curveL$, or
                \item $\phi:\equiv (\alpha\lor\beta)$, where $\alpha, \beta$ are formulas of $\curveL$
                \item $\phi:\equiv (\forall v)(\alpha)$, where $v$ is a variable and $\alpha$ is a formula of $\curveL$.
            \end{enumerate}
        \end{define}
        If a formula $\psi$ contains the subformula $\forall v \alpha$ [meaning that the string of symbols that constitute the formula is a substring of the string of symbols that make up $\psi$], we will say that the \textbf{scope} of the quantifier $\forall$ is $\alpha$. Any symbols in $\alpha$ will be said to lie within the scope of the quantifier $\forall$.
        \begin{callout}
            Notice that a firmula can have several different occurrences of the symbol $\forall$, and each occurrence of the quantifier will have its own scope.
        \end{callout}
        The \textbf{Atomic Formulas of $\curveL$} are those formulas that satisfy clause (1) or (2) of the definition. Also note that the terms of a language are not formulas. In addition to the similarities between terms and nouns, formulas are statements in the language. Statements unlike nouns can be true or false. 
        \newpoint{Induction:} We already used induction in many high school mathematics. It is now our goal to discuss the proofs by induction using the language, formulas, and terms we learned and generalize that notion of induction to a setting that will allow us to use induction to prove things about terms and formulas rather than just the natural numbers:
        \\
        \\
        In the inductive step of the proof, we show the following implication:
        \begin{center}
            If the formula holds for $k$, then the formula holds for $k+1$
        \end{center}
        We prove this implication by assuming the antecedent, that the theorem holds for a number $k$, and from that assumption we show that it also holds for $k+1$. Notice that the formula itself is universal, means it is for all $k$s but here we only assume a certain $k$, therefore we are not assuming the theorem itself. Looking at this from another way we've shown that for a set $S$ of numbers that holds the theorem true, there is these facts:
        \begin{enumerate}
            \item The number $1$ is an element of $S$. We prove this explicitly in the base case of the proof.
            \item If the number $k$ is an element of $S$, then the number $k+1$ is also an element of $S$. (This is the inductive step)
        \end{enumerate}
        But now, notice that we know that the collection of natural numbers can be defined as the smallest set such that the facts above implise, but it is not always the largest. So $S$, the collection of numbers which the theorem holds, is identical with the set of natural numbers, thus the theorem holds for every natural number $n$, as needed. So what makes the proof by induction work is that the natural numbers can be defined recursively. This is an important thing! Looking at the definition of the formula the first two clauses has the role of (1)st fact about the set $S$ we had here. They are explicitly defined. The last three clauses are the recursive cases, showing how if $\alpha, \beta$ are a formula, we can generate more formulas with those. Now since the collection of formulas is defined recursivelym we can use an inductive-style proof when we want to prove that something is true about every formula! The proof is consist of showing that (1) The theorem holds for every atomic formula. (2) In the inductive phase of the proof, we assume that the theorem is true about simple formulas ($\alpha,\beta$) amd use that assumption to prove that the theorem holds a more complicated formula $\phi$ that is generated by a recursive clause of the definition. This method of proof is called \textit{Induction on the complexity of the formula,} or \textit{Induction on the structure of the formula.}
        
