\documentclass[10pt,a4paper]{article}
\usepackage[utf8]{inputenc}
\usepackage{amsmath}
\usepackage{amsfonts}
\usepackage{amssymb}
\usepackage{url}
\usepackage{makeidx}
\usepackage{graphicx}
\usepackage{graphicx, adjustbox}
\usepackage{lmodern}
\usepackage{fourier}
\usepackage{float}
\usepackage{caption}
\usepackage{wrapfig}
\usepackage{mhchem}
\usepackage[left=2.5cm,right=2.5cm,top=1cm,bottom=2cm]{geometry}
\usepackage{multicol}
\usepackage{soul}
%Colors
\usepackage[dvipsnames]{xcolor}


\definecolor{black}{RGB}{0, 0, 0}
\definecolor{richblack}{RGB}{7, 14, 13}
\definecolor{charcoal}{RGB}{45, 67, 77}
\definecolor{delectricblue}{RGB}{93, 117, 131}
\definecolor{cultured}{RGB}{245, 245, 245}
\definecolor{lightgray}{RGB}{211, 216, 218}
\definecolor{silversand}{RGB}{190, 194, 198}
\definecolor{spanishgray}{RGB}{148, 150, 157}
\definecolor{darkliver}{RGB}{64, 63, 76}

\colorlet{lightdelectricblue}{delectricblue!30}
\colorlet{lightdarkliver}{darkliver!30}


%ColorDefines
\newcommand{\trueblack}[1]{\textcolor{black}{#1}}
\newcommand{\rich}[1]{\textcolor{richblack}{#1}}
\newcommand{\lightblack}[1]{\textcolor{charcoal}{#1}}
\newcommand{\lightrich}[1]{\textcolor{delectricblue}{#1}}
\newcommand{\liver}[1]{\textcolor{darkliver}{#1}}

%Boxes
\usepackage{tcolorbox}
\newtcolorbox{calloutbox}{center,%
    colframe =red!0,%
    colback=cultured,
    title={Callout},
    coltitle=richblack,
    attach title to upper={\ ---\ },
    sharpish corners,
    enlarge by=0.5pt}

\newtcolorbox[use counter=equation]{eq}{center,
	colframe =red!0,
	colback=cultured,
	title={\thetcbcounter},
	coltitle=richblack,
	detach title,
	after upper={\par\hfill\tcbtitle},
	sharpish corners,
    enlarge by=0.5pt }
    
\newtcolorbox{qt}{center,
	colframe=delectricblue,
	colback=white!0,
	title={\large "},
	coltitle=delectricblue,
	attach title to upper,
	after upper ={\large "},
	sharp corners,
	enlarge by=0.5pt,
	boxrule=0pt,
	leftrule=2pt}
	
\newtcolorbox{lecturequote}{center,%
    colframe =red!0,%
    colback=darkliver!15,
    title={In Lecture \thetcbcounter},
    coltitle=richblack,
    attach title to upper={\ ---\ },
    sharpish corners,
    enlarge by=0.5pt}
    
\newcounter{theo}
\newtcolorbox[use counter=theo]{theobox}
	{center,%
    colframe =red!0,%
    colback=cultured,
    title={Theorem \thetcbcounter},
    coltitle=richblack,
    attach title to upper={\ ---\ },
    sharpish corners,
    enlarge by=0.5pt}

\newcounter{examplecounter}
\newtcolorbox[use counter=examplecounter]{example}
	{center,%
    colframe =red!0,%
    colback=cultured,
    title={Example \thetcbcounter},
    coltitle=richblack,
    attach title to upper={\ ---\ },
    sharpish corners,
    enlarge by=0.5pt}

    

        
    
% Highlighters
\newcommand{\hldl}[1]{%
	\sethlcolor{lightdarkliver}%
	\hl{#1}
}
\newcommand{\hldb}[1]{%
    \sethlcolor{lightdelectricblue}%
    \hl{#1}%
}


% Images
\newcounter{figurecounter}
\setcounter{figurecounter}{1}

\newcommand{\img}[3]{
    \begin{figure}[h!]
        \centering
        \captionsetup{justification=centering,margin=0cm,labelformat=empty}
        \includegraphics[width=#2\linewidth]{./img/#1}
        \label{figure}
        \caption{\small\textbf{fig: \thefigurecounter} -- \textcolor{darkliver}{#3}}
    \end{figure}
    \addtocounter{figurecounter}{1}}

\newcommand{\imgr}[3]{
    \begin{wrapfigure}{r}{#2\textwidth}
        \centering
        \captionsetup{justification=centering,margin=0cm,labelformat=empty}
        \includegraphics[width=\linewidth]{./img/#1}
        \label{figure}
        \caption{\small \textbf{fig: \thefigurecounter} -- \textcolor{darkliver}{#3}}
    \end{wrapfigure}
    \addtocounter{figurecounter}{1}}

\newcommand{\imgl}[3]{
    \begin{wrapfigure}{l}{#2\textwidth}
        \centering
        \captionsetup{justification=centering,margin=0cm,labelformat=empty}
        \includegraphics[width=\linewidth]{./img/#1}
        \label{figure}
        \caption{\small \textbf{fig: \thefigurecounter} -- \textcolor{darkliver}{#3}}
    \end{wrapfigure}
    \addtocounter{figurecounter}{1}}

% New commands
\newenvironment{callout}
	{\begin{calloutbox}\color{charcoal}\textbf\textit}
	{\end{calloutbox}}

\newcommand{\mev}{\text{MeV}}
\newcommand{\fpe}{4\pi\epsilon_0}
\newcommand{\ch}[5]{{}^{#2}_{#3}\!\text{#1}^{#4}_{#5}}
\newcommand{\electron}{\ch{e}{}{}{-}{}}
\newcommand{\positron}{\ch{e}{}{}{+}{}}
\newcommand{\proton}{\ch{p}{}{}{}{}}
\newcommand{\neutron}{\ch{n}{}{}{}{}}

\title{Nuclear Physics \\ \large Journal of Study}
\author{Amir H. Ebrahimnezhad}
\parskip=12pt % adds vertical space between paragraphs

\begin{document}

    \maketitle

\newpage
\tableofcontents
\newpage
\section{Introduction}
The fundamental constituents of matter was always a question for the great minds of each era. From the greek philosophers that tried to explain the matter using four fundamental ones, to the recent years that the standard model tries to explain what are the interactions between the fundamental particles that form up the world around us.

The physicists have found smaller and smaller particles that make up the matter around us. The atom, the substructure of the atom, electrons, nuclei, The substructure of nuclei, protons and neutrons, and the substructure of protons and neutrons which are quarks. 

The two fundamental types of building blocks are the \textit{leptons}, which include the electron and the neutrino, and the quarks. In the scattering experiments, these were found to be smaller than $10^{-18}$m. They are possibly point-like particles.  For comparison, protons are as large as $10^{-15}$m. 

By the end of 19th-century there were two known forces in the world. Gravity and electromagnetism, the first described all the observed motions in the night sky and the macroscopic objects around, while the latter explained the interactions causing chemical reactions. With the development of nuclear physics two new forces were introduced. The reason behind the absence of these forces was the short-range nature of them.

Thus the four fundamental interactions on which all physical interactions are based upon are:
\begin{itemize}
     \item Gravitation
     \item Electromagnetic interaction
     \item Weak interaction
     \item Strong interaction
\end{itemize}

According the today's conceptions, we tend to assign a mediator for each interaction. Thus except from gravitation which is still a mistery in fundamental physics we have:

\begin{itemize}
     \item Weak interactions mediators are \textbf{W-Boson} and \textbf{Z-Boson}.
     \item Strong interactions mediator is \textbf{Gluon}
     \item Electromagnetic interactions mediatos is the photon.
\end{itemize}

The question to be raised is that why two electrons share electromagnetic interactions while a photon and an electron would not. This would raise the assumption that electrons should have a property that is only to the particles, which can share such interactions. Thus we would associate a property named charge for any interactions. For electromagnetic interactions an \textit{electric} charge, for weak and strong we would also have weak and strong charges. The latter is also called colour charge or colour for short.

\begin{callout}
     A particle is subjected to an interaction if and only if it carries the corresponding charge!
\end{callout}

\section{The Building Blocks of Matter}
\subsection{Atomic Theory and The Constituents of Matter}
People have questioned the very nature of matte for a long time, whether it is possible to divide the matter infinite times and still be able to do so, or if it is made of a grane-like particles which cannot be divided into substructures. The question was an open one, though philosophers like Democritus or physicists like Boltzmann believed in an atomic (atomos means uncuttable in greek.) theory along the way. 

\textbf{The Philosophy:} The idea was simple, matter is made up of discrete units that cannot be more divided into substructures. Democritus came out to be seen as the first person, who believed in such idea. But it was well forgotten until the newly found works of Aristotle in 12th century which was exactly the opposite. That the matter was continuous and infinite and could be subdivied without limit.

Later on the discovery of major works describing atomist teachings gave more attention to the subject, but since the Epicureanism contradicted orthodox Christian teachings, The teachings was considered unacceptable by most philosophers. The French Catholic priest Pierre Gassendi revived the theory by some modification, that the God created the atoms.

\textbf{John Dalton:} At the end of 18th century there were two laws arising in chemistry, which were silently pointing towards the atomic nature of the matter. 

\begin{enumerate}
    \item The first one was the conservation of mass, which states that the total mass of the reactants must be equal to the total mass of the products.
    \item Second one was the law of definite proportions. This law states that if a compound is broken down into its constituet chemical elements, then the masses of the constituents will always have the same proportions by weight, regardless of the quantity or sourve of the original substance.
\end{enumerate}

Dalton proposed that each chemical element is composed of atoms of a single, unique type, and though they cannot be altered or destrozed by chemical means, they can combine to form more complex structures. This marked the first truly scientific theory of the atom, since Dalton reached his conclusions by experimentation and examination of the results in an empirical fashion.

\textbf{Brownian Motion:} In 1827, the British botanist Robert Brown observed that dust particles inside pollen grains floating in water constantly jiggled about for no apparent reason. In 1905, Albert Einstein theorized that this Brownian motion was baused by the water molecules continuously knocking the grains about, and develiped a hypothetical mathematical model to describe it. The model was validated experimentally in 1908 by French physicist Jean Perrin, thus providing additional validation for particle theory (and by extension atomic theory).

\textbf{Statistical Mechanics:} In 1738, Daniel Bernoulli postulated that the pressure of gases and heat were both caused by the underlying motion of molecules. Nonetheless for the theory of ideal gases it was also necessary to postulate the existence of atoms.

James Clerk Maxwell used statisitcal mechanics in physics, a notion which Boltzmann and Rudolf Clausius used to expand and work on thermodynamics, and especially the law of entropy (2nd law of thermodynamics). All of statistical mechanics and the laws of heat, gas, and entropy were necessarily postulated upon the existence of atoms.

\subsection{Subatomic Particles}
\textbf{Electron:}  After the discovery of electric force and the existence of two types of charges, in 1859 Julius Plücker observed the radiation emitted from the cathode caused phosphorescent light. The first building block of the atom to be identified was the electron. Later on Arthur Schuster expanded upon the experiments by placing metal plates parallel to the cathode rays and applying an electric potential between the plates. The field deflected the rays toward the positively charged plate, providing further evidence that the rays carried negative charge. By measuring the amount of deflection for a given level of current, in 1890 Schuster was able to estimate the charge-to-mas ratio of the ray components.

While studying naturally fluorescing minerals in 1896, the French physicist Henri Becquerel discovered that they emitted radiation without any exposure to an external energy source. These radioactive materials became the subject of much interest. J. J. Thomson performed experiments that indicated the particles were unique, rather than waves, atoms, or molecules as was believed earlier. The electron's charge was more carefully measured by the American Physicist Robert Millikan.

In the Standard Model of particle physics, electrons belong to the group of subatomic particles called leptons, which are believed to be fundamental or elementary particles with no substructure. Electrons have the lowest mass of any charged lepton and belong to the first-generation of fundamental particles. Leptons differ from the other basic constituent of matter, the quarks, by their lack of strong interaction. All members of the lepton group are fermions, because they all have half-off integer spin; the electron has spin $\frac12$.

\textbf{Nuclei:} In the Atom model of Thomson, the electrons, and an equivalent number of positively charged particles were unifomly distributed thoughout the atom. The resulting atom was electrically neutral. Later on the Geiger-Marsden Experiment with Rutherford showed the image to be wrong. By shooting alpha particles to a thin golden foil. The explanation of the scattering data wasa  central Coulomb field caused by a massive, positively charged nucleus. Thus the atomic model had another step towards modeling reality more precisely.

\textbf{Proton:} Since the mass of a proton can be approximately of a hydrogen atom (the electron contributes really samll to the total mass.) the concept of a hydrogen-like partcile as a constituent of other atoms was developed over time. Because there were some indications such as the periodic table, the ratio of mass between different atoms, etc... 


In 1927, Rutherford proved that the hydrogen nucleus is present in other nuclei, While bombarding nitrogen with $\alpha$-particles, he observed positively charged particles with an unusually long range, which must have been ejected from the atom as well. From this he concluded that the nitrogen atom had been destroyed in these reactions and a light constituent of the nucleus had been ejected. He had already discovered similar long-ranged particles when bombarding hydrogen. From this he concluded that these partocles were hydrogen nuclei which, therefore, had to be constituents of nitrogen as well:
$$
^{14}\text{N} + ^4\text{He} \rightarrow ^{17}\text O + \text p
$$ 

Protons are spin $\frac12$ fermions and are not elementary particles, which mean they have substructures, quarks that make up the proton are held together with strong interaction.

\textbf{Neutron:} The Neutron was also detected by bombarding nuclei with $\alpha$-particles. Chadwick in 1932 found an appropriate experimental approach. He used the irradiation of beryllium with $\alpha$-particles from a polonium source, and thereby established the neutron as a fundamental consituent of nuclei.

\subsection{Nuclides}

\textbf{The Atomic and Mass Number: } Knowing that the nuclei is consists of positively charged particles (protons) the atomic number is the number of protons the nuclei has. Denoted by $Z$, the charge of the nucleus therefore is $Q = Ze$, the elementary charge $e \approx 1.6 \time 10^{-19}$, for neutral atoms this is also the number of electrons orbiting around the nucleus. 

\begin{callout}
    The Chemical properties of an atom is determined by its atomic number.
\end{callout}
\begin{callout}
    The classical method of determining the charge of the nucleus is the measurement of the characteristic $X$-rays of the atom to be studied. Moseley's law states that the energy of the $K_\alpha$-line is proportional to $(Z-1)^2$. Nowdaysm the detection of these catacteristic $X$-rays are being used in material analysis.
\end{callout}

In addition to the atomic number we define $N$ at the number of neutrons, then the mass number is defined as:
\begin{equation}
    A = Z+N
\end{equation}
\begin{itemize}
    \item Nuclides with the same mass number $A$ are called isobars.
    \item Nuclides with the same atomic number $Z$ are called isotopes.
    \item Nuclides with the same neutron number $N$ are called isotones.
\end{itemize}

We define the binding energy as:

\begin{equation}
    B(Z,A) = \left[ZM(^1\text H) + (A -Z)M_n - M(A, Z)\right]\cdot c^2
\end{equation}
where 
\begin{align*}
    M(^1H) &= M_p + m_e \\
    M_p &= 938.272 \ \mev/c^2 = 1836.153 m_e\\
    M_n &= 939.565\mev/c^2 = 1838.684 m_e \\
    m_e &= 0.511\mev/c^2
\end{align*}

\subsection{Parametrisation of Binding Energies}
\textbf{Semi-Empirical Mass Fromula (SEMF):} Apart from the lightest elements, the binding energy per nucleon for most nuclei is about $8-9\mev$. The parametrisation of nuclear masses as a function of $A$ and $Z$, which is known as the \textit{Weizsäcker formula}, or the semi empirical mass formula, is used to approximate the mass and various other properties of an atomic nucleus form its number of protons and neutrons. As the name suggests, it is based partly on theory and partly on empirical measurements. The formula represents the \textbf{liquid-drop model} proposed by George Gamow.

\textbf{Liquid-Drop Model:} The liquid-drop model was first proposed by George Gamow and furthur developed by Nield Bohr and John Archibald Wheeler. It treats the nucleus as a drop of incompressible fluid of very high density, held together by the nuclear force, there is similarity to the structure of a spherical liquid drop. While a crude model, the liquid-drop model accounts for the spherical shape of most nuclei and makes a rough prediction of binding energy.

\img{liquiddropmodel}{0.7}{Illustration of the terms of the semi-empirical mass formula in the liquid-drop model of the atomic nucleus.}

\begin{enumerate}
    \item \textbf{Volume Energy}, when an assembly of nucleons of the same size is packed together into the smallest volume, each interior nucleon has a certain number of other nucleons in contact with it. So, this nuclear energy is proportional to the volume.
    \item \textbf{Surface Energy} corrects for the previous assumption made that every nucleon interacts with the same number of other nucleons. This term is negative and proportional to the surface area, and is therefore roughly equivalent to liquid surface tension.
    \item \textbf{Coulomb Energy} the potential energy from each pair of protons. As this is a repulsive force, the binding energy is reduced.
    \item \textbf{Asymmetry energy} (also called Pauli energy), which accounts for the Pauli exclusion principle. Unequal numbers of neutrons and protons imply filling higher energy levels for one type of particle, while leaving lower energy levels vacant for the other type.
    \item \textbf{Pairing energy}, which accounts for the tendency of proton pairs and neutron pairs to occur. An even number of particles is more stable than an odd number due to spin coupling.
\end{enumerate}

\img{bindingenergies}{0.5}{The binding energy per nucleon (in $\mev$) shown as a function of the neutron number $N$ and atomic number $Z$ as given by the semi-empirical mass formula. A dashed line is included to show nuclides that have been discovered by experiment.}

\textbf{The Formula:} Nuclear fussion and fission and other nuclear reactions, or measuring the mass of nucleus and comparing it with the mass of protons and neutrons it contains reveals that the mass of the nucleus as a whole is smaller than the sum of its consittuents, This is because the binding energies between the neuleons are yet to be considered. From this we would write:
\begin{equation}
    M = ZM_p + NM_n - \frac{E_B(N,Z)}{c^2}
\end{equation}

The semi-empirical mass formula states the binding energy as:
\begin{equation}
    E_B = a_V A - a_S A^{2/3} - a_C\frac{Z(Z-1)}{A^{1/3}}-a_A\frac{(N-Z)^2}{4A} +\delta(N,Z)
\end{equation}

\textbf{Volume Term:} The term $a_V A$ is known as the volume term. The volume of the nucleus is proportional to $A$, so this term is proportional to the volume, hence the name.

The basis for this term is the strong nuclear force. The strong force affects both protons and neutrons, and as expected, this term is independent of $Z$. Because the number of pairs that can be taken from $A$ particles is $A(A-1)/2$, one might expect a term proportional to $A^2$. However, the strong force has a very limited range, and a given nucleon may only interact strongly with its nearest neighbors and next nearest neighbors. Therefore, the number of pairs of particles that actually interact is roughly proportional to $A$, giving the volume term its form.

\textbf{Surface Term:} The term $a_S A^{2/3}$ is known as the surface term. This term, also based on the strong force, is a correction to the volume term.

The volume term suggests that each nucleon interacts with a constant number of nucleons, independent of A. While this is very nearly true for nucleons deep within the nucleus, those nucleons on the surface of the nucleus have fewer nearest neighbors, justifying this correction. This can also be thought of as a surface-tension term, and indeed a similar mechanism creates surface tension in liquids.

If the volume of the nucleus is proportional to $A$, then the radius should be proportional to $A^{1/3}$ and the surface area to $A^{2/3}$. This explains why the surface term is proportional to $A^{2/3}$. It can also be deduced that $a_S$ should have a similar order of magnitude to $a_V$.

\textbf{Coulomb Term:} The basis for this term is the electrostatic repulsion between the protons. The potential energy of sphere of uniform charge density is:
$$
E = \frac35 \frac{1}{\fpe}\frac{Q^2}{R}
$$

Using an empirical nuclear radius of $R\approx r_0 A^{1/3}$ and $Q=Ze$ since the repulsion will only exist for more than one proton, $Z^2$ becomes $Z(Z-1)$:

\begin{equation}
    E \approx \frac{3e^2Z(Z-1)}{20\pi\epsilon_0r_0 A^{1/3}} = a_C \frac{Z(Z-1)}{A^{1/3}}
\end{equation}

Thus the term $a_C\frac{Z(Z-1)}{A^{1/3}}$ is known as the Coulomb or Electrostatic term.
\newpage

\imgl{asymmetry}{0.5}{Asymmetry Term}
\textbf{Asymmetry Term:} The term $a_A\frac{(N-Z)^2}{A}$ is known as the asymmetry term (or Pauli term). The Theoretical justification for this term is more complex. The pauli exlusion principle states that no two identical fermions can occupy exactly the same quantum state in an atom. At a given energy leve, there are only finitely many quantum states available for particles. What this means in the nucleus is that as more particles are added, thses partices must occupy higher energy levels, increasing the total energy of the nucleus. Note that this effect is not based on any of the fundamental forces, only the Pauli exclusion principle. 

Protons and neutrons, being distinct types of particles, occupy different quantum states. One can think of two different "pools" of states – one for protons and one for neutrons. Now, for example, if there are significantly more neutrons than protons in a nucleus, some of the neutrons will be higher in energy than the available states in the proton pool. If we could move some particles from the neutron pool to the proton pool, in other words, change some neutrons into protons, we would significantly decrease the energy. The imbalance between the number of protons and neutrons causes the energy to be higher than it needs to be, for a given number of nucleons. This is the basis for the asymmetry term.


\textbf{Pairing Term:} The last term $\delta(A,Z)$ is known as the pairing term. This term captures the effect of spin coupling. It is given by:
\begin{equation}
    \delta(A,Z) =\left\{ 
    \begin{array}{ll}
        +\delta_0 & \text{for even } Z, N\\
        0 & \text{for odd } A\\
        -\delta_0 & \text{for odd } Z, N\\
    \end{array}\right.
\end{equation}

\textbf{Constants}
The constants of this section are:
\begin{align*}
    a_V = 15.67\mev/c^2\\
    a_S = 17.23\mev/c^2\\
    a_C = 0.714\mev/c^2\\
    a_A = 93.15\mev/c^2\\
    \delta_0 = -11.2\mev/c^2
\end{align*}

\subsection{Isospin and The Charge  Independence of the Nuclear Force}
Protons and Neutrons not only have nearly equal masses, they also have similar nuclear interactions. This is oarticularly visible in the study of mirror nuclei.

\textbf{Isospin:} Isospin was introduced as a concept in 1932, well before the 1960s development of the quark model. The man who introduced it, Werner Heisenberg, did so to explain symmetries of the then newly discovered neutron (symbol $n$):
\begin{itemize}
    \item The mass of the neutron and the proton (symbol p) are almost identical: they are nearly degenerate, and both are thus often called nucleons. Although the proton has a positive electric charge, and the neutron is neutral, they are almost identical in all other aspects.
    \item The mass of the neutron and the proton (symbol p) are almost identical: they are nearly degenerate, and both are thus often called nucleons. Although the proton has a positive electric charge, and the neutron is neutral, they are almost identical in all other aspects.
\end{itemize}

This behavior is not unlike the electron, where there are two possible states based on their spin. Other properties of the particle are conserved in this case. Heisenberg introduced the concept of another conserved quantity that would cause the proton to turn into a neutron and vice versa. In 1937, Eugene Wigner introduced the term "isospin" to indicate how the new quantity is similar to spin in behavior, but otherwise unrelated. Protons and neutrons were then grouped together as nucleons because they both have nearly the same mass and interact in nearly the same way, if the (much weaker) electromagnetic interaction is neglected. In particle physics, the near mass-degeneracy of the neutron and proton points to an approximate symmetry of the Hamiltonian describing the strong interactions. It was thus convenient to treat them as being different states of the same particle.

Although the neutron does in fact have a slightly higher mass due to isospin breaking (this is now understood to be due to the difference in the masses of the up and down quarks and the effects of the electromagnetic interaction), the appearance of an approximate symmetry is useful even if it does not exactly hold; the small symmetry breakings can be described by a perturbation theory, which gives rise to slight differences between the near-degenerate states.
\begin{callout}
    Isospin is similar to, but should not be confused with weak isospin. Briefly, weak isospin is the gauge symmetry of the weak interaction which connects quark and lepton doublets of left-handed particles in all generations; for example, up and down quarks, top and bottom quarks, electrons and electron neutrinos. By contrast (strong) isospin connects only up and down quarks, acts on both chiralities (left and right) and is a global (not a gauge) symmetry.
\end{callout}

The symmetry between protons and neutrons may therefore be described by this formalism. The proton and Neutron are treated as two states of the nucleon which form a doublet:
\begin{equation}
\text{Nucleon} \ \ I :1/2 \ \ \left\{
    \begin{array}{ll}
        \text{proton:} & I_3 = +1/2\\
        \text{neutron:} & I_3 = -1/2
    \end{array}\right.
\end{equation}

Formally, isospin is treated as a quantum mechanical angular momentum:
\begin{equation}
    I_3^{\text{nucleus}}= \sum  I_3^{\text{nucleons}} = \frac{Z-N}{2}
\end{equation}

\section{Nuclear Stability}
There are only a samll number of nuclei, which are stable, the other are unstable and would decay in various ways. As the number of protons increases there would be more electric repulsion between the protons that would lower the binding energy in total. The Iron an Nickel isotopes possess the maximum binding energy per nucleon, thus they are the most stable nuclides. For still heavier masses nuclei becomes unstable to fission and decay spontaneously into two or more lighter nuclei. Because of released energy in the process the mass of the original atom must be larger than the sum of the masses of the daughter atoms. For a two body decay thus, the inequality below must apply:
\begin{equation}
    M(A,Z) > M(A-A',Z-Z') + M(A', Z')
\end{equation}

\begin{itemize}
    \item An isotope is said to be stable if its lifetime is considerably larger than the age of the solar system.
    \item It is very oftern the case that one of the daughter nuclei is a $^4\text{He}$ nucleus. This decay mode is called $\alpha$-decay and the nucleus of Hydrogen is called $\alpha$-particle.
    \item If a heavy nucleus decays into two similarly massive daughter nuclei we speack of spontaneous fission. The probability of spontaneous fission exceeds that of $\alpha$-decay only for nuclei with $Z\leq 110$  and is a fairly unimportant process for the naturally occurring heavy elements.
\end{itemize}

\textbf{Radioactive Decay:} nuclear decay is the process in which an (unstable) nucleus losses energy by radiating particles, or light. The decaying nucleus is called the parent radionuclide, and the process produces at least one daughter nuclide, unless it decays light. There are six types for decay:
\img{ptableradioactive}{1}{Radioactivity is characteristic of elements with large atomic numbers. Elements with at least one stable isotope are shown in light blue. Green shows elements of which the most stable isotope has a half-life measured in millions of years. Yellow and orange are progressively less stable, with half-lives in thousands or hundreds of years, down toward one day. Red and purple show highly and extremely radioactive elements where the most stable isotopes exhibit half-lives measured on the order of one day and much less.}
\begin{enumerate}
    \item \textbf{Alpha Decay} occurs when the nucleus ejects an alpha particle (helium nucleus).

    \item \textbf{Beta Decay} occurs in two ways;
    \begin{itemize}
        \item beta-minus decay, when the nucleus emits an electron and an antineutrino in a process that changes a neutron to a proton.
        \item  beta-plus decay, when the nucleus emits a positron and a neutrino in a process that changes a proton to a neutron, also known as positron emission.
    \end{itemize}

    \item In \textbf{Gamma Decay} a radioactive nucleus first decays by the emission of an alpha or beta particle. The daughter nucleus that results is usually left in an excited state and it can decay to a lower energy state by emitting a gamma ray photon.

    \item In \textbf{Neutron Emission}, extremely neutron-rich nuclei, formed due to other types of decay or after many successive neutron captures, occasionally lose energy by way of neutron emission, resulting in a change from one isotope to another of the same element.
    
    \item In \textbf{Electron Capture}, the nucleus may capture an orbiting electron, causing a proton to convert into a neutron. A neutrino and a gamma ray are subsequently emitted.
    
    \item In \textbf{Cluster Decay} and nuclear fission, a nucleus heavier than an alpha particle is emitted.
\end{enumerate}

\imgr{radioactivecurves}{0.3}{Decay constant determines the rate of decay. Decay constant is denoted by $\lambda$, “lambda”. This constant probability may vary greatly between different types of nuclei, leading to the many different observed decay rates.}
\textbf{Decay Constant:} The radioactive decay law states that the probability per unit time that a nucleus will decay is constant, independent of time. This constant is called the decay constant and is denoted by $\lambda$.

THe rate of nuclear decay is also measured in terms of half-lives. The half-life is the amount of time that a given nuclei would lose half of its radioactivity. 

The decay constant is related to lifetime and half-life by:
\begin{equation}
    \tau = \frac1\lambda \ \ \ \text{and} \ \ \ t_{1/2} = \frac{\ln 2}{\lambda}
\end{equation}

The measurement of the decay constants of radioactive nuclei is based upon finding the activity (the number of decays per unit time)
\begin{equation}
    A = -\frac{dN}{dt} = \lambda N
\end{equation}

where $N$ is the number of radioactive nuclei in the sample. The unit of activity is defined to became:
\begin{equation}
    1 \text{Bq [Becquerel]} = 1 \text{decay/s}
\end{equation}

For short-lived nuclides, the fall-off over time of the activity
\begin{equation}
    A(t) = \lambda N(t) = \lambda N_0e^{-\lambda t}
\end{equation}
may be measured using fast elecctronic counters. This method of measuring is not suitable for lifetimes larger than about a year. For longer-lived nuclei both the number of nuclei in the sample and the activity must be measured in order to obtain the decay constant from eq[11]



\subsection{Beta Decay}
\imgr{betaminus}{0.3}{Feynman digram of the $\beta^-$ decay}
A Beta decay is a type of radioactive decay, on which one of the nucleons (proton or neutrons) transforms into the other by decaying a beta particle (positron or electron respectively) and a quasi-massless particle (neutrino or antineutrino respectively). The two types of beta decay is known as beta minus (for electron decay) and beta plus (for positron decay). 

\textbf{Description:} 
Beta decay is a consequence of the weak force, which is characterized by relatively lengthy decay times. Nucleons are composed of up quarks and down quarks, and the weak force allows a quark to change its flavour by emission of a W boson leading to creation of an electron/antineutrino or positron/neutrino pair. For example, a neutron, composed of two down quarks and an up quark, decays to a proton composed of a down quark and two up quarks.

Electron capture is sometimes included as a type of beta decay, because the basic nuclear process, mediated by the weak force, is the same. In electron capture, an inner atomic electron is captured by a proton in the nucleus, transforming it into a neutron, and an electron neutrino is released.

Beta decay conserves a quantum number known as the lepton number, or the number of electrons and their associated neutrinos (other leptons are the muon and tau particles). These particles have lepton number $+1$, while their antiparticles have lepton number $-1$. Since a proton or neutron has lepton number zero, $\beta^+$ decay (a positron, or antielectron) must be accompanied with an electron neutrino, while $\beta^-$ decay (an electron) must be accompanied by an electron antineutrino.

\imgl{betapositive}{0.3}{Feynman digram of the $\beta^+$ decay}
\textbf{Reaction and Phenomenology}
The $\beta^-$ decay reaction is written as:
\begin{equation}
    \ch{X}{A}{Z}{}{N} \rightarrow \ch{X'}{A}{Z+1}{}{N-1} + \ch{e}{}{}{-}{} + \bar\nu
\end{equation}
The underlying interation as shown in figure 6 is:
\begin{equation}
    \ch{n}{}{}{}{} \rightarrow \ch{p}{}{}{}{} + \ch{e}{}{}{-}{} + \bar\nu
\end{equation} 

The $beta^+$ decay reaction however is:
\begin{equation}
    \ch{X}{A}{Z}{}{N} \rightarrow \ch{X'}{A}{Z-1}{}{N+1} + \ch{e}{}{}{-}{} + \bar\nu
\end{equation}

with the underlying reaction as shown in figure 7:
\begin{equation}
    \ch{p}{}{}{}{} \rightarrow \ch{n}{}{}{}{} + \ch{e}{}{}{+}{} + \bar\nu
\end{equation} 

Let us consider nuclei with equal mass number $A$ we can write the SEMF as:

\begin{equation}
    M(A,Z) = \alpha\cdot A - \beta\cdot Z + \gamma \cdot Z^2 + \delta(N,Z)
\end{equation}
where
\begin{align*}
    \alpha &= M_N -a_V +a_S A^{-1/3} + \frac{a_A}{4}\\
    \beta  &= a_A + a_C + M_N - M_P- m_e\\
    \gamma &= \frac{a_C}{A^{1/3}} + \frac{a_A}{A} 
\end{align*}
The nucleus with the smallest mass in an isobaric spectrum is stable with respect to $\beta$-decay. Therefore by taking the derivative and setting equal to zero we find the most stable atomic number $Z$:
\begin{align*}
    \frac{dM}{dZ} &= 0 \\
    \rightarrow -\beta + 2\gamma Z &= 0
\end{align*}
Thus the most stable atomic number $Z$ for any given mass number is:
\begin{equation}
    \boxed{
        Z_{\text{most stable}} = \frac{\beta}{2\gamma} = \frac{a_A + a_C + M_N - M_P- m_e}{2\left(\frac{a_C}{A^{1/3}} + \frac{a_A}{A}\right)} = \frac{66.2787 A}{130.462 + A^{2/3} }
    }
\end{equation}
\textbf{Beta Decay in Odd Mass Nuclei}
Let's find the most stable atom with atomic mass $A=101$ using eq[18]. The result is $Z = 44$. Isobars with more neutrons, such as $\ch{Mo}{101}{42}{}{}$ and $\ch{Tc}{101}{43}{}{}$, would have beta minus decay.
\begin{align}
    \ch{Mo}{101}{42}{}{}\rightarrow \ch{Tc}{101}{43}{}{} + \ch{e}{}{}{-}{}+\bar{\nu_e}\\
    \ch{Tc}{101}{43}{}{}\rightarrow \ch{Ru}{101}{44}{}{} + \ch{e}{}{}{-}{} + \bar{\nu_e}
\end{align}

\begin{callout}
    Energetically $\beta^-$ decay is possible whenever the mass of the daughter atom $M(A,Z+1)$ is smaller than the mass of its isobaric neighbors.
    $$
    M(A,Z)> M(A, Z+1)
    $$
    We consider here the mass of the whole atom and not just that of  the nucleus alone and so the rest mass of the electron created in the decay is automatically taken into account
\end{callout}

The reverse is possible for $\ch{Rh}{101}{45}{}{}$ and $\ch{Pd}{101}{46}{}{}$
\begin{align}
    \ch{Pd}{101}{46}{}{}\rightarrow \ch{Rh}{101}{45}{}{} + \ch{e}{}{}{+}{}+\bar{\nu_e}\\
    \ch{Rh}{101}{45}{}{}\rightarrow \ch{Ru}{101}{44}{}{} + \ch{e}{}{}{-}{} + \bar{\nu_e}
\end{align}

\begin{callout}
    Since the mass of  a free neutron is largen than the proton mass, this process is only possible inside a nucleus. By contrast, neutrons outside nuclei can and do decay (They would do a $\beta^-$ decay since neutrons are not stable particles.).

    Energetically, beta plus decay is possible whenever the following relationship between masses is satisfied:
    $$
    M(A,Z) > M(A, Z-1) + 2m_e
    $$
    This relation takes into account the creation of a positron and the existence of an excess electron in the parent atom.
\end{callout}

\imgr{massparabolaA106}{0.3}{The Mass parabola of $A=106$ which shows two parabola because of even-even and odd-odd parity term.}
\textbf{Beta Decay in even mass Nuclei:} For even mass nucleis there exist two parabolas, one for the odd-odd nuclei and one for the even-even nuclei. Often there's more than one $\beta-$stable isobar, especially in the range $A>70$. For the case of $A=106$ in figure 8. There exists two parabolas, but also to $\beta$-stable atoms.

$\ch{Pd}{106}{46}{}{}$ and $\ch{Cd}{106}{48}{}{}$ isobars are on the lower parabolas, and $\ch{Pd}{106}{46}{}{}$ is the stablest. $\ch{Cd}{106}{48}{}{}$ is $\beta$-stable, since its two odd-odd neighbours both lie above it. The conversion of $\ch{Cd}{106}{48}{}{}$ is thus only opssible through a double $\beta$-decay into $\ch{Pd}{106}{46}{}{}$ 
\begin{equation}
    \ch{Cd}{106}{48}{}{} \rightarrow \ch{Pd}{106}{46}{}{} + 2\ch{e}{}{}{+}{} +2\nu_{\text{e}}
\end{equation}

The probability for such process is so small that $\ch{Cd}{106}{48}{}{}$ may be considered to be a stable nuclei. Odd-Odd nuclei always have at least one more strongly bound even-even neighbour. Thus they are unstable. The only exceptions are light nuclei such as $\ch{H}{2}{1}{}{}$, $\ch{Li}{6}{3}{}{}$, $\ch{B}{10}{5}{}{}$ and $\ch{N}{14}{7}{}{}$, which are stable to $\beta$-decay.

\begin{callout}
    Some odd-odd nuclei can undergo bot beta minus and beta plus decays. for example $\ch{K}{40}{19}{}{}$ and $\ch{Cu}{64}{29}{}{}$.
\end{callout}

\imgl{electroncapture}{0.5}{The leading-order Feynman diagrams for electron capture decay. An electron interacts with an up quark in the nucleus via a W boson to create a down quark and electron neutrino. Two diagrams comprise the leading (second) order, though as a virtual particle, the type (and charge) of the W-boson is indistinguishable.}
\subsection{Electron Capture}
Electron capture or K-electron capture is a process in which the nuclei would transform one proton into a neutron by capturing one of atoms electrons. The process is as below:

\begin{equation}
    \text p +\ch{e}{}{}{-}{} \rightarrow \text n + \nu_{\text e}
\end{equation}

This reaction occurs mainly in heavy nuclei where the nucelar radii are larger and the electron orbits are more compact. Usually the electrons that are captured are from the innermost (the "$K$") shell since such electrons are closest to the nucleus and their radial wave function has a maximum at the center of the nucleus.

Electron-capture reactions compete with $\beta^+$-decay. The following condition is a consequence of energy conversion.

\begin{equation}
    M(A,Z) > M(A,Z-1) + \epsilon
\end{equation}

The electron that is captured is one of the atom's own electrons, and not a new, incoming electron, as might be suggested by the way the above reactions are written. A few examples of electron capture are:
\begin{align}
    \ch{Al}{26}{13}{}{} +\electron \rightarrow \ch{Mg}{16}{12}{}{} +\nu_{\electron}\\
    \ch{Ni}{59}{28}{}{} +\electron \rightarrow \ch{Co}{59}{27}{}{} + \nu_{\electron}
\end{align}


\subsection{Alpha Decay}
\imgr{wrefandtrans}{0.4}{Illustration of the tunnelling probability of a wave packet.}
Protons and neutrons have bingind energy, even in heavy nuclei. This means that they cannot generally escape the nucleus by themselves. However it is possible that a bunch of them escape as one. The probability of such emission decreases if the group of escaping nucleons increase in number, Thus the most significant particle to be able of such process is the nucleus of Helium, $\ch{He}{4}{}{}{}$. Such decays are called Alpha decay and the particle (Helium nucleus) is called $\alpha$-particle.

Alpha particles were first described in the investigations of radioactivity by Ernest Rutherford in 1899, and by 1907 they were identified as He2+ ions. By 1928, George Gamow had solved the theory of alpha decay via tunneling. The alpha particle is trapped inside the nucleus by an attractive nuclear potential well and a repulsive electromagnetic potential barrier. Classically, it is forbidden to escape, but according to the (then) newly discovered principles of quantum mechanics, it has a tiny (but non-zero) probability of "tunneling" through the barrier and appearing on the other side to escape the nucleus. Gamow solved a model potential for the nucleus and derived, from first principles, a relationship between the half-life of the decay, and the energy of the emission, which had been previously discovered empirically and was known as the Geiger–Nuttall law.


Quantum mechanics, the theory which applies to the small scales of the universe wish that we consider the particles and quantum systems as wave-like quantity that can propagate like a wave. Therefore a quantum particle can pass through solid potentials where a classical particle would just bounce back. Although this propagation through potential to the other side of it is probabilistic, it is not zero, as in the case with classical particles.


The range pf lifetimes for the $\alpha$-decay of heavy nuclei is extremely large. These life times can be calculated in quantum mechanics by treating the $\alpha$-particle as a wave packet. The probability for the $\alpha$-particle to escape from the nucleus is given by the probability for its penetrating the Coulomb barrier (the tunnel effect). If wedivide the Coulomb barrier into thin potential walls and look at the probability of the $\alpha$-particle tunnelling through one of these, then the transmission $T$ is given by

\begin{equation}
    T \approx e^{-2\kappa \Delta r}, \ \ \ \text{where} \ \ \ \kappa = \sqrt{2m|E-V|}/\hbar
\end{equation}

and $\Delta r$ is the thickness of the barrier and $V$ is its height. $E$ is the energy of the $\alpha$-particle. A Coulomb barrier can be thought of as a barrier composed of alarge number of thin potential walls of different heights. The transmission can be
described accordingly by:

\begin{equation}
    T = e^{-2G}
\end{equation}

The Gamow Factor can be approximated by the integral:
\begin{equation}
    G = \frac{1}{\hbar}\int_R^{r_1} \sqrt{2m|E-V|} \approx \frac{2\pi\alpha(Z-2)}{\beta}
\end{equation}

Where $\beta = v/c$ is the velocity ofthe ourgoing $\alpha$-particle and $R$ is the nuclear radius. The probability per unit time $\lambda$ for an $\alpha$-particle to escape from the nucleus is therefore proportional to the probability of finding such $\alpha$-particle in the nucleus, the number of collisions($\propto v_0/2R$) of the $\alpha$-particle with the barrier and the transmission probability:

\begin{equation}
    \lambda = w(\alpha)\frac{v_0}{2R}e^{-2G}
\end{equation}

\begin{callout}
    Most $\alpha$-emitting nuclei are heavier than lean. For lighter nuclei with $A \leq 140$, $\alpha$-decay is energetically possible, but the energy released is extremely small. Therefore, their nuclear lifetimes are so long that decays are usually not observable.
\end{callout}

\subsection{Nuclear Fission}
\textbf{Spontaneous Fission:} The largest binding energy is found in those nuclei in the region of $\ch{Fe}{56}{}{}{}$. For heavier nuclei, it decreases, for example a nucleus with $Z>40$ can in principle split into two lighter nuclei. The potential barrier which must be tunnelled through is, however so large that such spontaneous fission reactions are generally speaking extremely unlikely. It is interesting to find the charge $Z$ above which the nuclei becomes fission unstable, the point from which the mutual Coulombic repulsion of the protons outweights the attraction of nuclear force. And estimate can be obtained by considering the surface and the coulomb energies during the fission deformation.

As the nucleus is deformed the surface energy increases, while the coulomb energy decreases. If the deformation leads to an energetically more favourable configuration, the nucleus is unstable. Quantitatively, we have consider a deformation from a sphere into an ellipsoid with axes:

\begin{align*}
    a &= R(1+\epsilon)\\
    b = R(1+\epsilon)^{-1/2} &\approx R(1-\epsilon/2)
\end{align*}

Thus the surface energy and the coulomb energy would be:
\begin{align}
    E_S = a_S A^{2/3}\left(1+\frac25\epsilon^2 + \dots\right)\\
    E_C = a_CZ^2A^{-1/3}\left(1-\frac15\epsilon^2+\dots\right)
\end{align}

Hence the deformation $\epsilon$ changes the total energy by:
\begin{equation}
    \Delta E = \frac{\epsilon^2}{5}\left(2a_SA^{2/3}-a_CZ^2A^{-1/3}\right)
\end{equation}

if $\Delta E<0$ the deformation is energetically favourable. Thus the fissoin barrier disappears for:

\begin{equation}
    \frac{Z^2}{A} \geq \frac{2a_S}{a_C}\approx 48:
\end{equation}

This is the case for nuclei with $Z>114$ and $A>270$.

\textbf{Induced Fission:} Since the fission barrier is only about $6\mev$. This energy can be supplied by if one uses a flow of low energy neutrons to induce neutron capture reactions. These puch  the nucleus into an excited state above the fission barrier and spits up. This process is known as induced nuclear fission.

\subsection{Gamma Decay}
Gamma rays are produced during gamma decay, which normally occurs after other forms of decay occur, such as alpha or beta decay. A radioactive nucleus can decay by the emission of an $\alpha$ or $\beta$ particle. The daughter nucleus that results is usually left in an excited state. It can then decay to a lower energy state by emitting a gamma ray photon, in a process called gamma decay.

The emission of a gamma ray from an excited nucleus typically requires only 10-12 seconds. Gamma decay may also follow nuclear reactions such as neutron capture, nuclear fission, or nuclear fusion. Gamma decay is also a mode of relaxation of many excited states of atomic nuclei following other types of radioactive decay, such as beta decay, so long as these states possess the necessary component of nuclear spin. When high-energy gamma rays, electrons, or protons bombard materials, the excited atoms emit characteristic "secondary" gamma rays, which are products of the creation of excited nuclear states in the bombarded atoms. Such transitions, a form of nuclear gamma fluorescence, form a topic in nuclear physics called gamma spectroscopy. Formation of fluorescent gamma rays are a rapid subtype of radioactive gamma decay.

In certain cases, the excited nuclear state that follows the emission of a beta particle or other type of excitation, may be more stable than average, and is termed a metastable excited state, if its decay takes (at least) 100 to 1000 times longer than the average 10-12 seconds. Such relatively long-lived excited nuclei are termed nuclear isomers, and their decays are termed isomeric transitions. Such nuclei have half-lifes that are more easily measurable, and rare nuclear isomers are able to stay in their excited state for minutes, hours, days, or occasionally far longer, before emitting a gamma ray. The process of isomeric transition is therefore similar to any gamma emission, but differs in that it involves the intermediate metastable excited state(s) of the nuclei. Metastable states are often characterized by high nuclear spin, requiring a change in spin of several units or more with gamma decay, instead of a single unit transition that occurs in only 10-12 seconds. The rate of gamma decay is also slowed when the energy of excitation of the nucleus is small.

An emitted gamma ray from any type of excited state may transfer its energy directly to any electrons, but most probably to one of the K shell electrons of the atom, causing it to be ejected from that atom, in a process generally termed the photoelectric effect (external gamma rays and ultraviolet rays may also cause this effect). The photoelectric effect should not be confused with the internal conversion process, in which a gamma ray photon is not produced as an intermediate particle (rather, a "virtual gamma ray" may be thought to mediate the process). 

\section{Scattering}
Scattering experiments are an importanttool for physics as a whole, generally we understand anything by throwing a beam of well-known things at it, and then see how they would scatter, or interact with each other. The object to be studied is called the target and is bombarded with a beam of particles with (mostly) well-defined energy. The interaction of the form

$$
a + b\rightarrow c+d
$$

occurs. where $a$ and $b$ denote the beam and the target and $c$ , $d$ are the products of the interaction. If the interaction does not change the particles themselves (they bounce each other off without changing each others properties) Then the scatterin is said to be elastic.

\textbf{Elastic Scattering:} In an elastic process:
$$
a + b \rightarrow a' + b'
$$
the same particles are presented both before and after the scattering. But keep in mind that the energy and momenta would change (other wise nothing even collided), that's why the apostrophe was added to the names.
\img{scatterings}{1}{Scattering processes: (a) elastic scattering; (b) inelastic scattering - production of an excited state which then decays into two particles; (c) inelastic production of new particles; (d) reaction of colliding beams}

It is easily seen that in order to resolve small target structures, larger beam energies are required. The reduced de Broglie wavelength $\bar\lambda = \lambda /2\pi$ of a particle with momentum $p$ is given by:

\begin{equation}
    \bar\lambda = \frac{\hbar}{p}=\frac{\hbar c}{\sqrt{2mc^2E_{\text{kin}}+E^2_{\text{kin}}}}\approx \left\{
        \begin{array}{ll}
            \hbar/\sqrt{2mE_{\text{kin}}} & \text{for } \ E_{\text{kin}} <\!< mc^2\\
            \hbar c/E_{\text{kin}} \approx \hbar c / E & \text{for } E_{\text{kin}} >\!> mc^2
        \end{array}
    \right.
\end{equation}
The largest wavelength that can resolve structures of linear extension $\Delta x$, is of the same order.

From the uncertainty principle the corresponding particle momentum is:
\begin{equation}
    p \gtrsim\frac{\hbar}{\Delta x} ,  \ \ \ pc \gtrsim \frac{\hbar c}{\Delta x} \approx \frac{200\mev \text{fm}}{\Delta x}
\end{equation}

\subsection{Cross-Sections}
The most important quantity for the description and interpretation of scattering experiments is the cross-section $\sigma$, which is a yardstick of the probability of a reaction between the two colliding particles.

\textbf{Geometric reaction cross-section:} Imagine a thin scattering target of thickness $d$ with $N_b$ scattering centres and with particle density $n_b$. Each target particle has a cross-sectional area $\sigma_b$, to be determined by experiment. 
A reaction occurs whenever a beam particle hits a target particle, and we assume that the beam particle is then removed from the beam. We do not distinguish between the final target states, i.e., whether the reaction is elastic or inelastic. The total reaction rate $N$, i.e., the total number of reactions per unit time, is given by the difference in the beam particle rate $N_a$ a upstream and downstream of the target. This is a direct measure for the cross-sectional area $\sigma_b$.

We further assume that the beam has cross-sectional area $A$ and particle density na. The number of projectiles hitting the target per unit area and per unit time is called the flux $\Phi_a$. This is just the product of the particle density and the particle velocity $v_a$:

\begin{equation}
    \Phi_a = \frac{\dot{N_a}}{A}=n_a\cdot v_a
\end{equation}
The total number of target particles within the beam area is $N_b = n_b\cdot A\cdot d$. Hence the reaction rate $\dot{N}$ is given by the product of the incoming flux and the total cross-sectional area seen by the particles:

\begin{equation} 
    \dot{N}=\Phi_a\cdot N_b \cdot \sigma_b
\end{equation}
This formula is valid as long as the scattering centres do not overlap and particles are only scattered off individual scattering centres. The area presented by a single scattering centre to the incoming projectile a, will be called the geometric reaction cross-section: in what follows:

\begin{equation}
    \sigma_b = \frac{\dot{N}}{\Phi_a\cdot N_b}
\end{equation}

\subsection{{The Golden Rule}}
\end{document}