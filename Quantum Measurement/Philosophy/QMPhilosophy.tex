\documentclass[10pt]{article}
\usepackage{amsmath}
\usepackage[utf8]{inputenc}
\usepackage{amsmath}
\usepackage{amsfonts}
\usepackage{amssymb}
\usepackage{url}
\usepackage{makeidx}
\usepackage{graphicx}
\usepackage{graphicx, adjustbox}
\usepackage{lmodern}
\usepackage{fourier}
\usepackage{float}
\usepackage{caption}
\usepackage{wrapfig}
\usepackage{mhchem}
\usepackage[left=2.5cm,right=2.5cm,top=1cm,bottom=2cm]{geometry}
\usepackage{multicol}
\usepackage{soul}
%Colors
\usepackage[dvipsnames]{xcolor}


\definecolor{black}{RGB}{0, 0, 0}
\definecolor{richblack}{RGB}{7, 14, 13}
\definecolor{charcoal}{RGB}{45, 67, 77}
\definecolor{delectricblue}{RGB}{93, 117, 131}
\definecolor{cultured}{RGB}{245, 245, 245}
\definecolor{lightgray}{RGB}{211, 216, 218}
\definecolor{silversand}{RGB}{190, 194, 198}
\definecolor{spanishgray}{RGB}{148, 150, 157}
\definecolor{darkliver}{RGB}{64, 63, 76}

\colorlet{lightdelectricblue}{delectricblue!30}
\colorlet{lightdarkliver}{darkliver!30}


%ColorDefines
\newcommand{\trueblack}[1]{\textcolor{black}{#1}}
\newcommand{\rich}[1]{\textcolor{richblack}{#1}}
\newcommand{\lightblack}[1]{\textcolor{charcoal}{#1}}
\newcommand{\lightrich}[1]{\textcolor{delectricblue}{#1}}
\newcommand{\liver}[1]{\textcolor{darkliver}{#1}}

%Boxes
\usepackage{tcolorbox}
\newtcolorbox{calloutbox}{center,%
    colframe =red!0,%
    colback=cultured,
    title={Callout},
    coltitle=richblack,
    attach title to upper={\ ---\ },
    sharpish corners,
    enlarge by=0.5pt}

\newtcolorbox[use counter=equation]{eq}{center,
	colframe =red!0,
	colback=cultured,
	title={\thetcbcounter},
	coltitle=richblack,
	detach title,
	after upper={\par\hfill\tcbtitle},
	sharpish corners,
    enlarge by=0.5pt }
    
\newtcolorbox{qt}{center,
	colframe=delectricblue,
	colback=white!0,
	title={\large "},
	coltitle=delectricblue,
	attach title to upper,
	after upper ={\large "},
	sharp corners,
	enlarge by=0.5pt,
	boxrule=0pt,
	leftrule=2pt}
	
\newtcolorbox{qu}{center,%
    colframe =red!0,%
    colback=spanishgray!25,
    title={My Question},
    coltitle=richblack,
    attach title to upper={\ ---\ },
    sharpish corners,
    enlarge by=0.5pt}
    
\newcounter{theo}
\newtcolorbox[use counter=theo]{theobox}
	{center,%
    colframe =red!0,%
    colback=cultured,
    title={Theorem \thetcbcounter},
    coltitle=richblack,
    attach title to upper={\ ---\ },
    sharpish corners,
    enlarge by=0.5pt}

\newcounter{examplecounter}
\newtcolorbox[use counter=examplecounter]{example}
	{center,%
    colframe =red!0,%
    colback=cultured,
    title={Example},
    coltitle=richblack,
    attach title to upper={\ ---\ },
    sharpish corners,
    enlarge by=0.5pt}

    

        
    
% Highlighters
\newcommand{\hldl}[1]{%
	\sethlcolor{lightdarkliver}%
	\hl{#1}
}
\newcommand{\hldb}[1]{%
    \sethlcolor{lightdelectricblue}%
    \hl{#1}%
}


% Images
\newcounter{figurecounter}
\setcounter{figurecounter}{1}

\newcommand{\img}[3]{
    \begin{figure}[h!]
        \centering
        \captionsetup{justification=centering,margin=0cm,labelformat=empty}
        \includegraphics[width=#2\linewidth]{./img/#1}
        \label{figure}
        \caption{\small\textbf{fig: \thefigurecounter} -- \textcolor{darkliver}{#3}}
    \end{figure}
    \addtocounter{figurecounter}{1}}

\newcommand{\imgr}[3]{
    \begin{wrapfigure}{r}{#2\textwidth}
        \centering
        \captionsetup{justification=centering,margin=0cm,labelformat=empty}
        \includegraphics[width=\linewidth]{./img/#1}
        \label{figure}
        \caption{\small \textbf{fig: \thefigurecounter} -- \textcolor{darkliver}{#3}}
    \end{wrapfigure}
    \addtocounter{figurecounter}{1}}

\newcommand{\imgl}[3]{
    \begin{wrapfigure}{l}{#2\textwidth}
        \centering
        \captionsetup{justification=centering,margin=0cm,labelformat=empty}
        \includegraphics[width=\linewidth]{./img/#1}
        \label{figure}
        \caption{\small \textbf{fig: \thefigurecounter} -- \textcolor{darkliver}{#3}}
    \end{wrapfigure}
    \addtocounter{figurecounter}{1}}

% New commands
\newenvironment{callout}
	{\begin{calloutbox}\color{charcoal}\textbf\textit}
	{\end{calloutbox}}

%%% Commands for this file only
\newcommand{\hamil}{\mathcal{H}}
\newcommand{\braket}[2]{\left\langle#1\vert#2 \right\rangle}
\newcommand{\ket}[1]{\left\vert#1\right\rangle}
\newcommand{\bra}[1]{\left\langle#1\right\vert}
\newcommand{\specproj}[2]{\textbf{E}^{#1}(#2)}
\newcommand{\trace}[1]{\text{Tr}\left[#1\right]}
\newcommand{\sys}{\mathcal{S}}
\newcommand{\m}{\mathcal{A}}
\newcommand{\state}{\textbf{\textit{S}}}
\newcommand{\observable}{\textbf{\textit{O}}}
\newcommand{\newpoint}[1]{\indent $\blacktriangleright$ \textbf{#1}$\ \ \Rightarrow$}
\newcommand{\divider}{\par\noindent\rule{\textwidth}{0.4pt}\\ \ \\}

\title{Quantum Measurement \\ \large Philosophy}
\author{Amir H. Ebrahimnezhad}

\begin{document}
    \maketitle
    \tableofcontents
    \section{Computability of the World}
        \subsection{Is absolute Knowledge of the World Possible?}
            \newpoint{The Dream of Parmenides}
            It seems too good to be true: the idea of inalterable, timeless knowledge; a knowledge that does not depend upon changing observations and experiences, because it has its sole origin in the logic of rational thought; a knowledge that, nonetheless, allows all experience to be ordered within a grand, universal context; a knowledge that is given a priori and is conditioned by nothing but itself. In short: \textbf{Knowledge that can claim for itself absolute certainty and validity.}
            \\
            \\
            Parmenides of Elea and the school of philosophers he founded were probably the first to entertain the vision, more than 2000 years ago, that it might be possible to attain an absolute knowledge of the world. With their effort to recognize the nature of “true being”, they initiated an understanding of the world according to which the phenomenal reality is merely the deceptive illusiveness of a true and unchangeable world. This hidden world of true being, Parmenides believed, is only accessible by pure reasoning.
            \\
            \\
            \textbf{Thus, the idea coagulated that true knowledge of the world could only be arrived at by following the path of rational thought.}
            \\
            \\
            It is interesting to note that this weak point in the ontology of Parmenides was already pointed out by Aristotle, who critised Parmenides' simple linguistic usage of the word “being”. \textbf{Instead, Aristotle introduced a distinction between that which is (“actuality”) and that which might be (“potentiality”). This he ultimately raised to a point of departure for his own metaphysics, in which he differs clearly from that of the Eleates.}
            \\
            \\
            \divider
            \newpoint{In Search for the Archimediean Point of Knowledge} Despite the terminological difficulties with respect to the concept of “true being”, which already became apparent in ancient philosophy, \textbf{the vision that one can arrive at absolute knowledge has persisted right down to modern times.} It was sustained, just as Parmenides intended it, by the conviction that such knowledge must be based solely upon rational thought, which is free from the deceptions practised on us by sensation and experience.
            \\
            \\
            The paradigm for the rational reconstruction of reality has been the so-called deductive method, which had been applied in a rigorous way by the mathematician Euclid for the axiomatic foundations of geometry. Ideally—\textbf{this was the leading thought of modern rationalism—true knowledge of reality should, as in geometry, be deducible in its entirety from a highest, and in itself irrefutable, principle.} This principle was assumed not only to be the ultimate reference point of all knowledge, \textbf{but also to be capable of providing a justification for the claim that the knowledge deduced from it is coherent and true.}
            \\
            \\
            Kant is well known to have pushed his way forward to the foundations of knowledge in his epochal work “Critique of Pure Reason” of 1781. The analysis of the conditions under which knowledge becomes possible led him to the concept of the \textbf{“transcendental subject” as the source of knowledge, prior to all experience. In his view, the subject's perception of the external world is affected by, as he called it, “things in themselves”. Following Kant, “things in themselves” constitute reality intrinsically, that is, independently of how we may experience reality.} They make up the cause of the phenomena and their determinedness, but they are themselves not recognizable.
            \begin{itemize}
                \item This conception was rejected first by Fichte, in 1794, in his “Doctrine of Science”. He argued that the knowledge-engendering function of “things in themselves” leaves knowledge still dependent upon the external world, and that knowledge therefore lacks the property of being unconditional. However, in Fichte's view, unconditionality is an indispensable prerequisite if the knowledge acquired by the transcendental subject is to be absolute and no longer dependent upon changes in external experience.
                \item \textbf{Fichte therefore set out from the idea that the actions of the transcendental subject must be completely unconditional, that is, caused only by the subject itself.}
            \end{itemize}
            Fichte's further arguments along this line can be summarized as follows: As the “$I$” opposes a “non-$I$” to itself, it also posits its object of cognition. Moreover, the differentiated world of objects finally arises from the repeated self-limitation “negations”) of the “$I$” as the “$I$” again and again poses itself, within the divisible “$I$”, a divisible “non–$I$”. In this way an increasingly fine network of borderlines arises between the “$I$” and the “non–$I$” and of demarcation of the “non–$I$” by the “$I$”. According to Fichte, the world of objects thus appears as a manifold of demarcations, which arise through the iterative self-demarcation of the autonomous “$I$”.
            \begin{callout}
                This is a really beautiful way, to distinguish things from one another using a very simple pure knowledge “$A=A$”. 
            \end{callout}
            The radical subjectivism that we encounter here was already in Fichte's time a target of criticism. For example Schelling, initially a loyal follower of Fichte, remarked—not without a certain element of mockery—that the divine works of Plato, Sophocles and other great minds were actually his own, as they—if one construes the subjective idealism  consistently—are engendered by him through productive intuition. \textbf{As a consequence of this, he argued, the identity principle itself would remain “after extraction of all substance from the speculation” as no more than “empty chaff”.}
            \begin{callout}
                Fichte's philosophical approach, promoting the perceiving subject to the sole and unconditioned source of knowledge, led inevitably to a contradiction with empirical reality. In respect of its understanding of reality, the subjective idealism of Fichte clearly reveals the same weaknesses, the same loss of reality as did Parmenides' doctrine of true being.
            \end{callout}
            In his “ideas for a Philosophy of Nature”, published in 1797, Schelling attempted to correct this deficiency by first objectifying the subject-object identity and not as, Fichte had done, regarding it as an identity proceeding ecxlusively from the subject. Moreover, according to Schelling the subject-object identity must be considered as absolute. This means that subject and object are not two separate entities that stand in an identity relationship one to another, but rather that the entire Subjective is at the same time the entire Objective, and the entire Objective is at the same time the entire Subjective.
            \\
            \\
            Unlike Fichte, Schelling regarded the real world as more than just an epiphenomenon of the ideal world. Rather, he saw conceptual and material appearances as two manifestations of one and the same entity, and understood this as an absolute subject-object identity. At the same time he realised that he had to pass beyond the concept of Fichte's “Doctrine of Science” and to regard the “$I$” as an all-embracing world concept, one that encompassed botht the entire Subjective and the entire Objective. 
            \begin{qt}
                “Nature is the visible mind, the mind is invisible Nature”. This is to be taken as meaning that the perceiving subject can regard itself in Nature as in a mirror. Nature is the visible mind. Conversely, mind is invisible Nature, insofar as mind mirrors Nature at the highest level of its being. Thus, mind in Nature and Nature in mind can contemplate one another.
            \end{qt}
            \divider
            \newpoint{Utopian Fallacies}
            In Schelling's system, the task of empirical science is—at best—to verify the principles dictated to it by natural philosophy. On no account could they be disproved: the refutation of these principles would immediately have refuted the principles of reason and, thus, pursued the possibilities of cognition ad absurdum. \textbf{In fact, the principles of natural philosophy were seen as unchallengeably certain. If empirical results did not accord with them, then the principles remained unchallenged, whereas the empirical observations were taken to be obviously at fault, or incomplete, or deceptive.}
            \\
            \\
            A further aspect of Schelling's epistemology should be emphasized. In accordance with the identity principle, the ideal and the real together make up a whole that cannot be transcended. The whole is at the same time an allegory for the absolute, which however only reveals itself in the dichotomous form—that is, in ideal and real essence— to the subject. However, the absolute, whe it “expands” into the ideal and the real, must not lead out of the absolute. As the absolute, it must always remain identical with itself in its entire absoluteness.
            \begin{qt}
                Natural philosophy and empirical research into Nature are thus concerned with two fundamentally different objects of knowledge. One is concerned with “Nature as a subject” and the other with “Nature as an object”. “Nature as a subject” is a metaphor for the infinite productivity of Nature (“natura naturans”). It is downright natural dynamics. Its driving forces are the creatively acting natural principles, the discovery of which is the task of natural philosophy. “Nature as an object”, in contrast, is the productivity of Nature as made manifest in her products (“natura naturata”). These products are in themselves finite and appear as a terminated network of actions, the elucidation of which is the task of empirical research into Nature. However, to avoid the conceptional separation of Nature into two forms, Schelling employed an artifice. According to this, the productivity of Nature is not really extinguished in its products; rather, it still remains active with a force of production that, however, is infinitely delayed. As already encountered in the philosophy of the Eleates, the concept of the infinite again must be invoked in order to save the consistency of the epistemological model.
            \end{qt}
            In summary, we can say that Schelling's philosophy of Nature ran counter to today's scientific method in two important respects: \begin{itemize}
                \item Theory occupies a more important place than empiricism. Claims to truth need not stand the test of experience; they are exclusively derived from logical reasoning. In short: Knowledge a priori is given precedence over knowledge a posteriori.
                \item The research strategy propagated by Descartes, Newton and others, according to which one should proceed from the simple to the complex, from the part to the whole, from the cause to the effect, is turned by Schelling into its opposite. The analytical method, based upon dismantling, abstraction and simplification, is discarded—or at least diminished in importance—in favour of a holistic method.
            \end{itemize}
            \indent Schelling's philosophz emerged from the sober logic of the rationalistic perception of reality. One may therefore ask how this philosophy ever acquired the attribute "romantic". To answer this is no easy matter, as--to start with-- the term "romantic" does not have a clear meaning. It is one of those elastic words whose meaning is only clear within the particular context in which it is used. \textbf{At the beginning of the 19th century, Schelling's philosophy offered an alternative to the then predominant mechanistic view of Nature. It presented the mechanistic view of reality as a constricted perspective of a world which in actual fact is a complex whole. Moreover, Schelling's conception of Nature as an all-embracing organism appeared to correspond perfectly to the romantic ideal of an organic, indivisible unity of Man and Nature. In this way his philosophy took on a constitutive rôle for the romantic understanding of the world.} Nevertheless, the organismic conception of Nature propagated by Schelling led directly into the fog of a romantic transfiguration of Nature, in which, even today, adherents of a romantic understanding of Nature appear to be straying about.
            \\
            \\
            According to the organismic view, the phenomena of Nature inherently make up a unified whole and must be recognised from the perspective of this unity. For this, even in our times, again and again the idea of a holistic method for the understanding of the organism is propagated—a method believed to be in contrast to causal-analytical thinking. \textbf{However, the idea of an irreducible whole is anything but transparent. It cannot even be explicated meaningfully, let alone be determined by analytical thought. In the end, all that remains is the tautological conception of "the whole" as some kind of "whole".}
            \\
            \\
            The addendum to the Introduction of Schelling's "Ideas of a Philosophy of Nature", in which he repeatedly attempts to express the inexpressible, is rich in morsels of poetic word-creation and pictorial comparison that exhaust themselves in nebulous abstraction. We read, for example, that the absolute is “enclosed and wrapped up into itself", or that the absolute “is born out of the night of its being into the day”. There Schelling speaks of the “æther of absolute ideality” and the "mystery of Nature".
            \\
            \\
            The poetic language that Schelling makes extensive use of is clearly the inevitable accompaniment of a philosophy in which human thinking perpetually seeks to transcend itself. Only thinking about the absolute can be reflected in it, but not the absolute itself. Consequently there arises an unsolvable intellectual problem, an aporia, regarding the absolute, which Schelling tried to circumvent by introducing the concept of intellectual intuition (“intellektuale Anschauung”). This means the contemplative act of self-ascertainment of the absolute by the introspective self-consideration of the absolute. Thus, intellectual intuition appears like an inwardly inverted Archimedean point, from out of which the absolute was supposed to be made comprehensible.
            \\
            \\
            Schelling generalised the concept of intellectual intuition and abstracted it from the beholder and thus, as he put it, only considers the purely objective part of this act. Despite its romantic exaggeration by the organismic view of Nature, Schelling's philosophy is nonetheless, even more mechanistic than the mechanistic sciences that he criticised so violently. So that even Goethe raised against the proponents of romantic philosophy of Nature. Though he was initially sympathetic to the aims of these philosophers and have them active encourangement, later, however, he turned away from their, "dark", "ambiguous" and "hollow" talk, which he felt was "in the manner of prophers".
            \\
            \\
            Even the closest philosophers of the Jena Romantics' Circle, such as Friedrich Schlegen and Johann Wilhelm Ritter, criticised the notion that pure speculation, unaided by any experience, could provide the basis for any profound knowledge about the world. We shall according to Ritter, : approach imperceptibly the true theory, without searching for it-- we shall find it by observing what really happens, for what more do we desire of the atheory than that it tells us what is really happening?"
        \subsection{Are There Unsolvable World Enigmas?}
            \indent In conformity with the scientific Zeitgeist of the late 19th century, Du Bois-Reymond was a committed adherent to the mechanistic world-view, according to which the ultimate goal of knowledge of Nature was to demonstrate that all events in the material world could be traced back to movements of atoms governed by natural laws.
            \begin{callout}
                Every natural process—that was the credo of the mechanistic maxim—rests ultimately upon the mechanics of atoms.
            \end{callout}
            \begin{qt}
                the real history of mankind is the history of the natural sciences
            \end{qt}
            \indent It must therefore have appeared all the more contradictory when at the same time Du Bois-Reymond claimed, with almost missionary zeal, that certain world problems could never be solved by the natural sciences. Here, he was referring especially to humanistic issues. These, he maintained, could never be explained within the framework of a mechanistic view of Nature, because feelings, emotions and thoughts were fundamentally incapable of being naturalised and therefore could not be explained in terms of the mechanical laws of non-living matter.
            \\
            \\
            \indent In a famous speech, given to the Convocation of German Naturalists and Physicians, Du Bois-Reymond listed “seven world enigmas”, which he considered to be unsolvable:
            \begin{enumerate}
                \item The nature of force and matter
                \item The origin of movement 
                \item The origin of life
                \item The apparently purposeful, planned and goal-oriented organisation of Nature 
                \item The origin of simple sensory perception 
                \item Rational thought and the origin of the associated language
                \item The question of free will
            \end{enumerate}
            \indent In that sense, he ended his speech on the limitations of possible knowledge of Nature with the words “Ignoramus et ignorabimus” (we do not know, we shall not know).
            \\
            \\
            \indent Soon, his most powerful adversary emerged in the no less famous scientist Ernst Haeckel. The latter was professor of zoology in Jena and had already emerged as a vehement supporter and promulgator of the Darwinian theory of evolution. The central element in Darwin's thinking, according to which the evolution of life rested on the mechanism of natural selection, appeared to confirm to the fullest extent the materialistic world picture favoured by Haeckel.
            \\
            \\
            \indent The “Ignorabimus” thesis made up the strongest countercurrent to the optimistic attitude to knowledge taken by the natural scientists of the 19th century, which is why it provoked such violent criticism from Haeckel. The modesty of “Ignorabimus”, according to Haeckel, is a false modesty; in reality, he asserted, it is an expression of presumptuousness, as it claims to lay down limits to knowledge of the natural world that apply for all time, and to raise ignorance to the status of an absolute truth.




\end{document}