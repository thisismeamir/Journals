\documentclass[9pt,a4paper, twocolumn]{article}


\usepackage[utf8]{inputenc}
\usepackage[T1]{fontenc}
\usepackage{amsmath}
\usepackage{amsfonts}
\usepackage{amssymb}
\usepackage{url}
\usepackage{makeidx}
\usepackage{graphicx}
\usepackage{graphicx, adjustbox}
\usepackage{lmodern}
\usepackage{fourier}
\usepackage{float}
\usepackage{caption}
\usepackage{wrapfig}
\usepackage{mhchem}
\usepackage{multicol}
\usepackage{soul}


\usepackage[top = 1cm, bottom = 1cm, left = 0.7cm, right = 0.7cm]{geometry}



%Colors
\usepackage[dvipsnames]{xcolor}


\definecolor{black}{RGB}{0, 0, 0}
\definecolor{richblack}{RGB}{7, 14, 13}
\definecolor{charcoal}{RGB}{45, 67, 77}
\definecolor{delectricblue}{RGB}{93, 117, 131}
\definecolor{cultured}{RGB}{245, 245, 245}
\definecolor{lightgray}{RGB}{211, 216, 218}
\definecolor{silversand}{RGB}{190, 194, 198}
\definecolor{spanishgray}{RGB}{148, 150, 157}
\definecolor{darkliver}{RGB}{64, 63, 76}

\colorlet{lightdelectricblue}{delectricblue!30}
\colorlet{lightdarkliver}{darkliver!30}


%ColorDefines
\newcommand{\trueblack}[1]{\textcolor{black}{#1}}
\newcommand{\rich}[1]{\textcolor{richblack}{#1}}
\newcommand{\lightblack}[1]{\textcolor{charcoal}{#1}}
\newcommand{\lightrich}[1]{\textcolor{delectricblue}{#1}}


%Boxes
\usepackage{tcolorbox}
\newtcolorbox{calloutbox}{center,%
    colframe =red!0,%
    colback=cultured,
    title={Callout},
    coltitle=richblack,
    attach title to upper={\ ---\ },
    sharpish corners,
    enlarge by=0.5pt}

\newtcolorbox[use counter=equation]{eq}{center,
	colframe =red!0,
	colback=cultured,
	title={\thetcbcounter},
	coltitle=richblack,
	detach title,
	after upper={\par\hfill\tcbtitle},
	sharpish corners,
    enlarge by=0.5pt }
    
\newtcolorbox{qt}{center,
	colframe=delectricblue,
	colback=white!0,
	title={\large "},
	coltitle=delectricblue,
	attach title to upper,
	after upper ={\large "},
	sharp corners,
	enlarge by=0.5pt,
	boxrule=0pt,
	leftrule=2pt}
	
\newtcolorbox{exc}{center,%
    colframe =red!0,%
    colback=darkliver!15,
    title={Excercise},
    coltitle=richblack,
    attach title to upper={\ ---\ },
    sharpish corners,
    enlarge by=0.5pt}
    
\newcounter{theo}
\newtcolorbox[use counter=theo]{theorem}
	{center,%
    colframe =red!0,%
    colback=cultured,
    title={Theorem \thetcbcounter},
    coltitle=richblack,
    attach title to upper={\ ---\ },
    sharpish corners,
    enlarge by=0.5pt}

\newcounter{defcounting}
\newtcolorbox[use counter=defcounting]{define}
{center,%
	colframe=darkliver!50,%
	colback=white!0,
	title={\textcolor{black}{\textbf{\textbf{Definition}} \  \thetcbcounter  \ --}},
	coltitle=darkliver!50,
	attach title to upper,
	after upper ={ },
	sharp corners,
	enlarge by=0.5pt,
	boxrule=0pt,
	leftrule=2pt,
    rightrule = 0pt}

\newcounter{lemmacount}
\newtcolorbox[use counter=lemmacount]{lemma}
{center,%
    colframe=charcoal!50,%
    colback=white!0,
    title={\textcolor{black}{\textbf{\textit{Lemma}} \  \thetcbcounter  \ --}},
    coltitle=darkliver!50,
    attach title to upper,
    after upper ={ },
    sharp corners,
    enlarge by=0.5pt,
    boxrule=2pt}
    

    \newcounter{propcount}
    \newtcolorbox[use counter=propcount]{proposition}
    {center,%
        colframe=charcoal!50,%
        colback=white!0,
        title={\textcolor{black}{\textbf{\textit{Proposition}} \  \thetcbcounter  \ --}},
        coltitle=darkliver!50,
        attach title to upper,
        after upper ={ },
        sharp corners,
        enlarge by=0.5pt,
        boxrule=2pt}
        
    \newcounter{colocount}
    \newtcolorbox[use counter=colocount]{corollary}
    {center,%
        colframe=charcoal!50,%
        colback=white!0,
        title={\textcolor{black}{\textbf{\textit{Corollary}} \  \thetcbcounter  \ --}},
        coltitle=darkliver!50,
        attach title to upper,
        after upper ={ },
        sharp corners,
        enlarge by=0.5pt,
        boxrule=2pt}
        
\newcounter{examplecounter}
\newtcolorbox[use counter=examplecounter]{example}
	{center,%
    colframe =red!0,%
    colback=cultured,
    title={Example},
    coltitle=richblack,
    attach title to upper={\ ---\ },
    sharpish corners,
    enlarge by=0.5pt}

    

        
    
% Highlighters
\newcommand{\hldl}[1]{%
	\sethlcolor{lightdarkliver}%
	\hl{#1}
}
\newcommand{\hldb}[1]{%
    \sethlcolor{lightdelectricblue}%
    \hl{#1}%
}


% Images
\newcounter{figurecounter}
\setcounter{figurecounter}{1}

\newcommand{\img}[3]{
    \begin{figure}[h!]
        \centering
        \captionsetup{justification=centering,margin=0cm,labelformat=empty}
        \includegraphics[width=#2\linewidth]{./img/#1}
        \label{figure}
        \caption{\small\textbf{fig-\thefigurecounter} -- \textcolor{darkliver}{#3}}
    \end{figure}
    \addtocounter{figurecounter}{1}}

\newcommand{\imgr}[3]{
    \begin{wrapfigure}{r}{#2\textwidth}
        \centering
        \captionsetup{justification=centering,margin=0cm,labelformat=empty}
        \includegraphics[width=\linewidth]{./img/#1}
        \label{figure}
        \caption{\small \textbf{fig: \thefigurecounter} -- \textcolor{darkliver}{#3}}
    \end{wrapfigure}
    \addtocounter{figurecounter}{1}}

\newcommand{\imgl}[3]{
    \begin{wrapfigure}{l}{#2\textwidth}
        \centering
        \captionsetup{justification=centering,margin=0cm,labelformat=empty}
        \includegraphics[width=\linewidth]{./img/#1}
        \label{figure}
        \caption{\small \textbf{fig: \thefigurecounter} -- \textcolor{darkliver}{#3}}
    \end{wrapfigure}
    \addtocounter{figurecounter}{1}}

% New commands
\newenvironment{callout}
	{\begin{calloutbox}\color{charcoal}\textbf\textit}
	{\end{calloutbox}}

% for this file
\newcommand{\newpoint}[1]{\ \\ \indent$\mathsection$ \textbf{#1}}
\newcommand{\curveL}{\mathcal{L}}
\newcommand{\curveA}{\mathcal{A}}
\newcommand{\curveP}{\mathcal{P}}
\newcommand{\thm}{\text{Thm}}
\newcommand{\proof}{\ \\ \ \\ $\blacktriangleright$ \textit{proof: }}
\newcommand{\distinct}{ \\ \hrule}


\title{Lecture on Solid State}
\date{\today}
\author{Amir H. Ebrahimnezhad \\ \small \textit{University of Tehran Department of Physics.}}

\parskip=12pt % adds vertical space between paragraphs


\begin{document}
     \maketitle
     \section{First Lecture}
     In this lecture we are going to discuss the early state of solid state physics. At first let us recall the description of Heat capacity in a volume:
     \begin{equation}
        C_v = \frac{\partial u}{\partial T}|_V
     \end{equation}
     In reality the heat constant is made of two parts:
     \begin{equation}
        C_v = C_{\text{elec}} + C_{\text{lattice}}
     \end{equation}
     The lattice is the network of nuckeuses, in this section we are going to look into the lattice part. The energy of the lattice is described as below:
     \begin{equation}
        u_{\text{lattice}}=\sum_k\sum_p u_{kp}
     \end{equation}
     where $k$ is the network properties and $p$ is the polarization. Assuming that there exists a wave in the lattice there is a graph for possible $\omega$ in a $k = \frac{2\pi}{\lambda}$ since the wave can be longitudinal and transversive we can have polarizations. 
     \\
     \\
     In our simplest models we can write the equation for energy as a collection of simple harmonic oscillator:
     \begin{equation}
        u_{\text{lattice}} = \sum_{k,p} <n_{k,p}\hbar \omega_{k,p}
     \end{equation}    
     $n_{k,p}$ is the average heat number of occupation, meaning that in temperature $T$ how many heat equilibrium do we have. and $\hbar \omega_{k,p}$ is the energy of oscillation with momentum $k$ and polarization $p$.  calculation of occupation number is to be left for lateer but the answer is of the form:
     \begin{equation}
        <n> = \frac{1}{e^{\hbar \omega}{k_B T} - 1}
     \end{equation}
     therefore the energy would be:
     \begin{equation}
        u = \sum_{k,p} \frac{\hbar \omega_{k,p}}{e^{\beta \hbar\omega_{k,p}} - 1}
     \end{equation}
     If the number of nodes in the lattice is big enough we can transform the summation into integral with a transformation multiplication:
     \begin{equation}
        \int  d_k \frac{dk}{d\omega}d\omega
     \end{equation}
     So the work is to find $d_k$. This is also equivalent with writing:
     \begin{equation}
        \int D_p(\omega)d\omega
     \end{equation}
     Here $D(\omega) $ the phonon density of state, therefore we write:
     \begin{equation}
        u = \sum_p\int d\omega D_p(\omega)\frac{\hbar \omega}{e^{\frac{\hbar\omega}{k_B T}}-1} 
     \end{equation}
     Then we have the heat capacity of lattice as:
     \begin{equation}
        C_{\text{lattice}} \equiv \frac{\partial u}{\partial T} = k_B \sum_p\int_0^\infty d\omega D(\omega)\frac{x^2 e^x}{(e^x - 1)^2}
     \end{equation}
     where $x = \frac{\hbar \omega}{k_B T}$. Now let us find the phonon density let us write again:
     \begin{equation}
        D(\omega)d\omega = d_k \frac{dk}{d\omega}d\omega
     \end{equation}
     We would use the periodic boundary conditions, which means that:
     \begin{equation}
        u(x) = u(x+L)
     \end{equation}
     Since tha lattice has a reoccuring shape this is a good boundary condition for first investigation. fFor this to happen the first thing we can write is of the form:
     \begin{equation}
        u_0 e^{kx-\omega t} = u_0 e^{k(x+L)-\omega t} 
     \end{equation}
     This means simply:
     \begin{equation}
        k = \frac{2\pi}{L}n 
     \end{equation}
     here $n$ is $0,\pm 1, \pm2, \dots, \pm N$ where $N$ is the number of cells. One can find the allowed $k$s of $k$-space (here 1 dimensional) with:
     \begin{equation}
        \text{Number of allowed } \ k \equiv \frac{\text{Volume of } \ k}{\text{Distance of nodes}} = \frac{K}{\frac{2\pi}{L}}
     \end{equation}
     We can easily write this for a $3$ dimensional lattice, by just having:
     \begin{equation}
        u(\vec x) = u(\vec x+\vec L)
     \end{equation}
     and for three dimensions the number of allowed $k$ are:
     \begin{equation}
        N_k = \frac{\frac43 \pi k^3}{\left(
            \frac{2\pi}{L_x}
        \right)\left(
            \frac{2\pi}{L_y}
        \right)\left(
            \frac{2\pi}{L_z}
        \right)} =\frac{\frac43 \pi k^3}{\frac{8\pi}{V}}
     \end{equation}  
     It is obvious that since the number of modes with a polarization $p$ between $\omega$ and $\omega +d\omega$ is as follows:
     \begin{equation}
        D(\omega)d\omega
     \end{equation}
     leading to the formula:
     \begin{equation}
        D(\omega) = \frac{dN}{d\omega} \rightarrow dN = d\omega D(\omega)
     \end{equation}
     Thus we can write :
     \begin{align}
        D(\omega) &= \frac{dN}{d\omega} \\
        &= \frac{dN}{dk}\frac{dk}{d\omega} \\
        &= \frac{4\pi k^2}{\left(\frac{2\pi}{L}\right)^3}\frac{dk}{d\omega}\\
        D(\omega)&=\frac{Vk^2}{2\pi^2}\frac{dk}{d\omega}
     \end{align}
     to have $dk/d\omega$ we have to use approximation: 1. Debay approximation 2. Einsteins approximation.
     \subsection{Debay Approximation}
     debay noticed that The first third Dispersion Curve can be approximated linearly:
     \begin{align}
        \omega &\propto k \\
        \omega &= \nu k
     \end{align}
     therefore $\frac{dk}{d\omega} =\frac1\nu$, and we can find the phonon density:
     \begin{equation}
        D(\omega) = \frac{Vk^2}{2\pi^2}\frac1\nu
     \end{equation}
     since this is an approximation we don't go over all $k$s, instead we would go until a known $k$ called debay's $k$, or $k_D$. therefor ethe number of all modes would be:
     \begin{equation}
        N = \frac{\frac43 \pi k^3_D}{(2\pi)^3/V} 
     \end{equation} 
     And the energy would become:
     \begin{equation}
        u =\int d\omega D(\omega) <n(\omega)> \hbar \omega = \int_0^{\omega_D}d\omega \left(\frac{V \omega^2}{2\pi^2 \nu^3}\right) \frac{\hbar \omega}{e^{\beta\hbar \omega} - 1}
     \end{equation}
     \begin{equation}
        u = \frac{3V\hbar}{2\pi^2\nu^3}\int_0^{\omega_D}d\omega\frac{\omega^3}{e^{\beta\hbar\omega}- 1}
     \end{equation}
     
\end{document}