\documentclass[10pt,a4paper,twocolumn]{article}
\usepackage[utf8]{inputenc}
\usepackage{amsmath}
\usepackage{amsfonts}
\usepackage{amssymb}
\usepackage{url}
\usepackage{makeidx}
\usepackage{graphicx}
\usepackage{graphicx, adjustbox}
\usepackage{lmodern}
\usepackage{fourier}
\usepackage{float}
\usepackage{caption}
\usepackage{wrapfig}
\usepackage{mhchem}
\usepackage[left=2.5cm,right=2.5cm,top=1cm,bottom=2cm]{geometry}
\usepackage{multicol}
\usepackage{soul}
%Colors
\usepackage[dvipsnames]{xcolor}


\definecolor{black}{RGB}{0, 0, 0}
\definecolor{richblack}{RGB}{7, 14, 13}
\definecolor{charcoal}{RGB}{45, 67, 77}
\definecolor{delectricblue}{RGB}{93, 117, 131}
\definecolor{cultured}{RGB}{245, 245, 245}
\definecolor{lightgray}{RGB}{211, 216, 218}
\definecolor{silversand}{RGB}{190, 194, 198}
\definecolor{spanishgray}{RGB}{148, 150, 157}
\definecolor{darkliver}{RGB}{64, 63, 76}

\colorlet{lightdelectricblue}{delectricblue!30}
\colorlet{lightdarkliver}{darkliver!30}


%ColorDefines
\newcommand{\trueblack}[1]{\textcolor{black}{#1}}
\newcommand{\rich}[1]{\textcolor{richblack}{#1}}
\newcommand{\lightblack}[1]{\textcolor{charcoal}{#1}}
\newcommand{\lightrich}[1]{\textcolor{delectricblue}{#1}}
\newcommand{\liver}[1]{\textcolor{darkliver}{#1}}

%Boxes
\usepackage{tcolorbox}
\newtcolorbox{calloutbox}{center,%
    colframe =red!0,%
    colback=cultured,
    title={Callout},
    coltitle=richblack,
    attach title to upper={\ ---\ },
    sharpish corners,
    enlarge by=0.5pt}

\newtcolorbox[use counter=equation]{eq}{center,
	colframe =red!0,
	colback=cultured,
	title={\thetcbcounter},
	coltitle=richblack,
	detach title,
	after upper={\par\hfill\tcbtitle},
	sharpish corners,
    enlarge by=0.5pt }
    
\newtcolorbox{qt}{center,
	colframe=delectricblue,
	colback=white!0,
	title={\large "},
	coltitle=delectricblue,
	attach title to upper,
	after upper ={\large "},
	sharp corners,
	enlarge by=0.5pt,
	boxrule=0pt,
	leftrule=2pt}
	
\newtcolorbox{exc}{center,%
    colframe =red!0,%
    colback=darkliver!15,
    title={Excercise},
    coltitle=richblack,
    attach title to upper={\ ---\ },
    sharpish corners,
    enlarge by=0.5pt}
    
\newcounter{theo}
\newtcolorbox[use counter=theo]{theobox}
	{center,%
    colframe =red!0,%
    colback=cultured,
    title={Theorem \thetcbcounter},
    coltitle=richblack,
    attach title to upper={\ ---\ },
    sharpish corners,
    enlarge by=0.5pt}

\newcounter{examplecounter}
\newtcolorbox[use counter=examplecounter]{example}
	{center,%
    colframe =red!0,%
    colback=cultured,
    title={Example},
    coltitle=richblack,
    attach title to upper={\ ---\ },
    sharpish corners,
    enlarge by=0.5pt}

    

        
    
% Highlighters
\newcommand{\hldl}[1]{%
	\sethlcolor{lightdarkliver}%
	\hl{#1}
}
\newcommand{\hldb}[1]{%
    \sethlcolor{lightdelectricblue}%
    \hl{#1}%
}


% Images
\newcounter{figurecounter}
\setcounter{figurecounter}{1}

\newcommand{\img}[3]{
    \begin{figure}[h!]
        \centering
        \captionsetup{justification=centering,margin=0cm,labelformat=empty}
        \includegraphics[width=#2\linewidth]{./img/#1}
        \label{figure}
        \caption{\small\textbf{fig: \thefigurecounter} -- \textcolor{darkliver}{#3}}
    \end{figure}
    \addtocounter{figurecounter}{1}}

\newcommand{\imgr}[3]{
    \begin{wrapfigure}{r}{#2\textwidth}
        \centering
        \captionsetup{justification=centering,margin=0cm,labelformat=empty}
        \includegraphics[width=\linewidth]{./img/#1}
        \label{figure}
        \caption{\small \textbf{fig: \thefigurecounter} -- \textcolor{darkliver}{#3}}
    \end{wrapfigure}
    \addtocounter{figurecounter}{1}}

\newcommand{\imgl}[3]{
    \begin{wrapfigure}{l}{#2\textwidth}
        \centering
        \captionsetup{justification=centering,margin=0cm,labelformat=empty}
        \includegraphics[width=\linewidth]{./img/#1}
        \label{figure}
        \caption{\small \textbf{fig: \thefigurecounter} -- \textcolor{darkliver}{#3}}
    \end{wrapfigure}
    \addtocounter{figurecounter}{1}}

% New commands
\newenvironment{callout}
	{\begin{calloutbox}\color{charcoal}\textbf\textit}
	{\end{calloutbox}}

\newcommand{\mev}{\text{MeV}}
\newcommand{\gev}{\text{GeV}}
\newcommand{\fpe}{4\pi\epsilon_0}
\newcommand{\ch}[5]{{}^{#2}_{#3}\!\text{#1}^{#4}_{#5}}
\newcommand{\braket}[2]{\left\langle #1 \vert #2 \right\rangle}
\newcommand{\mbraket}[3]{\left\langle #1 \vert #2 \vert #3 \right\rangle}
\newcommand{\ket}[1]{\left\vert #1 \right\rangle}
\newcommand{\bra}[1]{\left\langle #1 \right\vert} 
\newcommand{\hamiltonian}{\mathcal{H}}
\newcommand{\vect}[1]{\mathbf{#1}}
\newcommand{\cross}{\times}

\title{Electromagnetism \\ Journal}

\begin{document}
          \maketitle
          \tableofcontents
          \section{lecture 1}
               \textbf{Ohm Law}
               \begin{itemize}
                    \item Forces are two types: electric forces $q\vect E$and magnetic forces $q(\vect v \cross \vect B)$.
                    \item For a given material the current density can be written as:
                    \begin{equation}
                         \vect J = \sigma \vect F
                    \end{equation}
                    where $\sigma$ is the conductivity ($\frac1\sigma$ is resistivity). and the equation is called the Ohm Law. if the total force is mostly made of the electric force rahter than the magnetic force the ohm law can be written: 
                    \begin{equation}
                         \vect J = \sigma \vect E
                    \end{equation}              
               \end{itemize}
                                   
               \begin{exc}
                    How can we find $\sigma$ based on the statistical behaviour of the materials?
               \end{exc}

               \begin{callout}
                    Every information about the material (the mass of free charges, field of the material itself, etc... is stored in $\sigma$)
               \end{callout}
               \begin{itemize}
                    \item $\sigma$ has the dimension $1/\text{ohm}\cdot \text{meter}$.
               \end{itemize}

               \begin{exc}
                    Which shape of material can have the least resistivity? Use green functions solution to electrostatis to solve Amir.
               \end{exc}
               \begin{itemize}
                    \item Electrodynamics starts with these two new consideration.
                    \begin{align*}
                         \frac{\partial B}{\partial t} \rightarrow E\\
                         \frac{\partial E}{\partial t} \rightarrow B
                    \end{align*}
               \end{itemize}
               \begin{exc}
                    How to derive E and B with respect to each other
          \end{document}