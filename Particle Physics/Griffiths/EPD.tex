\documentclass[10pt,a4paper]{article}
\usepackage[utf8]{inputenc}
\usepackage{amsmath}
\usepackage{amsfonts}
\usepackage{amssymb}
\usepackage{url}
\usepackage{makeidx}
\usepackage{graphicx}
\usepackage{graphicx, adjustbox}
\usepackage{lmodern}
\usepackage{fourier}
\usepackage{float}
\usepackage{caption}
\usepackage{wrapfig}
\usepackage{mhchem}
\usepackage[left=2.5cm,right=2.5cm,top=1cm,bottom=2cm]{geometry}
\usepackage[dvipsnames]{xcolor}
\usepackage{multicol}
\usepackage{soul}
\usepackage{tikz}
\usepackage{tikz-feynman}
%Colors

\definecolor{black}{RGB}{0, 0, 0}
\definecolor{richblack}{RGB}{7, 14, 13}
\definecolor{charcoal}{RGB}{45, 67, 77}
\definecolor{delectricblue}{RGB}{93, 117, 131}
\definecolor{cultured}{RGB}{245, 245, 245}
\definecolor{lightgray}{RGB}{211, 216, 218}
\definecolor{silversand}{RGB}{190, 194, 198}
\definecolor{spanishgray}{RGB}{148, 150, 157}
\definecolor{darkliver}{RGB}{64, 63, 76}

\colorlet{lightdelectricblue}{delectricblue!30}
\colorlet{lightdarkliver}{darkliver!30}


%ColorDefines
\newcommand{\trueblack}[1]{\textcolor{black}{#1}}
\newcommand{\rich}[1]{\textcolor{richblack}{#1}}
\newcommand{\lightblack}[1]{\textcolor{charcoal}{#1}}
\newcommand{\lightrich}[1]{\textcolor{delectricblue}{#1}}
\newcommand{\liver}[1]{\textcolor{darkliver}{#1}}

%Boxes
\usepackage{tcolorbox}
\newtcolorbox{calloutbox}{center,%
    colframe =red!0,%
    colback=cultured,
    title={Callout},
    coltitle=richblack,
    attach title to upper={\ ---\ },
    sharpish corners,
    enlarge by=0.5pt}

\newtcolorbox[use counter=equation]{eq}{center,
	colframe =red!0,
	colback=cultured,
	title={\thetcbcounter},
	coltitle=richblack,
	detach title,
	after upper={\par\hfill\tcbtitle},
	sharpish corners,
    enlarge by=0.5pt }
    
\newtcolorbox{qt}{center,
	colframe=delectricblue,
	colback=white!0,
	title={\large "},
	coltitle=delectricblue,
	attach title to upper,
	after upper ={\large "},
	sharp corners,
	enlarge by=0.5pt,
	boxrule=0pt,
	leftrule=2pt}
	
\newtcolorbox{ecx}{center,%
    colframe =red!0,%
    colback=darkliver!15,
    title={Exercise},
    coltitle=richblack,
    attach title to upper={\ ---\ },
    sharpish corners,
    enlarge by=0.5pt}
    
\newcounter{theo}
\newtcolorbox[use counter=theo]{theobox}
	{center,%
    colframe =red!0,%
    colback=cultured,
    title={Theorem \thetcbcounter},
    coltitle=richblack,
    attach title to upper={\ ---\ },
    sharpish corners,
    enlarge by=0.5pt}

\newcounter{examplecounter}
\newtcolorbox[use counter=examplecounter]{example}
	{center,%
    colframe =red!0,%
    colback=cultured,
    title={Example \thetcbcounter},
    coltitle=richblack,
    attach title to upper={\ ---\ },
    sharpish corners,
    enlarge by=0.5pt}

    

        
    
% Highlighters
\newcommand{\hldl}[1]{%
	\sethlcolor{lightdarkliver}%
	\hl{#1}
}
\newcommand{\hldb}[1]{%
    \sethlcolor{lightdelectricblue}%
    \hl{#1}%
}


% Images
\newcounter{figurecounter}
\setcounter{figurecounter}{1}

\newcommand{\img}[3]{
    \begin{figure}[h!]
        \centering
        \captionsetup{justification=centering,margin=0cm,labelformat=empty}
        \includegraphics[width=#2\linewidth]{./img/#1}
        \label{figure}
        \caption{\small\textbf{fig: \thefigurecounter} -- \textcolor{darkliver}{#3}}
    \end{figure}
    \addtocounter{figurecounter}{1}}

\newcommand{\imgr}[3]{
    \begin{wrapfigure}{r}{#2\textwidth}
        \centering
        \captionsetup{justification=centering,margin=0cm,labelformat=empty}
        \includegraphics[width=\linewidth]{./img/#1}
        \label{figure}
        \caption{\small \textbf{fig: \thefigurecounter} -- \textcolor{darkliver}{#3}}
    \end{wrapfigure}
    \addtocounter{figurecounter}{1}}

\newcommand{\imgl}[3]{
    \begin{wrapfigure}{l}{#2\textwidth}
        \centering
        \captionsetup{justification=centering,margin=0cm,labelformat=empty}
        \includegraphics[width=\linewidth]{./img/#1}
        \label{figure}
        \caption{\small \textbf{fig: \thefigurecounter} -- \textcolor{darkliver}{#3}}
    \end{wrapfigure}
    \addtocounter{figurecounter}{1}}

% New commands
\newenvironment{callout}
	{\begin{calloutbox}\color{charcoal}\textbf\textit}
	{\end{calloutbox}}

\newcommand{\mev}{\text{MeV}}
\newcommand{\gev}{\text{GeV}}
\newcommand{\fpe}{4\pi\epsilon_0}
\newcommand{\ch}[5]{{}^{#2}_{#3}\!\text{#1}^{#4}_{#5}}
\newcommand{\electron}{\ch{e}{}{}{-}{}}
\newcommand{\positron}{\ch{e}{}{}{+}{}}
\newcommand{\proton}{\ch{p}{}{}{}{}}
\newcommand{\muon}{\ch{\mu}{}{}{-}{}}
\newcommand{\neutron}{\ch{n}{}{}{}{}}
\newcommand{\neutrino}[1]{\ch{\nu}{}{}{}{#1}}
\newcommand{\braket}[2]{\left\langle #1 \vert #2 \right\rangle}
\newcommand{\mbraket}[3]{\left\langle #1 \vert #2 \vert #3 \right\rangle}
\newcommand{\ket}[1]{\left\vert #1 \right\rangle}
\newcommand{\bra}[1]{\left\langle #1 \right\vert} 
\newcommand{\hamiltonian}{\mathcal{H}}

\title{Elementary Particle Dynamic}
\author{Amir H. Ebrahimnezhad}
\date{}

\begin{document}
        \maketitle
        \tableofcontents
        \section{Quantum Electrodynamics (QED)}
            \textit{This chapter introduces the fundamental forces by which elementary particles interact, and the Feynman diagram we use to represent these interactions. The treatment is entirely qualitative and can be read quickly to get a sense of the 'lay of the land'.}
            There are four fundamental forces in nature: strong, electromagnetic, weak and gravitation. Quantum Electrodynamics is the oldest, the simplest, and the most successful of the dynamical theories; the others are self-consiously modeled on it. All electromagnetic phenomena are ultimately reducible to the following elementary process:
            \\
            \\
            \img{pfinteraction}{0.3}{Photon Fermion interaction, the most simple case.}
            
            To describe more complicated processes, we simply combine two or more repliicas of this primitive vertex. As an example the electromagnetic interaction betweeen two electrons that would cause repulsive force can be shown as: The process is: $\electron \electron \rightarrow \electron \electron$.

            \img{mollerscattering}{0.2}{Electrons repulsion. Known as Møller Scattering}
            
            In another case a pair of electron-positron can attract each other. The diagram is known as Bhabha scattering and is the following process: $\electron \positron \rightarrow \electron \positron$

            \img{bhabhascattering}{0.2}{Electron Positron attraction. Known as Bhabha Scattering}

            As long as the vertex we are using is the same as fig.1. the diagram is valid in QED. The external lines are telling what process is happening and the internal lines are showing the mechanism for that process.
            \begin{callout}
                In terms of Feynman diagrams, crossing symmetry corresponds to twisting or rotating the figure. If we allow more vertices, the possibilities repidly proliferate. The 'innards' of the diagram are irrelevant as far as the observed process is concerned. Internal lines represent particles that are not observed -- indeed, that cannot be observed without entierly changing the process. We call them virtual particles. Only the external lines represent 'real' particles.
            \end{callout}
            \textbf{Feynman diagrams meaning:} Feynman diagrams are purely symbolic; they do not represent particle trajectories. The horizontal dimension is time, but vertical spacing does not correspond to physical separation. For instance, in Bhabha scattering the electron and positron are attracted, not repelled. All that the diagram says is: \textit{Once there was an electron and a positron; they exchanged a photon; then there was an electron and a positron again.}
            \\
            \\
            \indent Quantitatively, eahc feynman diagram stands for a particular number, which can be calculated using the so-called Feynaman rules. The sum total of all Feynman diagrams with the given external lines represents the actual pphysica process. Of course, there's a wee problem here: There are infinitely many feynman diagrams fo any particular reaction! Fortunately, each vertex within a diagram introducesa a factor of $\alpha = e^2/\habr c$, the fine structure constant. Because this is such a small number, diagrams with more 

\end{document}