\documentclass[10pt,a4paper]{article}
\usepackage[utf8]{inputenc}
\usepackage{amsmath}
\usepackage{amsfonts}
\usepackage{amssymb}
\usepackage{url}
\usepackage{makeidx}
\usepackage{graphicx}
\usepackage{graphicx, adjustbox}
\usepackage{lmodern}
\usepackage{fourier}
\usepackage{float}
\usepackage{caption}
\usepackage{wrapfig}
\usepackage{mhchem}
\usepackage[left=2.5cm,right=2.5cm,top=1cm,bottom=2cm]{geometry}
\usepackage[dvipsnames]{xcolor}
\usepackage{multicol}
\usepackage{soul}
\usepackage{tikz-feynman} 
%Colors

\definecolor{black}{RGB}{0, 0, 0}
\definecolor{richblack}{RGB}{7, 14, 13}
\definecolor{charcoal}{RGB}{45, 67, 77}
\definecolor{delectricblue}{RGB}{93, 117, 131}
\definecolor{cultured}{RGB}{245, 245, 245}
\definecolor{lightgray}{RGB}{211, 216, 218}
\definecolor{silversand}{RGB}{190, 194, 198}
\definecolor{spanishgray}{RGB}{148, 150, 157}
\definecolor{darkliver}{RGB}{64, 63, 76}

\colorlet{lightdelectricblue}{delectricblue!30}
\colorlet{lightdarkliver}{darkliver!30}


%ColorDefines
\newcommand{\trueblack}[1]{\textcolor{black}{#1}}
\newcommand{\rich}[1]{\textcolor{richblack}{#1}}
\newcommand{\lightblack}[1]{\textcolor{charcoal}{#1}}
\newcommand{\lightrich}[1]{\textcolor{delectricblue}{#1}}
\newcommand{\liver}[1]{\textcolor{darkliver}{#1}}

%Boxes
\usepackage{tcolorbox}
\newtcolorbox{calloutbox}{center,%
    colframe =red!0,%
    colback=cultured,
    title={Callout},
    coltitle=richblack,
    attach title to upper={\ ---\ },
    sharpish corners,
    enlarge by=0.5pt}

\newtcolorbox[use counter=equation]{eq}{center,
	colframe =red!0,
	colback=cultured,
	title={\thetcbcounter},
	coltitle=richblack,
	detach title,
	after upper={\par\hfill\tcbtitle},
	sharpish corners,
    enlarge by=0.5pt }
    
\newtcolorbox{qt}{center,
	colframe=delectricblue,
	colback=white!0,
	title={\large "},
	coltitle=delectricblue,
	attach title to upper,
	after upper ={\large "},
	sharp corners,
	enlarge by=0.5pt,
	boxrule=0pt,
	leftrule=2pt}
	
\newtcolorbox{ecx}{center,%
    colframe =red!0,%
    colback=darkliver!15,
    title={Exercise},
    coltitle=richblack,
    attach title to upper={\ ---\ },
    sharpish corners,
    enlarge by=0.5pt}
    
\newcounter{theo}
\newtcolorbox[use counter=theo]{theobox}
	{center,%
    colframe =red!0,%
    colback=cultured,
    title={Theorem \thetcbcounter},
    coltitle=richblack,
    attach title to upper={\ ---\ },
    sharpish corners,
    enlarge by=0.5pt}

\newcounter{examplecounter}
\newtcolorbox[use counter=examplecounter]{example}
	{center,%
    colframe =red!0,%
    colback=cultured,
    title={Example \thetcbcounter},
    coltitle=richblack,
    attach title to upper={\ ---\ },
    sharpish corners,
    enlarge by=0.5pt}

    

        
    
% Highlighters
\newcommand{\hldl}[1]{%
	\sethlcolor{lightdarkliver}%
	\hl{#1}
}
\newcommand{\hldb}[1]{%
    \sethlcolor{lightdelectricblue}%
    \hl{#1}%
}


% Images
\newcounter{figurecounter}
\setcounter{figurecounter}{1}

\newcommand{\img}[3]{
    \begin{figure}[h!]
        \centering
        \captionsetup{justification=centering,margin=0cm,labelformat=empty}
        \includegraphics[width=#2\linewidth]{./img/#1}
        \label{figure}
        \caption{\small\textbf{fig: \thefigurecounter} -- \textcolor{darkliver}{#3}}
    \end{figure}
    \addtocounter{figurecounter}{1}}

\newcommand{\imgr}[3]{
    \begin{wrapfigure}{r}{#2\textwidth}
        \centering
        \captionsetup{justification=centering,margin=0cm,labelformat=empty}
        \includegraphics[width=\linewidth]{./img/#1}
        \label{figure}
        \caption{\small \textbf{fig: \thefigurecounter} -- \textcolor{darkliver}{#3}}
    \end{wrapfigure}
    \addtocounter{figurecounter}{1}}

\newcommand{\imgl}[3]{
    \begin{wrapfigure}{l}{#2\textwidth}
        \centering
        \captionsetup{justification=centering,margin=0cm,labelformat=empty}
        \includegraphics[width=\linewidth]{./img/#1}
        \label{figure}
        \caption{\small \textbf{fig: \thefigurecounter} -- \textcolor{darkliver}{#3}}
    \end{wrapfigure}
    \addtocounter{figurecounter}{1}}

% New commands
\newenvironment{callout}
	{\begin{calloutbox}\color{charcoal}\textbf\textit}
	{\end{calloutbox}}

\newcommand{\mev}{\text{MeV}}
\newcommand{\gev}{\text{GeV}}
\newcommand{\fpe}{4\pi\epsilon_0}
\newcommand{\ch}[5]{{}^{#2}_{#3}\!\text{#1}^{#4}_{#5}}
\newcommand{\electron}{\ch{e}{}{}{-}{}}
\newcommand{\positron}{\ch{e}{}{}{+}{}}
\newcommand{\proton}{\ch{p}{}{}{}{}}
\newcommand{\muon}{\ch{\mu}{}{}{-}{}}
\newcommand{\neutron}{\ch{n}{}{}{}{}}
\newcommand{\neutrino}[1]{\ch{\nu}{}{}{}{#1}}
\newcommand{\braket}[2]{\left\langle #1 \vert #2 \right\rangle}
\newcommand{\mbraket}[3]{\left\langle #1 \vert #2 \vert #3 \right\rangle}
\newcommand{\ket}[1]{\left\vert #1 \right\rangle}
\newcommand{\bra}[1]{\left\langle #1 \right\vert} 
\newcommand{\hamiltonian}{\mathcal{H}}
\newcommand{\hatvec}[1]{\hat{\vec #1}}
\title{Particle Physics}

\begin{document}
          \maketitle
          \tableofcontents

          \section{Lecture 1}
          \subsection{Interactions}
          \textbf{Electromagnetic Force:}
          \begin{itemize}
               \item  The mediator is photon with $m_\gamma =0$ 
               \item the interaction is shown in the feynman diagram as a point which is called vertex which has a magnitude, showing the magnitude of our interaction.
               \item any fermions (not considering neutrinos) can have electromagnetic interaction. the coupling constant (the magnitude of interaction) for charged leptons (electron muon and tau) is $-e$, and for quarks are $\frac23 e$ or $-\frac13 e$.
               \item electromagnetic interaction wouldn't chage the flavour, it doesn't have flavour changing. If an electron interacts with a photon, the electron wouldn't change flavour (it won't become a muon!)
               \item the theory that any force needs a mediator comes from Quantum Fiels Theory.
               \item The quantum field theory of electrodynamics is QED (Quantum Electrodnamics.)
          \end{itemize}
          \textbf{Strong Interaction:}
          \begin{itemize}
               \item the mediator is gluon (there's actually eight gluons) and their mass is zero, $m_g=0$ 
               \item The theory is called Quantum Chromodynamics (QCD).
               \item only quarks can have strong interaction. Coupling of strong interaction $g_s$ need colour charge.
               \item interacting with glouns, just like photons, won't change flavour (the interaction of an up quark with gluon won't change it to a down quark, or any quark for that matter).
          \end{itemize}
          \textbf{Weak Interaction}
          \begin{itemize}
               \item The first type mediator is $W^{\pm}$ which also has a mass, $m_W = 80.1 \gev$.
               \item the other type is $Z^0$ which has $m_Z = 90.1 \gev$.
               \item People have always tried to make a grand theory, first maxwell showed electric and magnetic forces are the same and then people showed that the electromagnetic and weak interactions can be thought as one calling the theory electroweak interaction.
               \item $Z^0$ interacts with every fermions (including neutrinos!), the coupling $g_Z$. and we don't have any flavour changing.
               \item $W^\pm$ is different, it should change the flavour because of charge conservation. an electron interacting with $W^-$ will become a $\nu_e$.
               \begin{align}
                    \begin{pmatrix}
                         \nu_e \\ e
                    \end{pmatrix} ,
                    \begin{pmatrix}
                         \nu_\mu \\ \mu
                    \end{pmatrix} ,
                    \begin{pmatrix}
                         \nu_\tau \\ \tau
                    \end{pmatrix} 
               \end{align}
               and for quarks it's more complicated:
               
               \begin{align*}
                    \begin{pmatrix}
                         u \\ d
                    \end{pmatrix} ,
                    \begin{pmatrix}
                         u \\ s
                    \end{pmatrix} ,
                    \begin{pmatrix}
                         c \\ b
                    \end{pmatrix} ,
                    \begin{pmatrix}
                         c \\ d
                    \end{pmatrix} ,
                    \begin{pmatrix}
                         c \\ s
                    \end{pmatrix} ,
                    \begin{pmatrix}
                         c \\ b
                    \end{pmatrix} ,
                    \begin{pmatrix}
                         t \\ d
                    \end{pmatrix} ,
                    \begin{pmatrix}
                         t \\ s
                    \end{pmatrix} ,
                    \begin{pmatrix}
                         t \\ b
                    \end{pmatrix} ,
               \end{align*}
          \end{itemize}
          \begin{callout}
               The things we have siad are not the interactions, their are allowed vertexes in standard model.
          \end{callout}
          \begin{callout}
               the charge of every mediator but $W^\pm$ is zero. This means that we have no flavour changing neutral current (FCNC), but rahter Flavour Changing Charged Current (FCCC).
          \end{callout}
          \subsection{A general form of reactions}
          A general form of any reaction is:
          \begin{equation}
               a + b \rightarrow c+d
          \end{equation}
          for example a simple reaction:
          \begin{equation}
               \electron +\electron\rightarrow\electron +\electron
          \end{equation}
          or another example:
          \begin{equation}
               \electron +\nu_e \rightarrow\electron+\nu_e
          \end{equation}
          \begin{callout}
               These are Fermions interactions, the Bosons can interact (other than photon) with each other.
          \end{callout}
     \subsection{Hadrons}
     \begin{itemize}
          \item Hadrons are not elementary particles and are made of quarks:
          \item There are two groups of hadrons, Mesons and Baryons
          \item Baryons are made of three quarks or three anti quarks, proton ($\ket{uud}$) and neutron ($\ket{ddu}$) are examples.
          \item Mesons are made of a pair of quark anti quarks, $\Pi^+$ ($\ket{u\bar d}$) is an example.
          \item \textbf{Proton:} it's a stable particle with mass $m_P = 938.27\mev$, found in 1919, the life-time of a proton is: $2.1 \time 10^{29} \text{years}$!. with spin $\frac12$.
          \item \textbf{Neutron:} is not stable and has the mass $m_N = 939.56\mev$ ($m_P-m_N = 1.29\mev$) has the lifetime $15\text{min}$ (as a free particle).with spin $\frac12$.
          \item Baryons doesn't have only three quarks, the three quarks that make the attributes of the brayons are called valance quarks, but there are many other quarks in a baryon, generally called the sea quarks.
     \end{itemize}
     \begin{callout}
          How did they found experimentally the lifetime of proton? Solve
     \end{callout}
     \section{Lecture 2}
          \subsection{SI units vs Natural units}
          \begin{itemize}
               \item In SI units we use kg for mass meter for length and second for time. This is not the case for standard model and elementary particle physics. Because SM comes from field theory and there we have two constants $c =2.998\times 10^8 \frac ms$ and $ \hbar \frac{h}{2\pi} = 2.055\times 10^{-34}Js$.
               \item the Natural units are as below:
               \begin{itemize}
                    \item \textbf{Energy: } in SI units is Joule $J=kgm^2s^{-2}$, and in natural units we use electron volt, $eV = 1.602\times 10^{-19}\times 1 = 1.602\times 10^{-19}$ which means that, $1\gev \approx 10^{-10}J$.
                    \item \textbf{Momentum: } in SI units is $kgms^{-1}$, and in natural units we use $E/c$ or in terms of unit $\gev/c$. where $c$ is the speed of light.
                    \item \textbf{Mass: } in SI units is kilogram, and in natural units we use $\gev/c^2$. 
                    \item \textbf{Time: } in SI it's seconds, and in natural units it is $(\gev/\hbar)^{-1}$.
                    \item \textbf{Length: } in SI it's meterm and in natural units it's $(\gev/\hbar c)^{-1}$.
                    \item \textbf{Assumption:} we can assume that $c = \hbar = 1$ then all the units become either $\gev$ or $\gev^{-1}$
                    \item \textbf{Returning:} The following calculations are to get back to SI units from the Natural units:
                    \begin{align*}
                         \frac{\hbar c}{\gev} = 0.197 fm\\
                         \frac{\hbar}{\gev} = 0.658\times 10^{-24}s\\
                         \frac{\gev}{c^2} = 0.178\time10^{-26}kg
                    \end{align*}

                    \item \textbf{Cross section: } the dimension is $[L^2]$, in SI $m^2$, and in natural units it can be$\gev^{-2}$, but since in Particle physics, we define barn $b = 10^{-28}m^2$ since it's $2576.72\gev^{-2}$ we would probably use $fb, pb$ more.
                    \item \textbf{heaviside-lorentz: } because $c =1 =\frac{1}{\epsilon_0\mu_0}$ we assume $\epsilon_0,\mu_0$ as one too.
                    \item heaviside-lorentz $\alpha = \frac{e^2}{4\pi\epsilon_0\hbar c} \rightarrow \alpha = \frac{e^2}{4\pi} \rightarrow e^2 = 4\pi\alpha$.
               \end{itemize}
          \end{itemize}
          \subsection{Interactions, Order, Coupling and Energy}
          \begin{itemize}
               \item \textbf{Scattering: } in scattering we generally have: $a+b\rightarrow c+d$[Write the schematic of collision and the feynman diagram Amir.], The probability of this happening is given through the cross section.
               \item \textbf{Decay: } decay is generally $a\rightarrow b+c$[Write the schematic of collision and the feynman diagram Amir.], The probability is given through the width of decay [pahnaye vapashi.]
               \item \textbf{Orders: } loops and the convergence of lower orders of diagrams.[research about them. Leading order diagrams, Next leading order diagram.] The reason of the convergence is becase the probability that is calculated through $M^N = \alpha^{2N}$ with $N$ being the number of the leading order, converges since $\alpha =\frac{1}{137}$ but $\alpha$ depends on the energy.
               \item \textbf{Coupling and Energy}: calculations in field theory shows that the convergence of next order diagrmas is not the case in Quantum Chromodynamics (QCD), since as energy increases $\alpha$ increases aswell and from a point it is larger than one meaning that the formula in the last point does not converge but rather diverge. The point were $\alpha_{QED}$ becomes one is so far (on energy spectrum) that we have not yet even reached in experiments so everything works well. But for $\alpha{QCD}$ it's reversed. it is larger in lower energies nad smaller in higher energies it is fine. Thus in this course we only work in the areas that we are able to work using SM and perturbation theory.
          \end{itemize}
          \begin{ecx}
               find the newton constant for gravity, plank mass, protons mass and protons radius in terms of natural units.
          \end{ecx}
     \section{Lecture 3}
          \subsection{QED}
          \begin{itemize}
               \item \textbf{Lepton Number:} Any lepton (electron, tau, muon and their neutrinos) has the lepton number $L = +1$, and their anti-particles $L = - 1$. 
               \item \textbf{Conservation of Lepton Number:} considering the reaction below:

          
          \begin{equation}
          \mu^{-} \rightarrow e^{-} \gamma
          \end{equation}
          This reaction is not possible, though the electric charge and the Lepton number is conserved, we have to look for electron number and muon number conservation aswell. Thus it's not possible. A possible decay for muon would be:

          
          \begin{equation}
               \mu^{-1} \rightarrow e^{-1} \bar{\nu_{e}}  \nu_{\mu}
          \end{equation}

          \item \textbf{Baryon Number:} for each quark we have baryon number $\frac13$ and thus Baryons have $B=1$, like proton and neutron. anti Baryons which have anti quark $B=-1$ and thus each anti quark has $-\frac13$. As an example the reaction below is possible.

          \begin{equation}
               n \rightarrow\proton e \bar{\nu_{e}} 
          \end{equation}

          \item another possible interaction is:
          \begin{equation}
               e^{-}e^{+} \rightarrow \gamma \gamma
          \end{equation}
          

          \end{itemize}
          \begin{callout}
               anti-particles arrow is opposite to it's momentum direction
          \end{callout}
     \section{Lecture 4}
          - \textbf{Absent}
     \section{Lecture 5}
          \begin{itemize}
               \item In weak interaction we have to mediator $Z$ and $W$ bosons. Leptons interact with this would conserve electron, tau, and muon number. But this is not the case in quarks, as we mentioned earlier there's a possibility to change the family via weak interaction.
               \begin{align*}
                    \begin{pmatrix}
                         u \\ d
                    \end{pmatrix} ,
                    \begin{pmatrix}
                         u \\ s
                    \end{pmatrix} ,
                    \begin{pmatrix}
                         c \\ b
                    \end{pmatrix} ,
                    \begin{pmatrix}
                         c \\ d
                    \end{pmatrix} ,
                    \begin{pmatrix}
                         c \\ s
                    \end{pmatrix} ,
                    \begin{pmatrix}
                         c \\ b
                    \end{pmatrix} ,
                    \begin{pmatrix}
                         t \\ d
                    \end{pmatrix} ,
                    \begin{pmatrix}
                         t \\ s
                    \end{pmatrix} ,
                    \begin{pmatrix}
                         t \\ b
                    \end{pmatrix} ,
               \end{align*}
               for instance the interaction below changes a strange quark into an up quark.
               \begin{equation}
                    \Lambda^0 (uds)\rightarrow p(uud) + \Pi^-(\bar u d)
               \end{equation}
                    Cabbibo 1963 and Glashow, illopoilos-Maiani (GIM) 1970 found the theorem. Kobogashi and Maskawa in 1973 theorized that the participated quark in weak interaction has the possibility of being three quarks, which is written in matrix form:
               \begin{equation}
                    \begin{pmatrix}
                         d'\\
                         s'\\
                         b'\\
                    \end{pmatrix}
                    =\begin{pmatrix}
                         V_{ud} & V_{us} & V_{ub}\\
                         V_{cd} & V_{cs} & V_{cb}\\
                         V_{td} & V_{ts} & V_{tb}
                    \end{pmatrix}
                    \begin{pmatrix}
                         d\\ s\\ b
                    \end{pmatrix}
               \end{equation}
               The matrix is called the CKM matrix. so the right form to write the interaction is:
               \begin{equation}
                    \begin{pmatrix}
                         u\\
                         d'
                    \end{pmatrix},
                    \begin{pmatrix}
                         c\\
                         s'
                    \end{pmatrix},
                    \begin{pmatrix}
                         t\\
                         b'
                    \end{pmatrix}
               \end{equation}
               and it is to say:
               \begin{equation}
                    \begin{pmatrix}
                         u \\
                         d'
                    \end{pmatrix}
                    = \begin{pmatrix}
                         u \\
                         V_{ud} d + V_{us} S + V_{ub} b
                    \end{pmatrix}
               \end{equation}
               \begin{callout}
                    There's a similar matrix for leptons called PMNS, but not in this course since the matrix enters only when we assume the neutrinos mass to be non zero (which we have not).
               \end{callout}
               The CKM matrix values are as follows:
               \begin{equation}
                    \begin{pmatrix}
                         0.974 & 0.22 & 0.006 \\
                         0.22 & 0.973 & 0.14 \\
                         0.008& 0.042 & .99
                    \end{pmatrix}
               \end{equation}
               \begin{callout}
                    Since we don't have PMNS matrix in this course we have the generation conservation (electron, muon, tau number conservation) but this is not the case for quarks since we have CKM matrix. What we have for quarks is the baryon number conservation: each quarks baryon number is $+\frac13$ and for anti quarks $-\frac13$
               \end{callout}
               \item Remebering in QCD, Gluons interact with themselves too, weak interaction also has interactions with eachother, and also with photons, (picture is taken 26,02, 2023)
               \item \textbf{Decay and Conservation Laws: } First consider that particles must decay to other particles, which are lighter (less mass) than them, Particles that do not decay are as follows:
               \begin{itemize}
                    \item $\gamma$, photon do not decay since the mass it has is zero! it would violate the energy conservation.
                    \item $\electron$, electron does not decay because it would violate charge conservation since there's no lighter particle with negative charge.
                    \item $\proton$, proton and anti proton does not decay because it would violate Baryon number.
                    \item $nu$, neutrinos would not decay because then it would violate lepton number, since they are the lightest particles with lepton numbers.
                    \item Some decays that do not violate conservation laws are:

                    %\begin{align*}
                       %  \mu^- \rightarrow \electron \bar\nu_e\nu_\mu \\
                       %  \tau^-\rightarrow \electron \bar\nu_e\nu_\tau\\
                       %  \tau^-\rightarrow \mu^- \bar\nu_\mu\nu_\tau\\
                       %  \Pi^+ \rightarrow \mu^ \bar\nu_\mu\\
                       % \Pi^0 \rightarrow \gamma\gamma\\
                       %  \Delta^{++} \rightarrow p^+ \Pi^+\\
                       %  \Sigma^- \rightarrow n \electron \bar\nu_e
                    %%%\end{align*}
                    All three fundamental interactions can cause decay and the decays would have similar lifetime (orders).
                    \begin{equation}
                         \text{decay}\left\{\begin{matrix}
                              \text{interction} & \text{order of} \tau(s)\\
                              \text{Strong} & 10^{-23}s\\
                              \text{EM} & 10^{-16}s\\
                              \text{Weak} & 10^{-13 }s \text{or for the case of neutron} 15 min
                         \end{matrix}
                         \right.
                    \end{equation}
                    \begin{callout}
                         The reason for neutron, that it takes a lot longer to decay is that in:
                         \begin{equation}
                              n\rightarrow p \electron +\bar{\nu_e}
                         \end{equation}
                         the mass difference is small in both sides of the equation.
                         The thing to note that the larger difference of mass between sides of interaction, the less time it takes to decay.
                    \end{callout}
                    So the conservations that we have to check for any decay to be possible is:
                    \begin{itemize}
                         \item Conservation of energy
                         \item Conservation of angular momentum
                         \item Conservation of electric charge
                         \item Conservation of strong charge (colour)
                         \item Conservation of Baryon number
                         \item Conservation of Flavour, for photon gluon and $Z$ boson.
                    \end{itemize}
                    \item \textbf{OZI rule:} People found a meson that is made of $c\bar c$  called, $J/\psi$ (yea that's both names at the same time!) This meson decys via strong interaction with the order $10^{-20}s$ a thousand times larger than the order of decays for other decays via stron interaction. Before we begin with the reason, look at the other meson called $\phi (s\bar s)$ that can decay into $3\Pi$ or $2K$ [picture 26,2,2023].
                    The OZI rule states that if you go through them with a single line (just like drawing a straight line that can cut all the gluons.) in their feynman diagram. it is suppressed and is lasting longer, same thing happens for $J/\psi$.[To be read about online.]
               \end{itemize}
          \end{itemize}
     \section{Lecture 6}
          
          \textbf{Special Relativity review:} Einsteins relativity showed that there's a finite limit to the speed which is $c$, the speed of light. Beside the limit there are two assumptions for the special relativity.
          \begin{itemize}
               \item The laws of physics are invariant in any inertial system.
               \item The speed of light is constant and the same in any inertial refrence frame.
          \end{itemize}
          to move between two frames we have to use what is known as the Lorentz transformations, instead of Galilei Transformations.
          \begin{equation}
               X' = \begin{pmatrix}
                         t' \\ x' \\ y'\\ z'
                    \end{pmatrix},
               X = \begin{pmatrix}
                         t\\ x\\ y\\ z
                    \end{pmatrix}
          \end{equation}
          And the tranformations are:
          \begin{equation}
               \gamma = \frac{1}{\sqrt{1-v^2}}
          \end{equation}
          So for $X$ and $X'$ which is moving along the $z$ axis we have:
          \begin{equation}
               \left\{  \begin{matrix}
                    t' = \gamma(t-v z)\\
                    x' = x
                    y' = y
                    z' = \gamma(z- vt)
               \end{matrix}\right.
          \end{equation}
          and in the matrix form we have:
          \begin{equation}
               \Lambda =\begin{pmatrix}
                    \gamma &0&0& - \gamma\beta\\
                    0&1&0&0\\
                    0&0&1&0\\
                    \gamma\beta&0&0&\gamma
               \end{pmatrix}
          \end{equation}
          and for components we have:
          \begin{equation}
               x'^{\mu} = \Lambda^{\mu}_{\nu}x^{\nu}
          \end{equation}
          to make upper indicies down we use the metric
          \begin{equation}
               g_{\mu\nu} =\begin{pmatrix}
                    1&0&0&0\\
                    0&-1&0&0\\
                    0&0&-1&0\\
                    0&0&0&-1
               \end{pmatrix}
               \Rightarrow g_{\mu\nu} x^{\nu} = x_{\mu}
          \end{equation}
          because we have for light:
          \begin{equation}
               a^2 = 0
          \end{equation}
          For a four-vector we define:
          \begin{equation}
               \left\{\begin{matrix}
                    a^2 >0 \ \text{timelike}\\
                    a^2 <0 \ \text{spacelike}\\
                    a^2 =0 \ \text{lightlike}
               \end{matrix}\right.
          \end{equation}
          \begin{ecx}
               Prove eq 20.
          \end{ecx}
          \begin{ecx}
               Show that for a relativ length with speed $v$ the stationary observer would calculate:
               \begin{align*}
                    L = \frac{L'}{\gamma}
               \end{align*}
               \\
               also show that time lengths becomes:
               \begin{align*}
                    \Delta t = \gamma\Delta t'
               \end{align*}
          \end{ecx}
          We can the calculate the speed:
          \begin{align*}
               \Delta z &= \gamma (\Delta z' + v\Delta t')\\
               \Delta t &= \gamma (\Delta t' + v \Delta z')\\
               \Rightarrow  u &=\frac{\Delta z}{\Delta t} = \frac{\Delta z'+v\Delta t'}{\Delta t' + \frac{v}{c^2}\Delta z'}\\
               &= \frac{\frac{\Delta z'}{\Delta t'}+v}{1+\frac{v}{c^2}\frac{\Delta z'}{\Delta t'}}\\
               \Rightarrow u &= \frac{u' + v}{1+\frac{vu'}{c^2}}
          \end{align*}
          As we seen a four-vector evolves using eq20. But we can have other objects, or better said higher rank tensors for example:
          \begin{equation}
               \begin{matrix}
                    \text{scalar} & s' = s
                    \text{four-vector (1st rank tensor)} & x^\mu = \Lambda^{\mu_\nu} x^{\nu}\\
                    \text{2nd rank tensor} S^{\mu\nu'} = \Lambda^\mu_\alpha \Lambda^\nu_\beta S^{\alpha\beta}\\
                    \vdots
               \end{matrix}
          \end{equation}
          \textbf{Energy-momentum four-vector:} we have the same analogy for momentum energy space leading to:
          \begin{equation}
               P=(E,\vec p)\rightarrow P\cdot P = E^2 - \vec p\cdot\vec p
          \end{equation}   
          \begin{ecx}
               What is the momentum inner product of a system of particles:
          \end{ecx}
          \textbf{derviation four-vector:} We can define a Four-vector operator for derivation as below:
          \begin{align}
               \partial_\mu = \frac{\partial}{\partial x^\mu} = \left(\frac{\partial}{\partial t} ,\frac{\partial}{\partial x^1},\frac{\partial}{\partial x^2},\frac{\partial}{\partial x^3}\right)\\
               \partial^\mu = \frac{\partial}{\partial x_\mu} = \left(\frac{\partial}{\partial t} ,\frac{\partial}{\partial x_1},\frac{\partial}{\partial x_2},\frac{\partial}{\partial x_3}\right)
          \end{align}
          \textbf{collision:} In classical kinematics we had two conservation laws: 1. Momentu conervation and Mass conservation. In collisions between elementary particles we have the momentum conservation but we can have the masses different at the end of collision thus we got 
          \begin{itemize}
               \item Momentum conservation $\rightarrow \vec P_a +\vec P_b = \vec P_c + \vec P_d$.
               \item Energy conservation   $\rightarrow E_a +E_b = E_c+E_d$.
          \end{itemize}
          Since the mass can vary we have three condition to consider:
          \begin{itemize}
               \item \textbf{Kinetic energy is conseved}, which means the collision was elastic
               \item \textbf{Kinetic energy gets larger}, which means that the mass is getting smaller:
               $$
                \pi^0 \rightarrow \gamma\gamma
               $$
          \end{itemize}
     \section{Lecture 7}
     [Absent]
     \section{Lecture 8}
          \begin{itemize}
               \item \textbf{Symmetries:} Notern theorem states that a symmetry implies a conservation law. 
               \begin{itemize}
                    \item Symmetries are two types: continous and descrete, each can be inner symmetry or outer symmetry.
                    \item inner symmetry are the ones that happens via a change in the field or the wave function.
                    \item outer  symmetry changes the position or time.
                    \item for instance for :
                    \begin{equation}
                         \text{Descrete Symmetries}\Rightarrow\left\{
                              \begin{matrix}
                                   \text{inner: Changing the sign of electric charge}: \text{particle} \rightarrow \text{antiparticle}\\
                                   \text{outer: Changing the sign of position or time vectors}: \vec x \rightarrow -\vec x
                              \end{matrix}\right.
                    \end{equation}
                    for continous we again have another categorizing, each inner or outer continous symmetry, can be local or not local.
               \end{itemize}
               \begin{equation}
                    \text{Continous Symmetries}\Rightarrow\left\{
                         \begin{matrix}
                              \text{inner:}\Rightarrow \left\{
                                   \begin{matrix}
                                        \text{non-local: Lepton number is a } U(1) \text{ group}.\\
                                        \text{Local: stron interaction is a } SU(3)_c \text{ Group}
                                   \end{matrix}\right.\\
                              \text{outer: Lorentz Transformation is a non-local Symmetry}
                         \end{matrix}\right.
               \end{equation}
               \item \textbf{Group theory:}
               \begin{itemize}
                    \item A group is a set of elements and an operation called multiplication which has the following properties:
                    \item it is closed:
                    \begin{equation}
                         \text{if} a\in G \and b\in G \Rightarrow a\cdot b \in G
                    \end{equation}
                    \item it has an identity member:
                    \begin{equation}
                         I\in G \Rightarrow a\cdot I = I\cdot a = a
                    \end{equation}
                    \item each member has some other member which satisfies:
                    \begin{equation}
                         \forall a \exists b \Rightarrow a\cdot b = b\cdot a = I
                    \end{equation}
                    \item Associativity:
                    \begin{equation}
                        \forall a,b,c \in G\Rightarrow a\cdot(b\cdot c) = (a\cdot b)\cdot c
                    \end{equation}
               \end{itemize}
               \item \textbf{Some Examples of groups:}
               \begin{itemize}
                    \item $O(n)$: these are orthogonal matrices which beside the group properties has:
                    \begin{equation} 
                         o\in O(n) \Rightarrow \left\{ \begin{matrix} n\times n \\
                              OO^T =O^TO = I
                         \end{matrix}\right.
                    \end{equation} 
                    \item $SO(n)$ is the special orthogonal group, these not only have the properties of orthogonal groups but also have the determinant of $1$.
                    \item $U(n)$ is the unitary group. 
                    \begin{equation} 
                         u\in U(n) \Rightarrow \left\{ \begin{matrix} n\times n \\
                              uu^\dagger =u^\dagger u = I
                         \end{matrix}\right.
                    \end{equation} 
               \end{itemize}
               \item \textbf{Symmetry in Quantum mechanics:} For a quantum mechanical transformation/ operation that we want it to have some properties:
               \begin{itemize}
                    \item first of all we want it to not change the probability distribution of $\psi$ which implies that this operator must be unitary:
                    \begin{align*}
                         \ket{\psi} \rightarrow \ket{\psi'} &= U\ket{\psi}\\
                         \rightarrow\braket{\psi}{\psi} &= \braket{\psi'}{\psi'} =\braket{\psi U^\dagger}{U\psi}\\
                         \rightarrow UU^\dagger &= U^\dagger U = I
                    \end{align*}
                    It must commute with Hamiltonian or:
                    \begin{align*}
                         \braket{\psi}{H\vert\psi}&\rightarrow\braket{\psi'\vert H}{\psi'}\\
                         \rightarrow U^\dagger HU &= H\rightarrow HU = UH \rightarrow [H,U]=0
                    \end{align*}
               \end{itemize} 
               \item \textbf{Unitary generator:} we define an infinitesimal transformation of the unitary operator (the generator of it):
               \begin{equation}
                    U = 1 + i\alpha \hat G
               \end{equation}
               To check the properties we have:
               \begin{align*}
                    U^\dagger U = 1 = (1-i\alpha\hat{G}^\dagger)(1+i\alpha\hat G) &\Rightarrow \hat{G}^\dagger = \hat G
                    \\
                    [H, U] = 0 \Rightarrow &[H, 1+i\alpha \hat G]\rightarrow [H,\hat G] = 0
               \end{align*}
               \item \textbf{Positional transformation:} Now given the unitary properties with the hamiltonian we check the positional and momentum transformation.
               \begin{align*}
                    x \rightarrow x +\epsilon \ \ \ &| \ \ \  \frac{\epsilon}{x}<<1\\
                    \psi(x) \rightarrow \psi(x+\epsilon) &= \psi(x) + \epsilon\frac{\partial\psi}{\partial x} = (1+\epsilon\frac{\partial}{\partial x})\psi(x)\\
                    \text{and for momentum as well:}
                    (1+i\epsilon\hat p_x)\psi(x)
               \end{align*}
               And for three dimensional space we have:
               \begin{equation}
                    \vec x \rightarrow\vec x + \vec \epsilon
               \end{equation}
               if we don't want to use a infinitesimal $\alpha$ for the generator and we choose a finite one we can write:
               \begin{align*}
                    \alpha &\rightarrow \text{finite} \\
                    &\lim_{n\rightarrow \infty}\frac{\alpha}{n}\\
                    U_n &= 1+ i\frac{\vec \alpha}{n}\hat G\\
                    U &= \lim_{n\rightarrow\infty}(1+i\frac{\vec \alpha}{n}\hat G)^n \Rightarrow e^{i\vec \alpha \hat G}
               \end{align*}
               \item \textbf{Angular Momentum:} recalling $\hat L$ operator, which was the angular momentum we had:
               \begin{align}
                    \hat{\vec L} &= \hat{\vec r}\times\hatvec p\Rightarrow\left\{\begin{matrix}
                         \hat L_x \\
                         \hat L_y \\ 
                         \hat L_z 
                    \end{matrix}\right.\\
                    [\hat L_i, \hat L_j] &= i\epsilon_{ijk}\hat L_k\\
                    \hat L^2 &= \hat L_x^2 +\hat L_y^2 +\hat L_z^2\\
                    [\hat L^2 , \hat L_i] &= 0\\
                    \hat L^2 , \hat L_z &\rightarrow\text{Same eingenvalues: } \ \ \ket{\ell , m}\\
                    \hat L_z \ket{\ell, m} &= m\ket{\ell, m}\\
                    \hat L^2 \ket{\ell, m} &= \ell(\ell+1)\ket{\ell,m}\\
                    \hat{L_\pm} &= \hat L_x \pm i \hat L_y\\
                    \hat{L_\pm} \ket{\ell, m} &= \sqrt{\ell(\ell+1) - m (m\mp 1  )}\ket{\ell ,m\pm 1}
               \end{align}
               \item\textbf{Isospin Symmetry:} Heisenberg thought about the symmetricallity between proton and neutron and tried to explain the nucleus as:
               \begin{equation}
                    \hat H = \hat H_{\text{strong}} + \hat H_{\text{EM}} + \hat H_{\text{mass/kinetic}}
               \end{equation}
               If we were able to eglect the electromagnetic interaction and the mass difference between the two we have:
               \begin{align}
                    p = \begin{matrix}1\\0\end{matrix}, n = \begin{matrix}0\\1\end{matrix}\Rightarrow N = \begin{matrix}p\\n\end{matrix}
               \end{align}
               This is a approximated symmetry, continous and universal (not local). The same thing can be said for the $u$ and $d$ quarks, defining:
               \begin{align}
                    u = \begin{matrix}1\\0\end{matrix}, d = \begin{matrix}0\\1\end{matrix}
               \end{align}
               This is the same as the isospin (although we might call it flavour symmetry) since the difference between neutron and proton is exactly the difference of a$u$ and $d$ quark.
          \end{itemize}

     \section{Lecture 9}
     \textbf{Flavour Symmetry (iso spin): } Heisenberg considered proton and neutron the same if we neglect their mass and their electric charge. This can also be interpreted as a indenticality of up and down quarks. 
     \\
     \\
     Heisenberg considered them the same and thus concluding that there's only one type of particle in the nucleus (called nucleon).  with two states up and down. Thus to show a necleon we have the state:
     $$
     N = \begin{matrix} u \\ d\end{matrix}
     $$
     Now we want to find the transfomation that turns a state into another, The transformation must be unitary so that it can be a symmetry transformation:
     \begin{equation}
          \begin{matrix} u' \\ d' \end{matrix} = \begin{matrix} u_{11} & u_{12} \\ u_{21} & u_{22} \end{matrix}
     \end{equation}
     The U matrix has 8 parameters but 4 constrains would lead to only 4 independent parameters, the determinant of a unitary matrix must be:
     \begin{equation}
          \det U = e^{i\phi}
     \end{equation}
     but there are matrices in the U(2) group that does not change anything namely:
     $$
     \begin{matrix}
          1 & 0 \\
          0 & 1
     \end{matrix}\cdot e^{i\phi}
     $$
     What we like to have is actually SU(2) groups which have determinant of 1.
     \begin{equation}
          \left\{ \begin{matrix} \det =1\\
               UU^\dagger = I
          \end{matrix}\right.
     \end{equation}

     The second one implies that the generators must be hermitian ($ G = G^\dagger$). and the first one implies as the following:
     \begin{align*}
          \hat U &= 1 + i\vec \epsilon \cdot\vec  G 
          \\
          &= 1+i\epsilon \begin{matrix}
               g_{11} & g_{12} \\
               g_{21} & g_{22}
          \end{matrix}
          \det \hat U & = (1+ i\epsilon g_{11})(1+ i\epsilon g_{22}) + \epsilon^2 g_{21}g_{12}\\
          &\sim 1 + i\epsilon(g_{11} + g_{22}) =1
          &\therefore g_{11} = - g_{22}
     \end{align*}
     This means that the generator must be traceless.
     \\ Thus the general form of the transformation is:
     \begin{equation}
          \hat U = 1 + i \left\{ \frac{a'}{2} \begin{matrix} 1 & 0 \\ 0 & -1\end{matrix} + \frac{b'}{2} \begin{matrix} 0 & 1 \\ 1 & 0\end{matrix} + \frac{c'}{2}\begin{matrix}0 & -i \\ i & 0 \end{matrix}
          \right\}
     \end{equation}
     The matrices are called pauli matrices and have the relation:
     \begin{align}
          \left[ \frac{\sigma_i}{2} , \frac{\sigma_j}{2} \right] &=  i\epsilon_{ijk} \frac{\sigma_k}{2}\\
          T^2 &\equiv \sum_i T_i^2\\
          \rightarrow \left[ T^2 , T_i\right] &= 0\text{\small They have same eigenvlues}          
     \end{align}
     we also defined: $ T_i \equiv \frac{\sigma_i}{2}$ Thus the state is to be written:
     \begin{equation}
          \phi = \ket{I,I_3}
     \end{equation}
     With the properties:
     \begin{align}
          T^2 \ket{I, I_3} &= I(I+1) \ket{I,I_3}\\
          T_3 \ket{I, I_3} &= I_3 \ket{I, I_3} \\
          T_{\pm} \ket{I,I_3} &= (T_1 \pm T_2)\ket{I,I_3} = \sqrt{I(I+1) - I_3(I_3 \pm 1)} \ket{I,I_3}
     \end{align}
     In terms of eigen states $u$ and $d$ we have:
     \begin{align}
          T_- d &= 0 \\ 
          T_+ d &= u\\
          T_- u &= d \\
          T_+ u &= 0
     \end{align}
     The use of defining isospins is that they can categorize all hadrons.
     \\
     If we have two quarks, the states of the system is:
     \begin{equation}
          \ket{I^{(1)}, I_3^{(1)}}\ket{I^{(2)}, I_3^{(2)}}
     \end{equation}
     The isospin of the whole system $I, I_3$ can be calculated.
     \begin{equation} 
          |I^{(1)} - I^{(2)}| \leq I \geq |I^{(1)} + I^{(2)}|
     \end{equation}
     Thus we have four states a triplet:
     \begin{equation}
          I = 1 \left\{\begin{matrix}\ket{1,1}\\ \ket{1,0} \\ \ket{1,-1}\end{matrix} \right.
     \end{equation}
     and a singlet
     \begin{equation}
          I = 0 \left\{\ket{0,0}\right.
     \end{equation}
     In terms of the actual tensor product between states $\begin{matrix} u_1 \\ d_1 \end{matrix}$ and $\begin{matrix} u_2 \\ d_2 \end{matrix}$.
     \begin{callout}
          the states are:
          \begin{align}
          \ket{1,1} &=  uu \\
          \ket{1,-1}&= dd\\
          \ket{1,0} &= \frac{1}{\sqrt 2}(ud+du)
          \ket{0,0} &= \frac{1}{\sqrt 2}{ud-du}
          \end{align}

     \end{callout}
     \textbf{Three Quarks:} For three quarks we have:
     \begin{equation}
          (\ket{u_1,d_1} \otimes \ket{u_2,u_2})\otimes \ket{u_3,d_3}
     \end{equation}
     Here The total isospin $I$ can have these states:
     \begin{align}
          I &= \frac32 \left\{ \begin{matrix}
               \ket{\frac32, \frac32} = \ket{uuu}\\
               \ \\
               \ket{\frac32, \frac12} = \frac{1}{\sqrt 3}(\ket{duu} + \ket{udu} + \ket{uud})\\ \ \\
               \ket{\frac32,-\frac12} = \frac{1}{\sqrt3}(\ket{udd} + \ket{dud} + \ket{ddu})\\ \ \\
               \ket{\frac32, -\frac32} = \ket{ddd}
          \end{matrix}\right.
          \\
           \ \\ 
          I &= \frac12\left\{ \begin{matrix}
               \ket{\frac12,\frac12}_S = \frac{1}{\sqrt 6}(2\ket{ddu} + -\ket{udd} - \ket{dud}) \\ \ \\
               \ket{\frac12,-\frac12}_S = \frac{1}{\sqrt 6}(2\ket{uud} - \ket{udu} - \ket{duu})
          \end{matrix}\right.\\ \ \\
          I &= \frac12\left\{ \begin{matrix}
               \ket{\frac12,\frac12}_A = \frac{1}{\sqrt 2}(\ket{udd} - \ket{dud}) \\ \ \\
               \ket{\frac12,-\frac12}_A = \frac{1}{\sqrt 6}(\ket{udu} - \ket{duu})
          \end{matrix}\right.
     \end{align}


\end{document}