\documentclass[10pt,a4paper]{article}
\usepackage[utf8]{inputenc}
\usepackage{amsmath}
\usepackage{amsfonts}
\usepackage{amssymb}
\usepackage{url}
\usepackage{makeidx}
\usepackage{graphicx}
\usepackage{graphicx, adjustbox}
\usepackage{lmodern}
\usepackage{fourier}
\usepackage{float}
\usepackage{caption}
\usepackage{wrapfig}
\usepackage{mhchem}
\usepackage[left=2.5cm,right=2.5cm,top=1cm,bottom=2cm]{geometry}
\usepackage[dvipsnames]{xcolor}
\usepackage{multicol}
\usepackage{soul}
\usepackage{tikz-feynman} 
%Colors

\definecolor{black}{RGB}{0, 0, 0}
\definecolor{richblack}{RGB}{7, 14, 13}
\definecolor{charcoal}{RGB}{45, 67, 77}
\definecolor{delectricblue}{RGB}{93, 117, 131}
\definecolor{cultured}{RGB}{245, 245, 245}
\definecolor{lightgray}{RGB}{211, 216, 218}
\definecolor{silversand}{RGB}{190, 194, 198}
\definecolor{spanishgray}{RGB}{148, 150, 157}
\definecolor{darkliver}{RGB}{64, 63, 76}

\colorlet{lightdelectricblue}{delectricblue!30}
\colorlet{lightdarkliver}{darkliver!30}


%ColorDefines
\newcommand{\trueblack}[1]{\textcolor{black}{#1}}
\newcommand{\rich}[1]{\textcolor{richblack}{#1}}
\newcommand{\lightblack}[1]{\textcolor{charcoal}{#1}}
\newcommand{\lightrich}[1]{\textcolor{delectricblue}{#1}}
\newcommand{\liver}[1]{\textcolor{darkliver}{#1}}

%Boxes
\usepackage{tcolorbox}
\newtcolorbox{calloutbox}{center,%
    colframe =red!0,%
    colback=cultured,
    title={Callout},
    coltitle=richblack,
    attach title to upper={\ ---\ },
    sharpish corners,
    enlarge by=0.5pt}

\newtcolorbox[use counter=equation]{eq}{center,
	colframe =red!0,
	colback=cultured,
	title={\thetcbcounter},
	coltitle=richblack,
	detach title,
	after upper={\par\hfill\tcbtitle},
	sharpish corners,
    enlarge by=0.5pt }
    
\newtcolorbox{qt}{center,
	colframe=delectricblue,
	colback=white!0,
	title={\large "},
	coltitle=delectricblue,
	attach title to upper,
	after upper ={\large "},
	sharp corners,
	enlarge by=0.5pt,
	boxrule=0pt,
	leftrule=2pt}
	
\newtcolorbox{ecx}{center,%
    colframe =red!0,%
    colback=darkliver!15,
    title={Exercise},
    coltitle=richblack,
    attach title to upper={\ ---\ },
    sharpish corners,
    enlarge by=0.5pt}
    
\newcounter{theo}
\newtcolorbox[use counter=theo]{theobox}
	{center,%
    colframe =red!0,%
    colback=cultured,
    title={Theorem \thetcbcounter},
    coltitle=richblack,
    attach title to upper={\ ---\ },
    sharpish corners,
    enlarge by=0.5pt}

\newcounter{examplecounter}
\newtcolorbox[use counter=examplecounter]{example}
	{center,%
    colframe =red!0,%
    colback=cultured,
    title={Example \thetcbcounter},
    coltitle=richblack,
    attach title to upper={\ ---\ },
    sharpish corners,
    enlarge by=0.5pt}

    

        
    
% Highlighters
\newcommand{\hldl}[1]{%
	\sethlcolor{lightdarkliver}%
	\hl{#1}
}
\newcommand{\hldb}[1]{%
    \sethlcolor{lightdelectricblue}%
    \hl{#1}%
}


% Images
\newcounter{figurecounter}
\setcounter{figurecounter}{1}

\newcommand{\img}[3]{
    \begin{figure}[h!]
        \centering
        \captionsetup{justification=centering,margin=0cm,labelformat=empty}
        \includegraphics[width=#2\linewidth]{./img/#1}
        \label{figure}
        \caption{\small\textbf{fig: \thefigurecounter} -- \textcolor{darkliver}{#3}}
    \end{figure}
    \addtocounter{figurecounter}{1}}

\newcommand{\imgr}[3]{
    \begin{wrapfigure}{r}{#2\textwidth}
        \centering
        \captionsetup{justification=centering,margin=0cm,labelformat=empty}
        \includegraphics[width=\linewidth]{./img/#1}
        \label{figure}
        \caption{\small \textbf{fig: \thefigurecounter} -- \textcolor{darkliver}{#3}}
    \end{wrapfigure}
    \addtocounter{figurecounter}{1}}

\newcommand{\imgl}[3]{
    \begin{wrapfigure}{l}{#2\textwidth}
        \centering
        \captionsetup{justification=centering,margin=0cm,labelformat=empty}
        \includegraphics[width=\linewidth]{./img/#1}
        \label{figure}
        \caption{\small \textbf{fig: \thefigurecounter} -- \textcolor{darkliver}{#3}}
    \end{wrapfigure}
    \addtocounter{figurecounter}{1}}

% New commands
\newenvironment{callout}
	{\begin{calloutbox}\color{charcoal}\textbf\textit}
	{\end{calloutbox}}

\newcommand{\mev}{\text{MeV}}
\newcommand{\gev}{\text{GeV}}
\newcommand{\fpe}{4\pi\epsilon_0}
\newcommand{\ch}[5]{{}^{#2}_{#3}\!\text{#1}^{#4}_{#5}}
\newcommand{\electron}{\ch{e}{}{}{-}{}}
\newcommand{\positron}{\ch{e}{}{}{+}{}}
\newcommand{\proton}{\ch{p}{}{}{}{}}
\newcommand{\muon}{\ch{\mu}{}{}{-}{}}
\newcommand{\neutron}{\ch{n}{}{}{}{}}
\newcommand{\neutrino}[1]{\ch{\nu}{}{}{}{#1}}
\newcommand{\braket}[2]{\left\langle #1 \vert #2 \right\rangle}
\newcommand{\mbraket}[3]{\left\langle #1 \vert #2 \vert #3 \right\rangle}
\newcommand{\ket}[1]{\left\vert #1 \right\rangle}
\newcommand{\bra}[1]{\left\langle #1 \right\vert} 
\newcommand{\hamiltonian}{\mathcal{H}}

\title{Particle Physics}

\begin{document}
          \maketitle
          \tableofcontents

          \section{Lecture 1}
          \subsection{Interactions}
          \textbf{Electromagnetic Force:}
          \begin{itemize}
               \item  The mediator is photon with $m_\gamma =0$ 
               \item the interaction is shown in the feynman diagram as a point which is called vertex which has a magnitude, showing the magnitude of our interaction.
               \item any fermions (not considering neutrinos) can have electromagnetic interaction. the coupling constant (the magnitude of interaction) for charged leptons (electron muon and tau) is $-e$, and for quarks are $\frac23 e$ or $-\frac13 e$.
               \item electromagnetic interaction wouldn't chage the flavour, it doesn't have flavour changing. If an electron interacts with a photon, the electron wouldn't change flavour (it won't become a muon!)
               \item the theory that any force needs a mediator comes from Quantum Fiels Theory.
               \item The quantum field theory of electrodynamics is QED (Quantum Electrodnamics.)
          \end{itemize}
          \textbf{Strong Interaction:}
          \begin{itemize}
               \item the mediator is gluon (there's actually eight gluons) and their mass is zero, $m_g=0$ 
               \item The theory is called Quantum Chromodynamics (QCD).
               \item only quarks can have strong interaction. Coupling of strong interaction $g_s$ need colour charge.
               \item interacting with glouns, just like photons, won't change flavour (the interaction of an up quark with gluon won't change it to a down quark, or any quark for that matter).
          \end{itemize}
          \textbf{Weak Interaction}
          \begin{itemize}
               \item The first type mediator is $W^{\pm}$ which also has a mass, $m_W = 80.1 \gev$.
               \item the other type is $Z^0$ which has $m_Z = 90.1 \gev$.
               \item People have always tried to make a grand theory, first maxwell showed electric and magnetic forces are the same and then people showed that the electromagnetic and weak interactions can be thought as one calling the theory electroweak interaction.
               \item $Z^0$ interacts with every fermions (including neutrinos!), the coupling $g_Z$. and we don't have any flavour changing.
               \item $W^\pm$ is different, it should change the flavour because of charge conservation. an electron interacting with $W^-$ will become a $\nu_e$.
               \begin{align}
                    \begin{pmatrix}
                         \nu_e \\ e
                    \end{pmatrix} ,
                    \begin{pmatrix}
                         \nu_\mu \\ \mu
                    \end{pmatrix} ,
                    \begin{pmatrix}
                         \nu_\tau \\ \tau
                    \end{pmatrix} 
               \end{align}
               and for quarks it's more complicated:
               
               \begin{align*}
                    \begin{pmatrix}
                         u \\ d
                    \end{pmatrix} ,
                    \begin{pmatrix}
                         u \\ s
                    \end{pmatrix} ,
                    \begin{pmatrix}
                         c \\ b
                    \end{pmatrix} ,
                    \begin{pmatrix}
                         c \\ d
                    \end{pmatrix} ,
                    \begin{pmatrix}
                         c \\ s
                    \end{pmatrix} ,
                    \begin{pmatrix}
                         c \\ b
                    \end{pmatrix} ,
                    \begin{pmatrix}
                         t \\ d
                    \end{pmatrix} ,
                    \begin{pmatrix}
                         t \\ s
                    \end{pmatrix} ,
                    \begin{pmatrix}
                         t \\ b
                    \end{pmatrix} ,
               \end{align*}
          \end{itemize}
          \begin{callout}
               The things we have siad are not the interactions, their are allowed vertexes in standard model.
          \end{callout}
          \begin{callout}
               the charge of every mediator but $W^\pm$ is zero. This means that we have no flavour changing neutral current (FCNC), but rahter Flavour Changing Charged Current (FCCC).
          \end{callout}
          \subsection{A general form of reactions}
          A general form of any reaction is:
          \begin{equation}
               a + b \rightarrow c+d
          \end{equation}
          for example a simple reaction:
          \begin{equation}
               \electron +\electron\rightarrow\electron +\electron
          \end{equation}
          or another example:
          \begin{equation}
               \electron +\nu_e \rightarrow\electron+\nu_e
          \end{equation}
          \begin{callout}
               These are Fermions interactions, the Bosons can interact (other than photon) with each other.
          \end{callout}
     \subsection{Hadrons}
     \begin{itemize}
          \item Hadrons are not elementary particles and are made of quarks:
          \item There are two groups of hadrons, Mesons and Baryons
          \item Baryons are made of three quarks or three anti quarks, proton ($\ket{uud}$) and neutron ($\ket{ddu}$) are examples.
          \item Mesons are made of a pair of quark anti quarks, $\Pi^+$ ($\ket{u\bar d}$) is an example.
          \item \textbf{Proton:} it's a stable particle with mass $m_P = 938.27\mev$, found in 1919, the life-time of a proton is: $2.1 \time 10^{29} \text{years}$!. with spin $\frac12$.
          \item \textbf{Neutron:} is not stable and has the mass $m_N = 939.56\mev$ ($m_P-m_N = 1.29\mev$) has the lifetime $15\text{min}$ (as a free particle).with spin $\frac12$.
          \item Baryons doesn't have only three quarks, the three quarks that make the attributes of the brayons are called valance quarks, but there are many other quarks in a baryon, generally called the sea quarks.
     \end{itemize}
     \begin{callout}
          How did they found experimentally the lifetime of proton? Solve
     \end{callout}
     \section{Lecture 2}
          \subsection{SI units vs Natural units}
          \begin{itemize}
               \item In SI units we use kg for mass meter for length and second for time. This is not the case for standard model and elementary particle physics. Because SM comes from field theory and there we have two constants $c =2.998\times 10^8 \frac ms$ and $ \hbar \frac{h}{2\pi} = 2.055\times 10^{-34}Js$.
               \item the Natural units are as below:
               \begin{itemize}
                    \item \textbf{Energy: } in SI units is Joule $J=kgm^2s^{-2}$, and in natural units we use electron volt, $eV = 1.602\times 10^{-19}\times 1 = 1.602\times 10^{-19}$ which means that, $1\gev \approx 10^{-10}J$.
                    \item \textbf{Momentum: } in SI units is $kgms^{-1}$, and in natural units we use $E/c$ or in terms of unit $\gev/c$. where $c$ is the speed of light.
                    \item \textbf{Mass: } in SI units is kilogram, and in natural units we use $\gev/c^2$. 
                    \item \textbf{Time: } in SI it's seconds, and in natural units it is $(\gev/\hbar)^{-1}$.
                    \item \textbf{Length: } in SI it's meterm and in natural units it's $(\gev/\hbar c)^{-1}$.
                    \item \textbf{Assumption:} we can assume that $c = \hbar = 1$ then all the units become either $\gev$ or $\gev^{-1}$
                    \item \textbf{Returning:} The following calculations are to get back to SI units from the Natural units:
                    \begin{align*}
                         \frac{\hbar c}{\gev} = 0.197 fm\\
                         \frac{\hbar}{\gev} = 0.658\times 10^{-24}s\\
                         \frac{\gev}{c^2} = 0.178\time10^{-26}kg
                    \end{align*}

                    \item \textbf{Cross section: } the dimension is $[L^2]$, in SI $m^2$, and in natural units it can be$\gev^{-2}$, but since in Particle physics, we define barn $b = 10^{-28}m^2$ since it's $2576.72\gev^{-2}$ we would probably use $fb, pb$ more.
                    \item \textbf{heaviside-lorentz: } because $c =1 =\frac{1}{\epsilon_0\mu_0}$ we assume $\epsilon_0,\mu_0$ as one too.
                    \item heaviside-lorentz $\alpha = \frac{e^2}{4\pi\epsilon_0\hbar c} \rightarrow \alpha = \frac{e^2}{4\pi} \rightarrow e^2 = 4\pi\alpha$.
               \end{itemize}
          \end{itemize}
          \subsection{Interactions, Order, Coupling and Energy}
          \begin{itemize}
               \item \textbf{Scattering: } in scattering we generally have: $a+b\rightarrow c+d$[Write the schematic of collision and the feynman diagram Amir.], The probability of this happening is given through the cross section.
               \item \textbf{Decay: } decay is generally $a\rightarrow b+c$[Write the schematic of collision and the feynman diagram Amir.], The probability is given through the width of decay [pahnaye vapashi.]
               \item \textbf{Orders: } loops and the convergence of lower orders of diagrams.[research about them. Leading order diagrams, Next leading order diagram.] The reason of the convergence is becase the probability that is calculated through $M^N = \alpha^{2N}$ with $N$ being the number of the leading order, converges since $\alpha =\frac{1}{137}$ but $\alpha$ depends on the energy.
               \item \textbf{Coupling and Energy}: calculations in field theory shows that the convergence of next order diagrmas is not the case in Quantum Chromodynamics (QCD), since as energy increases $\alpha$ increases aswell and from a point it is larger than one meaning that the formula in the last point does not converge but rather diverge. The point were $\alpha_{QED}$ becomes one is so far (on energy spectrum) that we have not yet even reached in experiments so everything works well. But for $\alpha{QCD}$ it's reversed. it is larger in lower energies nad smaller in higher energies it is fine. Thus in this course we only work in the areas that we are able to work using SM and perturbation theory.
          \end{itemize}
          \begin{ecx}
               find the newton constant for gravity, plank mass, protons mass and protons radius in terms of natural units.
          \end{ecx}
     \section{Lecture 3}
          \subsection{QED}
          \begin{itemize}
               \item \textbf{Lepton Number:} Any lepton (electron, tau, muon and their neutrinos) has the lepton number $L = +1$, and their anti-particles $L = - 1$. 
               \item \textbf{Conservation of Lepton Number:} considering the reaction below:

          
          \begin{equation}
          \mu^{-} \rightarrow e^{-} \gamma
          \end{equation}
          This reaction is not possible, though the electric charge and the Lepton number is conserved, we have to look for electron number and muon number conservation aswell. Thus it's not possible. A possible decay for muon would be:

          
          \begin{equation}
               \mu^{-1} \rightarrow e^{-1} \bar{\nu_{e}}  \nu_{\mu}
          \end{equation}

          \item \textbf{Baryon Number:} for each quark we have baryon number $\frac13$ and thus Baryons have $B=1$, like proton and neutron. anti Baryons which have anti quark $B=-1$ and thus each anti quark has $-\frac13$. As an example the reaction below is possible.

          \begin{equation}
               n \rightarrow\proton e \bar{\nu_{e}} 
          \end{equation}

          \item another possible interaction is:
          \begin{equation}
               e^{-}e^{+} \rightarrow \gamma \gamma
          \end{equation}
          

          \end{itemize}
          \begin{callout}
               anti-particles arrow is opposite to it's momentum direction
          \end{callout}
\end{document}