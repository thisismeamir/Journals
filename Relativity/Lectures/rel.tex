\documentclass[10pt,a4paper,twocolumn]{article}
\usepackage[utf8]{inputenc}
\usepackage{amsmath}
\usepackage{amsfonts}
\usepackage{amssymb}
\usepackage{url}
\usepackage{makeidx}
\usepackage{graphicx}
\usepackage{graphicx, adjustbox}
\usepackage{lmodern}
\usepackage{fourier}
\usepackage{float}
\usepackage{caption}
\usepackage{wrapfig}
\usepackage{mhchem}
\usepackage[left=2.5cm,right=2.5cm,top=1cm,bottom=2cm]{geometry}
\usepackage{multicol}
\usepackage{soul}
%Colors
\usepackage[dvipsnames]{xcolor}


\definecolor{black}{RGB}{0, 0, 0}
\definecolor{richblack}{RGB}{7, 14, 13}
\definecolor{charcoal}{RGB}{45, 67, 77}
\definecolor{delectricblue}{RGB}{93, 117, 131}
\definecolor{cultured}{RGB}{245, 245, 245}
\definecolor{lightgray}{RGB}{211, 216, 218}
\definecolor{silversand}{RGB}{190, 194, 198}
\definecolor{spanishgray}{RGB}{148, 150, 157}
\definecolor{darkliver}{RGB}{64, 63, 76}

\colorlet{lightdelectricblue}{delectricblue!30}
\colorlet{lightdarkliver}{darkliver!30}


%ColorDefines
\newcommand{\trueblack}[1]{\textcolor{black}{#1}}
\newcommand{\rich}[1]{\textcolor{richblack}{#1}}
\newcommand{\lightblack}[1]{\textcolor{charcoal}{#1}}
\newcommand{\lightrich}[1]{\textcolor{delectricblue}{#1}}
\newcommand{\liver}[1]{\textcolor{darkliver}{#1}}

%Boxes
\usepackage{tcolorbox}
\newtcolorbox{calloutbox}{center,%
    colframe =red!0,%
    colback=cultured,
    title={Callout},
    coltitle=richblack,
    attach title to upper={\ ---\ },
    sharpish corners,
    enlarge by=0.5pt}

\newtcolorbox[use counter=equation]{eq}{center,
	colframe =red!0,
	colback=cultured,
	title={\thetcbcounter},
	coltitle=richblack,
	detach title,
	after upper={\par\hfill\tcbtitle},
	sharpish corners,
    enlarge by=0.5pt }
    
\newtcolorbox{qt}{center,
	colframe=delectricblue,
	colback=white!0,
	title={\large "},
	coltitle=delectricblue,
	attach title to upper,
	after upper ={\large "},
	sharp corners,
	enlarge by=0.5pt,
	boxrule=0pt,
	leftrule=2pt}
	
\newtcolorbox{exc}{center,%
    colframe =red!0,%
    colback=darkliver!15,
    title={Excercise},
    coltitle=richblack,
    attach title to upper={\ ---\ },
    sharpish corners,
    enlarge by=0.5pt}
    
\newcounter{theo}
\newtcolorbox[use counter=theo]{theobox}
	{center,%
    colframe =red!0,%
    colback=cultured,
    title={Theorem \thetcbcounter},
    coltitle=richblack,
    attach title to upper={\ ---\ },
    sharpish corners,
    enlarge by=0.5pt}

\newcounter{examplecounter}
\newtcolorbox[use counter=examplecounter]{example}
	{center,%
    colframe =red!0,%
    colback=cultured,
    title={Example},
    coltitle=richblack,
    attach title to upper={\ ---\ },
    sharpish corners,
    enlarge by=0.5pt}

    

        
    
% Highlighters
\newcommand{\hldl}[1]{%
	\sethlcolor{lightdarkliver}%
	\hl{#1}
}
\newcommand{\hldb}[1]{%
    \sethlcolor{lightdelectricblue}%
    \hl{#1}%
}


% Images
\newcounter{figurecounter}
\setcounter{figurecounter}{1}

\newcommand{\img}[3]{
    \begin{figure}[h!]
        \centering
        \captionsetup{justification=centering,margin=0cm,labelformat=empty}
        \includegraphics[width=#2\linewidth]{./img/#1}
        \label{figure}
        \caption{\small\textbf{fig: \thefigurecounter} -- \textcolor{darkliver}{#3}}
    \end{figure}
    \addtocounter{figurecounter}{1}}

\newcommand{\imgr}[3]{
    \begin{wrapfigure}{r}{#2\textwidth}
        \centering
        \captionsetup{justification=centering,margin=0cm,labelformat=empty}
        \includegraphics[width=\linewidth]{./img/#1}
        \label{figure}
        \caption{\small \textbf{fig: \thefigurecounter} -- \textcolor{darkliver}{#3}}
    \end{wrapfigure}
    \addtocounter{figurecounter}{1}}

\newcommand{\imgl}[3]{
    \begin{wrapfigure}{l}{#2\textwidth}
        \centering
        \captionsetup{justification=centering,margin=0cm,labelformat=empty}
        \includegraphics[width=\linewidth]{./img/#1}
        \label{figure}
        \caption{\small \textbf{fig: \thefigurecounter} -- \textcolor{darkliver}{#3}}
    \end{wrapfigure}
    \addtocounter{figurecounter}{1}}

% New commands
\newenvironment{callout}
	{\begin{calloutbox}\color{charcoal}\textbf\textit}
	{\end{calloutbox}}

\newcommand{\mev}{\text{MeV}}
\newcommand{\gev}{\text{GeV}}
\newcommand{\fpe}{4\pi\epsilon_0}
\newcommand{\ch}[5]{{}^{#2}_{#3}\!\text{#1}^{#4}_{#5}}
\newcommand{\electron}{\ch{e}{}{}{-}{}}
\newcommand{\positron}{\ch{e}{}{}{+}{}}
\newcommand{\proton}{\ch{p}{}{}{}{}}
\newcommand{\neutron}{\ch{n}{}{}{}{}}
\newcommand{\neutrino}[1]{\ch{\nu}{}{}{}{#1}}
\newcommand{\braket}[2]{\left\langle #1 \vert #2 \right\rangle}
\newcommand{\mbraket}[3]{\left\langle #1 \vert #2 \vert #3 \right\rangle}
\newcommand{\ket}[1]{\left\vert #1 \right\rangle}
\newcommand{\bra}[1]{\left\langle #1 \right\vert} 
\newcommand{\hamiltonian}{\mathcal{H}}

\title{Relativity \\ Journal}

\begin{document}
          \maketitle
          \tableofcontents
          \section{Lecture One}
          \textbf{Newtonian Relativity}
          \begin{itemize}
               \item World time,  same rate(ricking) different unit, different zeros.
               \item Absolute space
               \item Galilean  Transformation: if we want to transoform between two different coordinates, one moving with the speed $v_x$
               \begin{equation}
                    \left\{\begin{matrix}
                         (t,x,y,z)\\
                         (t',x',y',z')
                    \end{matrix}
                         \right.
               \end{equation}
               and the relation would be:
               \begin{equation}
                         x' = x- v_xt
               \end{equation}
               with $z'$ and $y'$ being equal to $z$ and $y$ respectively. ftom the equation we can see that the acceleration is seen equally in both systems. 
               \item \textbf{F= ma} would be invariant in both systems (under galilean transf.), meaning both force and mass is invariant(acceleration was invariant by eq1).
               \item axiom of mass in newtons prespective:type 1 the thing that appears in $f=ma$ against the movement, innersi. It's is completely apart from the gravitation $m_I$. type 2 the thing that is described by the gravitation force on an object  $m_{gr}$.
               \textbf{mass:}  active mass: creator of the gravitational field. (field producer)/  passive mass: acted on with the gravitational field. (test particle).
               \item for example a pendulum with innertial mass $m_I$ would be governed by $F =-kx =m\frac{d^2 x}{dt^2}$ having:
               \begin{callout}
                    Weak Equivalence Principle
               \end{callout}
               \begin{align*}
                    F &=m_I a =-kx\\
                    &= m_I \frac{d^2x}{dt^2} +kx =0\\
                    &=\frac{d^2x}{dt^2} +\frac{k}{m_I}x =0\\
                    &=\frac{d^2x}{dt^2}+\omega^2 x = 0\\
                    F &= -m_g g\sin\theta\\
                    &=m_g g(\theta \ell)/\ell\\
                    &=(-m_g g/\ell)x \Rightarrow k=m_gg/\ell
               \end{align*}

               \item Mach on Absolute space: why ballging happens or a bucket of water rotating changes shape (the water), newton says it is because water understands it's rotating with respect to the absolute space. But mach says it's because it's relation with other things in the world. [to be read about mach opinions.]
               \end{itemize}
               \begin{exc}
                    Is Maxwell's equations invariant with respect to galilean transformations?
               \end{exc}
    \section{Lecture Two}
        \textbf{Special relativity}
        \begin{itemize}
            \item There are two principles in special relativity.
            \begin{itemize}
                \item RP $\equiv$ relativity principle, which means that all inertial frames are equal. Physical laws won't change
                \item Constant and an invariant velocity: there is a maximum speed in nature that information can surpass. This speed is the speed of light.
            \end{itemize}  
            \item these principles would have results such as: 
            \begin{itemize}
                \item length contraction
                \item Time dilation
                \item Relativity simultaneity
            \end{itemize}
        \end{itemize}
        \begin{callout}
            We are having a flat space time in SR. and the Physics would have to be the same. This means that we are mostly experimenting in the domain of special relativity, and since we are going beyond the newtonian relativity. We are expecting any theory that we make to be lorentz invariant (works with special relativity).
        \end{callout}
        \begin{itemize}
            \item TThe fact that we have equivalence between inertial frames, means that we can calculate our laws of physics in any of them! This helps in simplifying different phenomenas.
            \item \textbf{An example:} A man on a belt moving up on an incline would calculate the work $(mg)v\sin\theta$ but a stationary person would calculate $(mg\sin\theta)v$ which is the same.
            \item Now assume a train moving with velocity $v$. A man at the middle of the wagon turns up a light. From his point of view the lighht reaches both ends at the same time,, say $3$ o'clock then an observer outside the train would see that the light reaches at different instances.[Thought Experiment by Einstein.]
            \item \textbf{K-Calculus}
        \end{itemize}

\end{document}