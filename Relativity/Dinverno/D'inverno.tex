\documentclass[10pt,a4paper,twocolumn]{article}
\usepackage[utf8]{inputenc}
\usepackage{amsmath}
\usepackage{amsfonts}
\usepackage{amssymb}
\usepackage{url}
\usepackage{makeidx}
\usepackage{graphicx}
\usepackage{graphicx, adjustbox}
\usepackage{lmodern}
\usepackage{fourier}
\usepackage{float}
\usepackage{caption}
\usepackage{wrapfig}
\usepackage{mhchem}
\usepackage[left=2.5cm,right=2.5cm,top=2cm,bottom=3cm]{geometry}
\usepackage{multicol}
\usepackage{soul}



%Colors
\usepackage[dvipsnames]{xcolor}


\definecolor{black}{RGB}{0, 0, 0}
\definecolor{richblack}{RGB}{7, 14, 13}
\definecolor{charcoal}{RGB}{45, 67, 77}
\definecolor{delectricblue}{RGB}{93, 117, 131}
\definecolor{cultured}{RGB}{245, 245, 245}
\definecolor{lightgray}{RGB}{211, 216, 218}
\definecolor{silversand}{RGB}{190, 194, 198}
\definecolor{spanishgray}{RGB}{148, 150, 157}
\definecolor{darkliver}{RGB}{64, 63, 76}

\colorlet{lightdelectricblue}{delectricblue!30}
\colorlet{lightdarkliver}{darkliver!30}


%ColorDefines
\newcommand{\trueblack}[1]{\textcolor{black}{#1}}
\newcommand{\rich}[1]{\textcolor{richblack}{#1}}
\newcommand{\lightblack}[1]{\textcolor{charcoal}{#1}}
\newcommand{\lightrich}[1]{\textcolor{delectricblue}{#1}}
\newcommand{\liver}[1]{\textcolor{darkliver}{#1}}

%Boxes
\usepackage{tcolorbox}
\newtcolorbox{calloutbox}{center,%
    colframe =red!0,%
    colback=cultured,
    title={Callout},
    coltitle=richblack,
    attach title to upper={\ ---\ },
    sharpish corners,
    enlarge by=0.5pt}

\newtcolorbox[use counter=equation]{eq}{center,
	colframe =red!0,
	colback=cultured,
	title={\thetcbcounter},
	coltitle=richblack,
	detach title,
	after upper={\par\hfill\tcbtitle},
	sharpish corners,
    enlarge by=0.5pt }
    
\newtcolorbox{qt}{center,
	colframe=delectricblue,
	colback=white!0,
	title={\large "},
	coltitle=delectricblue,
	attach title to upper,
	after upper ={\large "},
	sharp corners,
	enlarge by=0.5pt,
	boxrule=0pt,
	leftrule=2pt}
	
\newtcolorbox{exc}{center,%
    colframe =red!0,%
    colback=darkliver!15,
    title={Excercise},
    coltitle=richblack,
    attach title to upper={\ ---\ },
    sharpish corners,
    enlarge by=0.5pt}
    
\newcounter{theo}
\newtcolorbox[use counter=theo]{theobox}
	{center,%
    colframe =red!0,%
    colback=cultured,
    title={Theorem \thetcbcounter},
    coltitle=richblack,
    attach title to upper={\ ---\ },
    sharpish corners,
    enlarge by=0.5pt}

\newcounter{def}
\newtcolorbox[use counter=def]{definition}
	{center,%
    colframe =richblack!100,%
    colback=cultured,
    title={Definition \thetcbcounter},
    coltitle=richblack,
    attach title to upper={\ ---\ },
    sharpish corners,
    enlarge by=0.5pt}

\newcounter{examplecounter}
\newtcolorbox[use counter=examplecounter]{example}
	{center,%
    colframe =red!0,%
    colback=cultured,
    title={Example},
    coltitle=richblack,
    attach title to upper={\ ---\ },
    sharpish corners,
    enlarge by=0.5pt}

    

        
    
% Highlighters
\newcommand{\hldl}[1]{%
	\sethlcolor{lightdarkliver}%
	\hl{#1}
}
\newcommand{\hldb}[1]{%
    \sethlcolor{lightdelectricblue}%
    \hl{#1}%
}


% Images
\newcounter{figurecounter}
\setcounter{figurecounter}{1}

\newcommand{\img}[3]{
    \begin{figure}[h!]
        \centering
        \captionsetup{justification=centering,margin=0cm,labelformat=empty}
        \includegraphics[width=#2\linewidth]{./img/#1}
        \label{figure}
        \caption{\small\textbf{fig-\thefigurecounter} -- \textcolor{darkliver}{#3}}
    \end{figure}
    \addtocounter{figurecounter}{1}}

\newcommand{\imgr}[3]{
    \begin{wrapfigure}{r}{#2\textwidth}
        \centering
        \captionsetup{justification=centering,margin=0cm,labelformat=empty}
        \includegraphics[width=\linewidth]{./img/#1}
        \label{figure}
        \caption{\small \textbf{fig: \thefigurecounter} -- \textcolor{darkliver}{#3}}
    \end{wrapfigure}
    \addtocounter{figurecounter}{1}}

\newcommand{\imgl}[3]{
    \begin{wrapfigure}{l}{#2\textwidth}
        \centering
        \captionsetup{justification=centering,margin=0cm,labelformat=empty}
        \includegraphics[width=\linewidth]{./img/#1}
        \label{figure}
        \caption{\small \textbf{fig: \thefigurecounter} -- \textcolor{darkliver}{#3}}
    \end{wrapfigure}
    \addtocounter{figurecounter}{1}}

% New commands
\newenvironment{callout}
	{\begin{calloutbox}\color{charcoal}\textbf\textit}
	{\end{calloutbox}}

% for this file
\newcommand{\newpoint}[1]{\indent$\blacktriangleright$ \textbf{#1}}

\title{Reading Journal of Relativity \\ \large Summary of d'Inverno book on relativity}
\author{Amir H. Ebrahimnezhad}
\date{}
\begin{document}
            \maketitle
            \section*{Introduction}
            I have started reading relativity as a course in my undergraduate physics program in 2023. This is the highlights of the book by d'Inverno, which was the course material at that time. Beside that I woudl use other sources form time to time to write in more detail (or better details) the things that I was learning. With that said please enjoy and if you found any errors or considerations (or edits, more explanations etc.) feel free to contact me via thisismeamir@outlook.com.

            \tableofcontents
            \newpage
            \section{K-calculus}
                \newpoint{Model Building:} Model building is the essential activity of mathematical physics. It is at first a great thing that the world is made of mostly computable systems, and secondly that the formulas are not really complicated.
                \begin{qt}
                        The Most incomprehensible thing about the world is that it's comprehensible - \textit{Albert Einstein}
                \end{qt}
                \indent The very success of the activity of modelling has throughout the history of science, turned out to be counterproductive. For instance the Newtons theory had so much success that after two centuries, when at the end of nineteenth century it was vecoming increasingly clear that something was fundamentally wrong with the current theories, there was considerable reluctance to make any fundamental changes to them.
                \\
                \\
                \indent It eventually required the genius of Einstein to overthrow the prejudices of centuries and demonstrate in a number of simple thought experiments that some of the most cherished assumptions of Newtonian theory were untenable.
                \\
                \\
                \indent This he did in a number of brilliant papers written in 1905, proposing a theory which has become known today as the Special Theory of Relativity.
                \begin{callout}
                        Together with every Theory, there should go it's range of validity; The Newtones theory is not wrong. It is an excellentt theory in its range of validity.
                \end{callout}
                \newpoint{Newtonian Framework: } An event intuitively means something happening in a fairly limited region of space for a short duration of time, mathematically it becomes a point in space and an instance of time. Everything that happens in the universe is either an event or a collection of events.
                \\
                \\
                \indent One of the central assumptions of the Newtonian Framework is that two observers will, once they have synchronized their clocks, always agree about the time of an event, irrespective of their relative motion. This implies that, for all observers, time is an absolute concept. In particular, all observers can agree to synchronize their clocks so that they all agree on the time of an event.
                \begin{callout}
                        \begin{itemize}
                            \item We are assuming clocks tick the same way as eachother.
                            \item We are assuming the synchronization won't change by moving.
                            \item We are assuming that there takes no time for information about an event to arrive to us.
                        \end{itemize}
                \end{callout}
                \newpoint{The Principle of Special Relativity:} \textit{Q: \ } Are the laws of physics the same for all observers or are there preferred states of motion, preferred refrence systems, and so on? \ \textit{A:} Newton's theory postulates the existence of preferred frames of reference. The existence of this is essentially implied by the first law:
                \begin{callout}
                        Every body continues in its state of rest or uniform motion in a straight line, unless it is compelled to change that state by forces acting on it.
                \end{callout}
                Thus, there exists a privileged set of bodies, namely, those not acted on by forces. The frame of reference of a co-moving observer is called an \hldb{inertial frame}. The transformation which connects two inertial frames together is called \hldb{Galilean Transformation}. Consider to inertial frames one stationary and one moving with velocity $\vec v$ with respect to the other. Then the Galilean Transformations are as below:
                \begin{equation}
                        \begin{pmatrix}
                            x' \\ y' \\ z' 
                        \end{pmatrix}
                        =
                        \begin{pmatrix}
                            x \\ y \\ z
                        \end{pmatrix}
                        -
                        \begin{pmatrix}
                            v_x \\ v_y \\ v_z
                        \end{pmatrix}t
                \end{equation}
                \indent where $t$ is the time in both frames since in Newton's theory we can assume absolute time. From a mathematical viewpoint this means that Newton's laws must be \hldb{invariant} under a Galilean Transformation. Which means that, if one inertial observer carries out some dynamical experiments and discovers a physical law, then any other inertial observer performing the same experiments must discover the same law. Put another way, these laws must be invariant under Galilean transformation. 
                \\
                \\
                \indent Einstein realized that the principle as stated above is empty because there is no such thing as a purely dynamical experiment. Even on a very elementary level, any dynamical experiment we think of performing involves observation, and looking is a part of optics, not dynamics. Thus Einstein took the logical step of removing the restriction of dynamics in the principle and took the following as his first postulate:
                \begin{definition}\textit{
                    All inertial observers are equivalent.}
                \end{definition}
                \indent \newpoint{The Constancy of the Velocity of Light:} The concept of a "rigid ruler" is introduced to measure distances and times, and to map the events of the universe. However, as pointed out by Bondi, the concept of a "rigid rulers" is hard to consider. It's hard to define rigidity in a world of frames which with respect to others appears non-rigid. 
                \\
                \\
                \indent The approach of the $k$-calculus is to dispense with the rigid ruler and use radar methods for measuring distances. In the radar method an observer measures the distance of an object by sending out a light signal which is reflected off the object and received back by the observer. The distance is then simply defined as half the time difference between emission and reception. 
                \\
                \\
                \indent But why do we use light? The reason is that its velocity is independent of many things. There are many experiments that would confirm the constancy of the velocity of light. However these were not known to Einstein in 1905, who adopted the second postulate mainly on philosophical grounds. We state the second postulate in the following form.
                \begin{definition}
                    \textit{The velocity of light is the same in all inertial systems.}
                \end{definition}
                
                \img{figoneworldline}{0.8}{The world-lines of observer $A$ and $B$}
                \newpoint{The K-factor: } For simplicity, we shall begin by working in two dimensions, one spatial dimension and one time dimension. Thus, we consider a system of observers distributed along a straight line, each equipped with a clock and a flashlight. We plot the events they map in two dimensional space-time diagram. Let us assume we have two observers, $A$ at rest and $B$ moving away from $A$ with uniform speed. Then, in a space-time diagram, the world-line of $A$ will be represented by a vertical straight line, and the world-line of $B$ by a straight line at an angle to $A$'s, as shown in fig.1.
                \begin{callout}
                    With that said, since we are assuming $c=1$. A light signal would have a straight line making an angle of $\pi/4$
                \end{callout}
                \indent Assume that observer $A$, starts to send signals of light with a frequency in time (Taking $T$ amount of time in his/her clock to send another signal). And the other observer ($B$) is then receiving signals and sending one immediately after each. we can write the proportionality of the two tickings (moments between each signal) by the following form:
                \begin{equation}
                    T_B = k T_A
                \end{equation}
                \begin{callout}
                    Indeed, we will consider the constant $k$ to be independent of $T$. Indeed, we will go further and assume that it is independent of the point in space-time where the measurement is made and only depends on the relative speed of the two inertial observers. From a mathematical viewpoint, this is the assumption that space-time is \rich{homogeneous}. The principle should be the same from the prespective of $B$, which sees $A$ getting further from him. Thus the space woudl have to be \rich{isotropic}, the same in every direction.
                \end{callout}
                In the radar method, and observer finds coordinates of an event by shining light off it. Assuming the light signal was sent in $t_1$ and came back at $t_2$, one finds the coordinates of the event $P$ as:
                \begin{equation}
                    (t,x) = (\frac12(t_1+t_2),\frac12(t_2-t_1))
                \end{equation}
                \img{figtwokfactor}{0.8}{Relating the $k$-factor to the relative speed of separation.}
                \newpoint{Relative Speed of Two Inertial Observers: }Consder the configuration in fig.2, and assume that $A$ and $B$ synchronize thier clocks to zero when they cross at even $O$. After a time $T$, $A$ sends a signal to $B$, which is reflected back at event $P$. From $B$'s point of view, a light signal is sent to $A$ after a time lapse of $kT$ by $B$'s clock. Then using eq.3 we have:
                \begin{equation}
                    (t,x) = (\frac12(k^2 + 1)T,\frac12 (k^2-1)T)
                \end{equation}
                Then using the definition of speed we have:
                \begin{align*}
                    v=\frac xt =\frac{k^2-1}{k^2+1}
                \end{align*}
                Solving for $k$ we have:
                \begin{equation}
                    k = \left(\frac{1+v}{1-v}\right)^{\frac12}
                \end{equation}
                \newpoint{Composition Law for Velocities:} Now consider three inertial observers with different velocities (we consider $A$ to be stationary.). It follows immediately that:
                \begin{equation}
                    k_{AC} = k_{AB}k_{BC}
                \end{equation}
                Using eq.5, we find:
                \begin{equation}
                    v_{AC} =\frac{v_{AB}+v_{BC}}{1+v_{AB}v_{BC}}
                \end{equation}
                \indent This is the composition law for velocities. This formula has been verified experimentally to very high precision. Although the composition law for velocities is not simple, the one for $k$-factors is and, in special relativity, it is the $k$-factors which are directly measurable quantities. Note also that, formally, of we substitute $v_{BC}$, representing the speed of light signal relative to $B$, then the resulting speed of the light signal relative to $A$ is also $1$. In agreement with the constancy of the velocity of light postulate. 
                \\
                \\
                \indent From the composition law, we didn't mean to state that nothing can move faster than the speed of light in special relativity, but rather that the speed of light is a border which can not be crossed or even reached. More precisely, special relativity allows for the existence of three classes of particles:
                \begin{enumerate}
                \item Particles that move slower than the speed of light are called \textbf{subluminal} particles. They include material particles and elementary particles such as electrons and neutrons.
                \item Particles that move with the speed of light are called \textbf{luminal} particles. They include the carrier of the electromagnetic field interaction, the photon, other zero rest mass particles and, theoretically, the carrier of the gravitational field interaction, called the graviton.
                \item Particles that move faster than the speed of light are called \textbf{superluminal} particles or \textbf{tachyons}. There was some excitement in about th existence of them, but all attempts to detect them to date have failed. This suggests that either tachyons do not exist, or they appear to not interact with ordinary matter. This would seem to be just as well, for otherwise  they could be used to signal back into the past and so would appear to violate causality.
                \end{enumerate}
                \img{worldlinesimu}{0.5}{Relativity of simultaneity.}
                \img{figthreetrainexp}{0.8}{Photons emanating from the two sources.}
                \newpoint{The Relativity of Simultaneity: } Einstein realized the crucial role that the simultaneity plays in the theory. For instance he gave the following thought experiment:
                \\
                \\
                \indent Imagine a train travelling along a straight track with velocity $v$ relative to a stationary observer $A$ on the bank of the track. In the train, $B$ is an observer situated at the centre of one of the carriages. We assume that for an instance of time, using some electrical devices, two light signals are sent each from one of the ends of the carriage. From the configuration, it is clear that, according to observer $A$, the two  photons will be emitted simultaneously. However, from $A$'s point of view, $B$ is travelling towards the light emanating from light source at the front of carriage, and away from the one in the back. Hence observer $B$ will observe the photon from light coming from the source at the back hits the front of the train before the other photon strikes the back. This is accordance with the space-time diagram given above (fig.3) where $P$ is the photon hitting the back of the train abd $Q$ is the photon hitting the front. These are simultaneous for observer $A$ on the bank but $Q$ occurs before $P$ for the observer $B$ on the train.
                \\
                \\
                \newpoint{Causality and the Clock Paradox: } In newtonian theory the notion of absolute time, gives rise to the causality of the universe. However, the assumption that nothing can travel faster than the speed of light comes to save us in special relativity. Since all inertial observers agree on the speed of light, the future light cone does not depend on the particular choice of inertial observer but is invariantly defined, and similarly for the past light cone. The observer-independent concept of light cone this divides space into three regions. The future has parts where can be influenced by the event at present, which we call the causal future, the past which could have influenced the present moment, which are called causal past. And points off these two cones which are called, elsewhere (or better called else where and whens). Points that cannot influence or be influenced by the present (spatial and temporal present.).
                \\
                \\
                Consider three observers, with the relative velocity $v_{AC} =-v_{AB}$. Assume that $A$ and $B$ synchronize their clocks at $O$ and that $C$'s clock is synchronized with $B$'s at $P$. The total time that $A$ records between $O$ and $Q$ us therefore $(k+k^{-1})T$, for $k\not= 1$, this is grater than the combined time intervals. But should not tht etime lapse between the two events agree? This is one form of the so-called clock paradox.
                \\
                \\
                However, it is not really a paradox; rather, what it shows is that, in relativity, time, like distance, is a route-dependent quantity. The moral is that, in special relativity, time is a more difficult concept to work with than the absolute time of Newton.
                \\
                \\
                \newpoint{The Lorentz Transformations:} Assuming two observers, sending light signals and catching the reflected ones off of an event $P$ in the space-time, for observer $A$ we got, $t = \frac12(t_1+t_2), \ \ x=\frac12(t_2-t_1)$ where $t_1, t_2$ are the instances in the $A$'s system where the signal was sent and recieved respectievely. We can write this in an other form: $t_1 = t-x$ and $t_2 = t+x$. Having the same for observer $B$: $t_1' = t'-x',  \ \ t_2' = t'+x'$. But since there's a relation between the prime and normal variables we have:
                \begin{equation}
                    t'-x' = k(t-x) , \ \ \ \ t+x = k(t'+x')
                \end{equation}
                After some rearrangement and using equation[5] we have:
                \begin{equation}
                    t' = \frac{t-vx}{(1-v^2)^{\frac12}}, \ \ x' = \frac{x-vt}{(1-v^2)^{\frac12}}
                \end{equation}
                This is also referred to as a boost in the x-direction with speed v, since it takes one from $A$'s coordinates to $B$'s coordinates, and $B$ is moving away from $A$, with speed $v$. Some simple algebra reveals the result:
                \begin{equation}
                    x'^{2}-t'^{2} = x^2-t^2
                \end{equation} 
                This shows that the quantity $x^2-t^2$ is invariant under a special Lorentz Transformation or boost.

            \section{The Key Attributes of Special Relativity}
                \newpoint{Standard derivation:} We want a set of equations that transforms from one system to another, we write thus:
                \begin{equation}
                    \begin{bmatrix}
                        t' \\ x' \\ y'\\ z'
                    \end{bmatrix}
                    = L \begin{bmatrix}
                        t \\ x \\ y\\ z
                    \end{bmatrix}
                \end{equation}
                
                Note that if a spherical light signal was sent at the point where the two observers meet, they would always (regardless of their relative speed) see the signal as a sphere, or in other words:
                \begin{align*}
                    I &= x^2+y^2+z^2 - c^2t^2 = 0 \\
                    I'&= x'^2+y'^2+z'^2 - c^2t'^2 =0
                \end{align*}
                Which means:
                \begin{equation}
                    x^2+y^2+z^2 - c^2t^2 = x'^2+y'^2+z'^2 - c^2t'^2
                \end{equation}
                Assuming for simplicity that the velocity if only in the $x,x'$ direction:
                \begin{equation}
                    x^2-t^2 = x'^2-t'^2 , \ \ \ c =1
                \end{equation}
                For the equation to be contant (since we want it to be invariant of the movement) we have the general form:
                \begin{equation}
                    \begin{bmatrix}
                        x' \\ t' 
                    \end{bmatrix}  = \begin{bmatrix}
                            \cosh \alpha & - \sinh \alpha
                            \\ -\sinh\alpha & \cosh\alpha
                    \end{bmatrix}
                    \begin{bmatrix}
                        x\\ t\end{bmatrix}
                \end{equation}
                Since $x' = 0 $ in the other system is equal to $x = vt$,
                \begin{align*}
                    0 &= vt\cosh\alpha - t\sinh\alpha
                    \\
                    \tanh\alpha &= v
                \end{align*}
                and we can obtain:
                \begin{align}
                    \cosh\alpha &= \frac1{(1-\tanh^2\alpha)^{\frac12}} = \frac1{(1-v^2)^{\frac12}}\\
                    \sinh\alpha &= \frac{v}{(1-v^2)^{\frac12}}
                \end{align}
                Therefore:
                \begin{definition}
                    we define $\beta(v)$ as:
                    \begin{equation}
                        \beta(v) := \frac{1}{(1-\frac{v^2}{c^2})^{\frac12}}
                    \end{equation}
                \end{definition}
                \newpage
                \newpoint{Mathematical Properties of Lorentz Transformation}
                \begin{enumerate}
                    \item A boost along th $x$-axis of speed $v$ is equivalent to a hyperbolic rotation in $(x,t)$-space through an amount $\alpha$ (called the rapidity) given by $\tanh\alpha = v/c$.
                    \item If we consider $v$ to be very small compared with $c$, for which we use the notation $v<<c$, we restore the Galilean transformation.
                    \item To get from primed to unprimed system, is easily done by interchanging the primed and unprimed coordinates arn repaceing $v$ by $-v$.
                    \item Special Lorentz Transformations form a group:
                    \begin{itemize}
                        \item The identity element is given by $v=0$.
                        \item The inverse element is given by $-v$.
                        \item The product of two bossts with velocities $v$ and $v'$ is another boost with velocity $v''$ which is:
                        \begin{equation}
                            v'' = \frac{v + v'}{1+vv'/c}
                        \end{equation}
                        \item It is associative
                    \end{itemize}
                    \item The square of the infinitesimal interval between infinitesimally separated events is invariant under lorentz transformation:
                    \begin{equation}
                        ds^2 = c^2dt^2 - dx^2 - dy^2 - dz^2
                    \end{equation}
                \end{enumerate}
                \img{lengthcontraction}{0.9}{A rod length in different observers}
                \newpoint{Length Contraction:}
                Consider a rod fixed in $S'$ with endpoint $x'_A , x'_B$, In the $S$, the ends are varying in time and are related to the prime system via lorentz transformation:
                \begin{equation}
                    x'_A = \beta(x_A - vt_A), \ \ x'_B = \beta(x_B-vt_B)
                \end{equation}
                In order to measure the lengths of the rod acroding to $S$, we have to find the $x$-coordinates of the end points at the same time accordin to $S$. Of we denote the rest length, namely the length in $S'$ by:
                \begin{align*}
                    \ell_0 = x'_B - x'_A
                \end{align*}
                Then the length in $S$ is:
                \begin{align*}
                    \ell = x_B-x_A
                \end{align*}
                Then, we got:
                \begin{equation}
                    \boxed{
                        \ell = \beta^{-1}\ell_0
                    }
                \end{equation}
                This shows that the length of a body is reduced by a factor, which is called length contraction.
                \begin{callout}
                    Clearly, the body will have greatest length in its rest frame, in which case it is called the lrest length, or the proper length. Also note that the length approaches zero as the velocity approaches the velocity of light.
                \end{callout}
                \img{timedialationone}{0.8}{Successive events recorded by a clock fixed in $S'$}
                \newpoint{Time Dialation:} The time also changes according to the special relativity and the notion of lorentz transformation. For a moving body a clock is fixed at point $x'_A$. Using Lorentz transformation we have in $S$:
                \begin{equation}
                    t_1 = \beta(t'_1 + vx'_A/c^2), \ \ t_2 = \beta(t'_1 +T_0 + vx'_A/c^2)
                \end{equation}
                
                Thus the time interval for $S$ is:
                \begin{equation}
                    \boxed{
                        T = \beta T_0
                    }
                \end{equation}
                Thus, moving clocks go slow by a factos. This phenomenon is called time dialation. The fastest rate of a clock is in its rest frame and is called its proper rate. Again , the effect has a reciprocal nature.
                \img{timedialationtwo}{0.8}{Proper time recorded by an accelerated clock}
                We can also define an accelerated clock, a clock which is unaffected by te acceleration; or that its instantaneous rate depends only on its instantaneous speed, therefore we define the proper time:
                \begin{equation}
                    d\tau = \left(1-\frac{v(t)^2}{c^2}\right)^{1/2}dt 
                \end{equation}
                Which leads to:
                \begin{equation}
                    \boxed{
                        \tau = \int_{t_0}^{t_1} \left(1-\frac{v(t)^2}{c^2}\right)^{1/2}dt
                    }
                \end{equation}
                \newpoint{Transformation of Velocities}
                Consider a particle in motion with Cartesian components
                \begin{equation}
                    (u_1, u_2, u_3) =  \left(\frac{dx}{dt},\frac{dy}{dt},\frac{dz}{dt} \right)
                \end{equation}
                By taking the differentials of the Lorentz transformation we get:
                \begin{equation}
                    dt' = \beta(dt-vdx), \ \ dx' = \beta(dx-vdt), \ \ dy' =dy ,  \ \ dz' = dz 
                \end{equation}
                With that said the velocities would come out as
                \begin{align}
                    u_1' = \frac{dx'}{dt'} = \frac{u_1 - v}{ 1- vu_1}
                    \\
                    u_2' = \frac{dy'}{dt'} = \frac{u_2}{\beta(1-vu_1)}
                    \\
                    u_3' = \frac{dy'}{dt'} = \frac{u_3}{\beta(1-vu_1)}
                \end{align}
                Notice how the velocities in the denomentaor depends only on the velocity of the object and the reference frame in the direction of the frames motion. A more complete way to write the vewlocities from one frame to another is:
                `'


\end{document}