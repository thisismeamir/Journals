\documentclass[10pt,a4paper]{article}
\usepackage[utf8]{inputenc}
\usepackage{amsmath}
\usepackage{amsfonts}
\usepackage{amssymb}
\usepackage{url}
\usepackage{makeidx}
\usepackage{graphicx}
\usepackage{graphicx, adjustbox}
\usepackage{lmodern}
\usepackage{fourier}
\usepackage{float}
\usepackage{caption}
\usepackage{wrapfig}
\usepackage{mhchem}
\usepackage[left=2.5cm,right=2.5cm,top=2cm,bottom=3cm]{geometry}
\usepackage{multicol}
\usepackage{soul}



%Colors
\usepackage[dvipsnames]{xcolor}


\definecolor{black}{RGB}{0, 0, 0}
\definecolor{richblack}{RGB}{7, 14, 13}
\definecolor{charcoal}{RGB}{45, 67, 77}
\definecolor{delectricblue}{RGB}{93, 117, 131}
\definecolor{cultured}{RGB}{245, 245, 245}
\definecolor{lightgray}{RGB}{211, 216, 218}
\definecolor{silversand}{RGB}{190, 194, 198}
\definecolor{spanishgray}{RGB}{148, 150, 157}
\definecolor{darkliver}{RGB}{64, 63, 76}

\colorlet{lightdelectricblue}{delectricblue!30}
\colorlet{lightdarkliver}{darkliver!30}


%ColorDefines
\newcommand{\trueblack}[1]{\textcolor{black}{#1}}
\newcommand{\rich}[1]{\textcolor{richblack}{#1}}
\newcommand{\lightblack}[1]{\textcolor{charcoal}{#1}}
\newcommand{\lightrich}[1]{\textcolor{delectricblue}{#1}}
\newcommand{\liver}[1]{\textcolor{darkliver}{#1}}

%Boxes
\usepackage{tcolorbox}
\newtcolorbox{calloutbox}{center,%
    colframe =red!0,%
    colback=cultured,
    title={Callout},
    coltitle=richblack,
    attach title to upper={\ ---\ },
    sharpish corners,
    enlarge by=0.5pt}

\newtcolorbox[use counter=equation]{eq}{center,
	colframe =red!0,
	colback=cultured,
	title={\thetcbcounter},
	coltitle=richblack,
	detach title,
	after upper={\par\hfill\tcbtitle},
	sharpish corners,
    enlarge by=0.5pt }
    
\newtcolorbox{qt}{center,
	colframe=delectricblue,
	colback=white!0,
	title={\large "},
	coltitle=delectricblue,
	attach title to upper,
	after upper ={\large "},
	sharp corners,
	enlarge by=0.5pt,
	boxrule=0pt,
	leftrule=2pt}
	
\newtcolorbox{exc}{center,%
    colframe =red!0,%
    colback=darkliver!15,
    title={Excercise},
    coltitle=richblack,
    attach title to upper={\ ---\ },
    sharpish corners,
    enlarge by=0.5pt}
    
\newcounter{theo}
\newtcolorbox[use counter=theo]{theobox}
	{center,%
    colframe =red!0,%
    colback=cultured,
    title={Theorem \thetcbcounter},
    coltitle=richblack,
    attach title to upper={\ ---\ },
    sharpish corners,
    enlarge by=0.5pt}

\newcounter{def}
\newtcolorbox[use counter=def]{definition}
	{center,%
    colframe =richblack!100,%
    colback=cultured,
    title={Definition \thetcbcounter},
    coltitle=richblack,
    attach title to upper={\ ---\ },
    sharpish corners,
    enlarge by=0.5pt}

\newcounter{examplecounter}
\newtcolorbox[use counter=examplecounter]{example}
	{center,%
    colframe =red!0,%
    colback=cultured,
    title={Example},
    coltitle=richblack,
    attach title to upper={\ ---\ },
    sharpish corners,
    enlarge by=0.5pt}

    

        
    
% Highlighters
\newcommand{\hldl}[1]{%
	\sethlcolor{lightdarkliver}%
	\hl{#1}
}
\newcommand{\hldb}[1]{%
    \sethlcolor{lightdelectricblue}%
    \hl{#1}%
}


% Images
\newcounter{figurecounter}
\setcounter{figurecounter}{1}

\newcommand{\img}[3]{
    \begin{figure}[h!]
        \centering
        \captionsetup{justification=centering,margin=0cm,labelformat=empty}
        \includegraphics[width=#2\linewidth]{./img/#1}
        \label{figure}
        \caption{\small\textbf{fig-\thefigurecounter} -- \textcolor{darkliver}{#3}}
    \end{figure}
    \addtocounter{figurecounter}{1}}

\newcommand{\imgr}[3]{
    \begin{wrapfigure}{r}{#2\textwidth}
        \centering
        \captionsetup{justification=centering,margin=0cm,labelformat=empty}
        \includegraphics[width=\linewidth]{./img/#1}
        \label{figure}
        \caption{\small \textbf{fig: \thefigurecounter} -- \textcolor{darkliver}{#3}}
    \end{wrapfigure}
    \addtocounter{figurecounter}{1}}

\newcommand{\imgl}[3]{
    \begin{wrapfigure}{l}{#2\textwidth}
        \centering
        \captionsetup{justification=centering,margin=0cm,labelformat=empty}
        \includegraphics[width=\linewidth]{./img/#1}
        \label{figure}
        \caption{\small \textbf{fig: \thefigurecounter} -- \textcolor{darkliver}{#3}}
    \end{wrapfigure}
    \addtocounter{figurecounter}{1}}

% New commands
\newenvironment{callout}
	{\begin{calloutbox}\color{charcoal}\textbf\textit}
	{\end{calloutbox}}

% for this file
\newcommand{\newpoint}[1]{\indent$\blacktriangleright$ \textbf{#1}}


\title{General Relativity Lecture Notes}
\author{Kid A}
\begin{document}
\maketitle
     \section{Equivalency Principle}
     \begin{itemize}
          \item States that the gravitational force and acceleration can be equivalent. Assuming a person falling in gravitational force, the person would accelerate but feels nothing.
          \item We have shown that some of the general relativity predictions can be calculated through classical gravity and special relativity. But one can show for instance, red shifting, can be predicted using only the Equivalency principle.
          \item Another prediction was done already by Soldner using the assumption that light particles have a mass. the prediction showed that the curve passed by light is give by:
          \begin{equation}
               k = \frac 1R = \frac{-g}{c^2}
          \end{equation}
          \item \newpoint{Showing the curved trace of light using EP:} To show that light bends using equivalency perinciple assume a Local inertial frame (an elevator in free fall) we can show that for a light passing through a lova inertial frame the path is a straight line thus if it comes in with an angle of $\theta$ the coordinates are:
          \begin{align}
               x= ct\cos\theta \\
               y= ct\sin\theta
          \end{align}
          but for a stationary observer out side the local inertial frame what he sees includes the acceleration of the pther observer:
          \begin{align}
               x' = x = ct\cos\theta\\
               t' = ct\sin\theta - \frac12 g  t^2
          \end{align}
          Therefore:
          \begin{align}
               x'\tan\theta - \frac12 g\frac{c^2t^2}{c^2} = x'\tan\theta-\frac12 \frac{gx'}{c^2\cos^2\theta}
          \end{align}
          Now we have to calculate the $k$ at the point $x'=0$ we therefore have:
          \begin{equation}
               k  = \frac 1R = \left.\frac{\frac{d^2y'}{dx'^2}}{\left(1+\left(
                    \frac{dy'}{dx'}
               \right)^2\right)^{\frac32}}\right|_{x'=0}
          \end{equation}
          Leading to:
          \begin{align}
               &=\left.\frac{-g/c^2\cos^2\theta}{\left[ 1+(\tan\theta - gx'/c^2\cos^2\theta)^2\right]^{\frac32}}\right|_{x'=0} \\
               &=\frac{-g/c^2\cos^2\theta}{(1+\tan^2\theta)^{\frac32}}= -\frac{g}{c^2\cos^2\theta}\cos^2\theta= \frac{-g}{c^2}
          \end{align}
          What we learn from this is that there supposedly is a relation between graivty and geometry!
     \end{itemize}
     \section{Equation of motions:}
     Once again assume the local inertial reference frame with a prticle in it. the equations of motions for the particle ins the local inertial frame would be:
     \begin{equation}
          f^\alpha = \frac{d^2\xi^\alpha}{d\tau^2}
     \end{equation}
     where: $d\tau^2 = \eta_{\alpha\beta}d\xi^\alpha \xi^\beta$. We can write this in terms of an observer outside the interial reference frame watching the whole system in a free fall $\xi(x^\mu)$ to be:
     \begin{align}
          f^\alpha &= \frac{d}{d\tau}\left(\frac{dx^{\mu}}{d\tau}{d\xi^{\alpha}}\right)\\
          f^\alpha &= \frac{d^2 x^\mu}{d\tau^2}\frac{d\xi^\alpha}{dx^\mu} + \frac{d}{d\tau}(\frac{d\xi^\alpha}{dx^\sigma})\frac{dx^\mu}{d\tau}\\
          &= \frac{\xi^\alpha}{dx^\mu}\frac{d^2 x^\mu}{d\tau^2}+\frac{d}{dx^\nu}(\frac{d\xi^\alpha}{dx^\mu})\frac{dx^\nu}{d\tau}{dx^\mu}{d\tau} = 0\\
          &= \frac{d\xi^\alpha}{dx^\mu}\frac{d^2x^\mu}{d\tau^2}+ \frac{d^2\xi^\alpha}{dx^\mu dx^\nu} \frac{dx^\mu}{d\tau}\frac{dx^\nu}{d\tau} =0\\
          &=\frac{dx^\lambda}{d\xi^\alpha}\frac{d\xi^\alpha}{dx^\mu} + \frac{d^2\xi^\alpha}{dx^\mu dx^\nu}\frac{dx^\lambda}{d\xi^\alpha}\frac{dx^\mu}{d\tau}\frac{dx^\nu}{d\tau} = 0
     \end{align}
     DEfining
     \begin{equation}
          \Gamma^\lambda_{\mu\nu} = \frac{dx^\lambda}{d\xi^\alpha}{\frac{d^2 \xi^\alpha}{dx^\mu dx^\nu}}
     \end{equation}
     which is called the Christoffel Symbols. Therefire we can write:
     \begin{equation}
          \frac{d^2x^\lambda}{d\tau^2} + \Gamma^\lambda_{\mu\nu} \frac{dx^\mu}{d\tau}\frac{dx^\nu}{d\tau} = 0
     \end{equation}
     With that said one can write the metric in the new form:
     \begin{equation}
          g_{\mu\nu} = \eta_{\alpha\beta}\frac{\partial\xi^\alpha}{\partial x^\mu}\frac{\partial\xi^\beta}{\partial x^\nu}
     \end{equation}
     \section{Geodesic Equatio:}
     from the equation of motions we found a relation between general coordinate systems to be like:
     \begin{equation}
          d\xi^\alpha = \frac{\partial \xi^\alpha}{\partial x^\mu}dx^\mu
     \end{equation}
     Which lead us to the geodesic equation:
     \begin{equation}
          \frac{d^2x^\lambda}{d\tau^2} + \Gamma^\lambda_{\mu\nu} \frac{dx^\mu}{d\tau}\frac{dx^\nu}{d\tau} = 0
     \end{equation}
     with the propertime to be:
     \begin{equation}
          d\tau^2 = \eta_{\alpha\beta}d\xi^\alpha d\xi^\beta
     \end{equation}
     and the $\Gamma^\lambda_{\mu\nu}$ is the Christoffel Symbols or by other name (Affine Connections): 
     \begin{equation}
          \Gamma^\lambda_{\mu\nu} = \frac{\partial x^\lambda}{\partial \xi^\alpha} \frac{\partial^2 \xi^\alpha }{\partial x^\mu \partial x^\nu}
     \end{equation}
     \begin{callout}
          Christoffel symbols and affine connection are very closely related so much so that christoffel symbols are commonly also called "Connection coefficients". In a curved space, comparing one vector (or other mathematical object tensor forms) to another is not so straightforward a task as it is in nice, flat space. 
          \\
          The affine connection is the conceptual link between two very nearby points where the vectors you would like to compare reside. The christoffel symbols are the means of correcting your flat-space, naive differentiation to account for the curvature of the space in which you're doing your calculations,between those two points. So you could even call the Christoffel Symbolds "the same thing" as the affine connection, in a sense similar to calling a vector and its components in some particular coordinate system "the same thing".
     \end{callout}
     then we could define the metric as:
     \begin{equation}
          g_{\mu\nu}= \frac{\partial \xi^\alpha}{\partial x^\mu}\frac{\partial \xi^\beta}{\partial x^\nu} \eta_{\alpha\beta}
     \end{equation}
     
     then the proertime would transform into:
     \begin{equation}
          d\tau^2 = g_{\mu\nu} dx^\mu dx^\nu 
     \end{equation}
     \newpoint{Showing that the metric and Christoffel Symbols would be able to determine a local inertial frame properties:}
     \begin{align}
         \frac{\partial \xi^\beta}{\partial x^{\lambda}} \Gamma^\lambda_{\mu\nu} &= \frac{\partial \xi^\beta}{\partial x^\lambda}\frac{\partial x^\lambda}{\partial \xi^\alpha} \frac{\partial^2\xi^\alpha}{\partial x^\mu\partial x^\nu}
          \\
          &=\delta_\alpha^\beta \frac{\partial^2\xi^\alpha}{\partial x^\mu\partial x^\nu} = \frac{\partial^2\xi^\beta}{\partial x^\mu \partial x^\nu}
     \end{align}
     Then assuming the followin expansion for our local inertia frame coordinates we have:
     \begin{equation}
          \xi^\alpha (x) = a^\alpha + b_\mu^\alpha ( x^\mu - X^\mu) + \frac12 b_\lambda^\alpha \Gamma^\lambda_{\mu\nu} (x^\mu - X^\mu)(x^\nu - X^\nu) + \dots 
     \end{equation}
     then we would have:
     \begin{align*}
          a^\alpha  = \xi^\alpha (X)
          \\
          b^\alpha_\mu = \frac{\partial \xi^\alpha}{\partial x^\mu}(X)
     \end{align*}
     \newpoint{Showing that $g_{\mu\nu}$ can produce the affine conncetion: } We would start by derivating from the metric:
     \begin{align}
          g_{\mu\nu} = \frac{\partial \xi^\alpha}{\partial x^\mu}\frac{\partial \xi^\beta}{\partial x^\nu}\eta_{\alpha \beta}
          \\
          \frac{\partial g_{\mu\nu}}{\partial x^\lambda} &= \frac{\partial^2 \xi^\alpha}{\partial x^\lambda \partial x^\mu}\frac{\partial \xi^\beta}{\partial x^\nu}\eta_{\alpha\beta} + \frac{\partial \xi^\alpha}{\partial x^\mu}\frac{\partial^2\xi^\beta}{\partial x^\lambda\partial x^\nu}\eta_{\alpha\beta}\\
          &= \left(\Gamma^\rho_{\lambda\mu} \frac{\partial \xi^\alpha}{\partial x^\rho}\right)\frac{\partial \xi^\beta}{\partial x^\nu}\eta_{\alpha\beta} + \left(\Gamma^\rho_{\lambda\nu} \frac{\partial \xi^\alpha}{\partial x^\mu}\right)\frac{\partial \xi^\beta}{\partial x^\rho} \eta_{\alpha\beta} \\ 
          \Rightarrow \frac12 g^{\nu\eta}\left(\frac{\partial g_{\mu\nu}}{\partial x^\lambda} + \frac{\partial g_{\lambda\nu}}{\partial x^\mu}  - \frac{\partial g_{\mu\lambda}}{\partial x^\nu}\right)
          \\
          &= g_{\kappa\nu}\Gamma^\kappa_{\lambda\mu} + g_{\kappa\mu}\Gamma^\kappa_{\lambda\nu} + g_{\kappa\nu}\Gamma^\kappa_{\mu\lambda} + g_{\kappa\lambda}\Gamma^\kappa_{\mu\nu} - g_{\kappa\lambda}\Gamma^\kappa_{\nu\mu} - g_{\kappa\mu}\Gamma^\kappa_{\nu\lambda}\\
          &= (2g_{\kappa\nu}\Gamma^\kappa_{\lambda\mu}) \times g^{\nu\eta} = \Gamma^\eta_{\lambda\mu}
     \end{align}
     Therefore we have:
     \begin{equation}
          \Gamma^{\eta}_{\lambda\mu} = \frac12 g^{\nu\eta} \left( \partial_\lambda g_{\mu\nu} + \partial_mu g_{\lambda \nu} 0 \partial_\nu g_{\mu\lambda}\right)
     \end{equation}
\end{document}